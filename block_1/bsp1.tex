\subsection{solution}
\textbf{Proof}: Reduction from \textsc{Halting} problem to \textsc{%
Neg-Assignment}.

Let $(\Pi ,I)$ be an arbitrary instance of \textsc{Halting}, i.e. $\Pi $ \
is a program that takes an input. Based on this, we have to construct an
instance $(\Pi ^{\prime },I^{\prime },k)$ for \textsc{Neg-Assignment} by
setting $I^{\prime }=I$ and setting initially $k\geq 0$. Then $\Pi ^{\prime
} $ is defined as follows:

%TCIMACRO{%
%\TeXButton{Procedure}{\begin{center}
%\fbox{
%\begin{minipage}[c]{.9\linewidth}
%
%boolean $\Pi'$(String $S$) $\{$\\
%int $k := 0$;\\
%$call \Pi(S)$;\\
%$k := -1$;\\
%return \textit{true};\\
%$\}$\\
%
%\end{minipage}
%}
%\end{center}}}%
%BeginExpansion
\begin{center}
\fbox{
\begin{minipage}[c]{.9\linewidth}

boolean $\Pi'$(String $S$) $\{$\\
int $k := 0$;\\
$call \Pi(S)$;\\
$k := -1$;\\
return \textit{true};\\
$\}$\\

\end{minipage}
}
\end{center}%
%EndExpansion

Let $x=(\Pi ,I)$ be an instance of \textsc{Halting} and $R(x)=(\Pi ^{\prime
},I^{\prime },k)$ the resulting instance from the reduction. Then, we have
to show that $x$ is reducible to $R(x)$ by following equivalence relation:%
\begin{eqnarray*}
&&(\Pi ,I)\text{ is a positive instance of \textsc{Halting}}%
\quad\Leftrightarrow\quad (\Pi ^{\prime },I^{\prime },k)\text{ is apositive
instance} \\
&&\text{of the procedure \textsc{Neg-Assignment}.}
\end{eqnarray*}

(i.e. $\Pi $ halts on $I$ and assigns to $k$ a negative integer value$\quad
\Leftrightarrow\quad \Pi^{\prime }$ returns \textit{true}.)

$\Rightarrow :$ Suppose that $\left( \Pi ,I\right) $ is a positive instance
of \textsc{Halting}, i.e. $\Pi $ terminates on $I$.

Then by construction of $(\Pi ^{\prime },I^{\prime },k)$, the call of $\Pi
(I^{\prime })$ in program $\Pi ^{\prime }$ terminates since $I^{\prime }=I$.
Hence, $\Pi ^{\prime }$ halts on input $I^{\prime }$, assignes $k:=-1$ and
returns \textit{true} respectively.

Thus, $(\Pi ^{\prime },I^{\prime },k)$ is a positive instance of \textsc{%
Neg-Assignment}.

\bigskip

$\Leftarrow :$ Suppose that $\left( \Pi ^{\prime },I^{\prime },k\right) $ is
a positive instance of \textsc{Neg-Assignment}, i.e. $\Pi ^{\prime }$
returns \textit{true} after assigning a negative value to $k$. Since $\Pi
^{\prime }$ involves the call of $\Pi (I^{\prime })$ with $I^{\prime }=I$, $%
\Pi $ terminates on $I$ and $k$ will be set to $-1$. Hence, $(\Pi ,I)$ is a
positive instance of \textsc{Halting}.

\bigskip
