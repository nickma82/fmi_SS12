\subsection{solution}
\textbf{Proof:}

To show that \textsc{Neg-Assignment} is semi-decideable, we have to build a
procedure $\Pi ^{\prime }$ that satisfies following conditions:

\begin{itemize}
\item $\Pi ^{\prime }$ takes as input an instance of \textsc{Neg-Assignment}%
, i.e. $\left( \Pi ,I,k\right) $,

\item $\Pi ^{\prime }$ simulates a run of $\Pi $ on $I$ and assigns internal
the variable $k$ an integer value,

\item if the simulation reaches the assignment where $k$ will be set to a
negative value, then $\Pi ^{\prime }$ returns \textit{true},

\item if the simulation halts before reaching the negative value assignment
of $k$ (i.e. $k\geq 0$), then $\Pi ^{\prime }$ returns \textit{false},

\item if the simulation of $\Pi $ on $I$ does not terminate, then $\Pi
^{\prime }$ cannot return any value and thus, $k$ will never set to a
negative value.
\end{itemize}

\bigskip Hence, the termination of the program $\Pi $ is only guaranteed on 
\textit{positive instances}.

Then $\Pi ^{\prime }$ can be argued as a semi-decision procedure for \textsc{%
Neg-Assignment} as follows:

Let $x=\left( \Pi ,I,k\right) $ be an arbitrary instance of \textsc{%
Neg-Assignment}, then

\begin{description}
\item[\normalfont\slshape Case 1:] Suppose that $x$ is a positive\ instance
(a "yes"-instance), s.t. the program $\Pi $ on input $I$ terminates and $k$
will be assinged by a negative value. Then by construction of $\Pi ^{\prime
} $, the simulation in $\Pi ^{\prime }$ on $I^{\prime }=I$ will encounter an
assignment to $k$, where $k$ get a negative value (i.e. $k:=-1$) and thus, $%
\Pi ^{\prime }$ returns \textit{true}.

\item[\normalfont\slshape Case 2:] 

\item[\textit{a)}] Suppose that $x$ is a "no"-instance, i.e. the program $\Pi $ on
input $I$ halts without reaching the negative value assignment of $k$. Then
by construction of $\Pi ^{\prime }$, the simulation in $\Pi ^{\prime }$ on
input $I^{\prime }=I$ is not able to change the initial value assignment of $%
k$, with $k\geq 0$, to a negative value. Hence, $\Pi ^{\prime }$ returns 
\textit{false}.

\item[\textit{b)}] Suppose that $x$ is negative instance and $\Pi $ does not halt on
input $I$, i.e. $\Pi $ does not terminate (loops forever). Then the
simulation in $\Pi ^{\prime }$ on $I^{\prime }=I$ will also run forever and $%
k$ will never assigned with a negative integer value. Hence, $\Pi ^{\prime }$
will run forever on the negative instance $(\Pi ,I,k)$ and will never return
any output.
\end{description}

Thus, both behaviors of the negative cases describes a correct behavior for
a semi-decision procedure.

\medskip 

Co-problem of \textsc{Neg-Assignment}:
If \textsc{Neg-Assignment} is a decision problem, \textsc{Co-Neg-Assignment} 
is the complement of \textsc{Neg-Assignment}, and both \textsc{Neg-Assignment} 
and \textsc{Co-Neg-Assignment} are semi-decidable, then \textsc{Neg-Assignment} 
is decidable.


\bigskip 

