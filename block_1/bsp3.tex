\subsection{solution}

\textbf{Proof:}

To prove that \textsc{Subset-Sum} $\in $ NP, we have to define a
polynomially balanced and polynomially decidable certificate relation for 
\textsc{Subset-Sum}:

Let $S=\{a_{1},\ldots ,a_{n}\}$ be an arbitrary finite set of integer
numbers with $n\geq 1$ and let $t$ $\in \mathbb{N}$, s.t. $t\geq 1$.

Then the certificate relation for \textsc{Subset-Sum} is defined as follows:

\begin{equation*}
R_{S}=\{\langle S,t\rangle \mid \exists \,S^{\prime }\subseteq S\text{ s.t. }%
t=\tsum\limits_{a\in S^{\prime }}a\text{ and }t\geq 1\}.
\end{equation*}

We argue that $R_{S}$ is indeed a certificate relation for \textsc{Subset-Sum%
}, since following holds:

\begin{gather*}
S\text{ is a positive instance of \textsc{Subset-Sum}}\quad \Leftrightarrow 
\\
\exists \text{ a certificate subset }S^{\prime }\subseteq S\text{ of
non-negative integers, s.t. }\tsum\limits_{a\in S^{\prime }}a=t\quad
\Leftrightarrow  \\
\exists \,t\in \mathbb{N}^{+}\text{, s.t. }\langle S,t\rangle \in R_{S}\text{%
.}
\end{gather*}

\begin{itemize}
\item $R_{s}$ is \textit{polynomial balanced} since every subset $S^{\prime
}\subseteq S$ of integer values can be represented in linear space (array)
and the sum of elements is $t$, which uses only constant space.

\item $R_{s}$ is \textit{polynomial-time decidable} since the sum of all
elements of $S^{\prime }\subseteq S$ needs linear time. Hence, $S^{\prime }$
can be finished at most in polynomial time.
\end{itemize}

\bigskip 
