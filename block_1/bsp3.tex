\subsection{solution}
\textbf{Proof:}

To prove that \textsc{Subset-Sum} $\in $ NP, we have to define a
polynomially balanced and polynomially decidable certificate relation for 
\textsc{Subset-Sum}:

Let $S=\{a_{1},\ldots ,a_{n}\}$ be an arbitrary finite set of integer
numbers with $n\geq 1$ and let $t$ $\in \mathbb{N}$, s.t. $t\geq 1$.

Then the certificate relation for \textsc{Subset-Sum} is defined as follows:

\begin{equation*}
R_{S}=\{\langle S,t\rangle \mid \exists \,S^{\prime }\subseteq S\text{ s.t. }%
t=\tsum\limits_{a\in S^{\prime }}a\text{ and }t\geq 1\}.
\end{equation*}

We argue that $R_{S}$ is indeed a certificate relation for \textsc{Subset-Sum%
}, since following holds:

\begin{gather*}
S\text{ is a positive instance of \textsc{Subset-Sum}}\quad \Leftrightarrow 
\\
\exists \text{ a certificate subset }S^{\prime }\subseteq S\text{ of
non-negative integers, s.t. }\tsum\limits_{a\in S^{\prime }}a=t\quad
\Leftrightarrow  \\
\exists \,t\in \mathbb{N}^{+}\text{, s.t. }\langle S,t\rangle \in R_{S}\text{%
.}
\end{gather*}

\begin{itemize}
\item $R_{s}$ is \textit{polynomial balanced} since every subset $S^{\prime
}\subseteq S$ of integer values can be represented in linear space (array)
and the sum of elements is $t$, which uses only constant space.

\item $R_{s}$ is \textit{polynomial-time decidable} since the sum of all
elements of $S^{\prime }\subseteq S$ needs linear time. Hence, $S^{\prime }$
can be finished at most in polynomial time.
\end{itemize}

\bigskip 

% %\hfill \newline

% To prove that \textbf{SUBSET SUM} $\in$ \textbf{NP} we have to define a polynomially balanced and 
% polynomially decidable certificate relation for \textbf{SUBSET SUM}:
% \newline
% \newline
% Let $R=\{S'\subseteq S | S'\}$ be a subset of sum of S. \\
% This is the case if $\sum (S')$ of it's elements ($S'$) is exactly $t$.
% \newline\newline

% % $R$ is indeed a certificate relation through the construction: \\
% % $S'$ is a positive instance of \textbf{SUBSET SUM} $\leftrightarrow$ the 
% % sum of S' elements is exactly t $\leftrightarrow \exists S' \subseteq \in R$.
% % \newline
% % \newline
% % Now we show that the given relation is polynomially balanced:
% % \newline
% % S' is always $\leq$ then S because by definition a subset sum ia a subset or equal the 
% % original set of integer numbers.
% % \newline
% % \newline
% % Next we have to show that there exists a polynomial time algorithm to sum a given subset. 
% % For this we can simply sum all elements of S' which is possible in linear time.
% % \newline
% % \newline
% % Since $ R $ is polynomially balanced and polinomial-time decidable, it follows that \textbf{SUBSET SUM} is in NP.

% $R_s = \{\langle S,t\rangle\} \| \exists S' \subseteq S \text{ s.t. } t = \sum_{a \in S}a \text{ and } t \geq 1\}$

% \noindent $\Rightarrow$ 
% We argue that $R_s$ is indeed a certificate relation for \textit{SUBS.-SUM} 
% since following holds:

% \noindent $S$ is a positive instance of \textit{SUBS.-SUM} $\leftrightarrow \exists$ a certificate
% subset $S' \subseteq S$ of non-negative integers, s.t.
% $$\sum_{a \in S'}a = t \leftrightarrow \exists t \in N^{+}, s.t. \langle S,t\rangle\in R_s$$
