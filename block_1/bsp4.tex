\documentclass[a4paper,parskip=half]{scrartcl}
%%%%%%%%%%%%%%%%%%%%%%%%%%%%%%%%%%%%%%%%%%%%%%%%%%%%%%%%%%%%%%%%%%%%%%%%%%%%%%%%%%%%%%%%%%%%%%%%%%%%%%%%%%%%%%%%%%%%%%%%%%%%%%%%%%%%%%%%%%%%%%%%%%%%%%%%%%%%%%%%%%%%%%%%%%%%%%%%%%%%%%%%%%%%%%%%%%%%%%%%%%%%%%%%%%%%%%%%%%%%%%%%%%%%%%%%%%%%%%%%%%%%%%%%%%%%
\usepackage{eurosym}
\usepackage[utf8]{inputenc}
\usepackage[T1]{fontenc}
\usepackage{lmodern}
\usepackage[english]{babel}
\usepackage{graphicx}
\usepackage{scrpage2}
\usepackage{ifthen}
\usepackage[ruled, boxed]{algorithm2e}
\usepackage{amsmath}
\usepackage{amsthm}
\usepackage{amssymb}
\usepackage{units}
\usepackage{enumerate}

\setcounter{MaxMatrixCols}{10}
%TCIDATA{OutputFilter=Latex.dll}
%TCIDATA{Version=5.00.0.2606}
%TCIDATA{<META NAME="SaveForMode" CONTENT="1">}
%TCIDATA{BibliographyScheme=Manual}
%TCIDATA{LastRevised=Friday, April 01, 2011 07:06:22}
%TCIDATA{<META NAME="GraphicsSave" CONTENT="32">}

\newcounter{exercise}
\setcounter{exercise}{1}
\newenvironment{exercise}[1]{\large \textsf{\textbf{Problem \arabic{exercise}}} \ifthenelse{#1>0}{(#1 points)}{} \\ \normalsize \addtocounter{exercise}{1}}{\vspace{2ex}}
\newenvironment{solution}{\large \textsf{\textbf{Solution:}} \\ \normalsize}{\vspace{2ex}}
\newenvironment{remark}{\large \textsf{\textbf{Remark}} \\ \normalsize}{\vspace{2ex}}
\newtheorem{claim}{Claim}
\newtheorem{lemma}{Lemma}
\input{tcilatex}

\begin{document}


\thispagestyle{scrheadings} ~

\subsection{solution}
\hfill \newline
At first we have to check if \textbf{Set-Partition} is a membership of NP.%
\newline
This can be shown by using a simple guess \& check procedure:\newline

Guess an arbitrary subset $S_1\subseteq S$ and verify if the sum of the
elements in $S_1$ and in $S\backslash S_1$ are equal. The summing of the
elements in the subsets takes linear time in the size of $S$.

To show NP-hardness of \textbf{Set-Partition}, we reduce \textbf{%
Subset-Sum} to \textbf{Set-Partition}.

Let $S=\{b_{1},\ldots ,b_{n}\}$ be a set of integers and an integer $t$ be a target. Let $S^{\prime }$ be an instance of 
\textbf{Set-Partition}, such that $S^{\prime }=S\cup \{c\}$ with $c=M-2\cdot
t$ and $M=\tsum\limits_{i\in S}b_{i}$.

We have to show following equation:%
\begin{equation*}
(S,t)\in 
%TCIMACRO{\TeXButton{Subset-Sum}{\mbox{\textbf{Subset-Sum}}}}%
%BeginExpansion
\mbox{\textbf{Subset-Sum}}%
%EndExpansion
\Leftrightarrow S^{\prime }\in 
%TCIMACRO{\TeXButton{Set-Partition}{\mbox{\textbf{Set-Partition}}}}%
%BeginExpansion
\mbox{\textbf{Set-Partition}}%
%EndExpansion
\end{equation*}

\begin{proof}
\hfill\newline
$\Leftarrow :$ Suppose that $S^{\prime }=\{b_{1},\ldots ,b_{n+1}\}$ is a
positive instance of \textbf{Set-Partition}. Then there exists a subset $%
S_1\subseteq S^{\prime }$,which specify the partition, such that $S_2=S^{\prime
}\backslash S_1$ and the sum of the elements in\ $S_1$ and $S_2$ are equal.

Let $S^1=\tsum\limits_{i\in S_1}b_{i}$ and $S^2=\tsum\limits_{i\in B}b_{i}$
be the sum of the elements in the subsets. Since $c=M-2\cdot t$ and $c\in
S^{\prime }$, hence $c$ is in one of the two partitions.

Without loss of generality, suppose that $c\in S_1$. Then $N=M+c=M+M-2\cdot
t=2\cdot M-2\cdot t=\tsum\limits_{i\in S^{\prime }}b_{i}$. Since $S^1=S^2
$ and $N=S^1+S^2$, it follows that $S^1=S^2=\frac{N}{2}=$ $\frac{%
2\cdot M-2\cdot t}{2}=M-t$.

Since $c\in S_1$ and $c=M-2\cdot t$, we can conclude that $S^1=t+c$. If $c$
is in the fist part of the partition, then $S_1\backslash \{c\}$ is a subset
of $S$ that sum to $t$, and if $c$ is in the second part of the partition,
then $S_2\backslash \{c\}$ is a subset of elements of $S$ that sum to $t$.
Hence, $(S,t)$ is a positive instance of \textbf{Subset-Sum}.

$\Rightarrow :$ Suppose that $(S,t)$ is a positive instance of \textbf{%
Subset-Sum}. Then there exists a subset $T\subseteq S$ such that $%
\tsum\limits_{i\in T}b_{i}=t$.

We have to show that $S^{\prime }=S\cup \{c\}$ is a positive instance of 
\textbf{Set-Partition}. Thus, we have to find two subsets of $S^{\prime }$
such that $S^1=S^2$. This can be done easily by setting $S_1=T\cup \{c\}$.
Then $S^1=t+M-2\cdot t=M-t=\frac{N}{2}$, wich is exactly the half of the
sum of elements in $S^{\prime }$, i.e. $S^1=\frac{N}{2}=S^2$. This
implies that $S^{\prime }$ is partitionable. Thus, $S^{\prime }$ is a
positive instance of \textbf{Set-Partition}.
\end{proof}

Since \textbf{Set-Partition} $\in $ NP and is NP-hard, it follows that it is
NP-complete.
\end{solution}

\begin{remark}
To see how algorithms are represented in pseudo-code see how algorithms are
described in the book \^{a}\euro \oe Introduction to Algorithms\^{a}\euro 
\"{\i}\textquestiondown 
%TCIMACRO{\U{bd} }%
%BeginExpansion
$\frac12$
%EndExpansion
by Cormen, Leiserson, Rivest, and Stein. A link to the book (that let\^{a}%
\euro \texttrademark s you browse the pages of the book) can be found on
Moodle.

Also recall the $\mathcal{O}$-notation: Let $f$ and $g$ be two functions
from the natural numbers onto the positive real numbers. If there exist
constants $c>0$ and $n_{0}\geq 0$ such that for all $n\geq n_{0}$, $f(n)\leq
c\cdot g(n)$, then we write $f(n)=\mathcal{O}(g(n))$. Let $n$ be the size of
the cost matrix of a finite game. The running time $T(n)$ of an algorithm
takes time polynomial in $n$, if there exists an integer constant $d>0$ such
that $T(n)=\mathcal{O}(n^{d})$. For more details see Chapter 3 in the above
book.
\end{remark}

\end{document}

