\subsection{solution}

At first we have to check if \textbf{Set-Partition} is a membership of NP.%
\newline
This can be shown by using a simple guess \& check procedure:\newline

\noindent Guess an arbitrary subset $S_1\subseteq S$ and verify if the sum of the
elements in $S_1$ and in $S\backslash S_1$ are equal. The summing of the
elements in the subsets takes linear time in the size of $S$.

\noindent To show NP-hardness of \textbf{Set-Partition}, we reduce \textbf{Subset-Sum} to \textbf{Set-Partition}.\\

\noindent For easier reading we define S, the set of finite integer numbers of \textbf{SUBSET SUM}  
as $S_S$ and $S_P$ as the set of finite integer numbers of \textbf{PARTITION}.\\

\noindent Particularly let $S_S=\{a_{1},\ldots ,a_{n}\}$ be a set of integers and an integer $t$ be a target.
Let $S_P$ be an instance of \textbf{Set-Partition}, such that $S_P=S_S \cup \{c\}$ 
with $c=M-2.t$ and $M=\tsum\limits_{i\in S_S}a_{i}$.\\

\noindent We have to show the following equation:
$$(S_S,t)\in \textbf{SUBSET SUM} \Leftrightarrow S_P \in \textbf{PARTITION}$$

\subsubsection{Proof}
\hfill\newline
$\Leftarrow$ direction:\\ 
Suppose that $S_P=\{a_{1},\ldots ,a_{n+1}\}$ is a
positive instance of \textbf{PARTITION}. Then there exists a subset 
$S_1\subseteq S_P$,which specifies the partition, such that $S_2=S_P\backslash S_1$ 
and the sum of the elements in $S_1$ and $S_2$ are equal.

\noindent Let $S_{1sum}=\tsum\limits_{i\in S_1}a_{i}$ and 
$S_{2sum}=\tsum\limits_{i\in S_2}a_{i}$
be the sum of the elements in the subsets. Since $c=M-2\cdot t$ 
and $c\in S_P$, hence $c$ is in one of the two partitions.

\noindent Without loss of generality, suppose that $c\in S_1$. 
Then $N=M+c=M+M-2\cdot t=2\cdot M-2\cdot t=\sum \limits_{i\in S_p} a_{i}$.
Since $S_1=S_2$ and $N=S_1+S_2$, it follows 
that $S_1=S_2=\frac{N}{2}=$ $\frac{2\cdot M-2\cdot t}{2}=M-t$.

\noindent Since $c\in S_1$ and $c=M-2\cdot t$, we can conclude that $S_1=t+c$. If $c$
is in the fist part of the partition, then $S_1\backslash \{c\}$ is a subset
of $S_P$ that sums to $t$, and if $c$ is in the second part of the partition,
then $S_2\backslash \{c\}$ is a subset of elements of $S_S$ that sum to $t$.
Hence, $(S_S,t)$ is a positive instance of \textbf{SUBSET SUM}.\\

\noindent $\Rightarrow$ direction:\\
Suppose that $(S_S,t)$ is a positive instance of 
\textbf{SUBSET SUM}. Then there exists a subset $T\subseteq S_S$ such that 
$\sum \limits_{i\in T} a_{i}=t$.

\noindent We have to show that $S_P=S_S\cup \{c\}$ is a positive instance of 
\textbf{PARTITION}. Thus, we have to find two subsets of $S_P$
such that $S_{1sum}=S_{2sum}$. This can be done easily by setting $S_1=T\cup \{c\}$.
Then $S_{1sum}=t+M-2\cdot t=M-t=\frac{N}{2}$, wich is exactly the half of the
sum of elements in $S_P$, i.e. $S_{1sum}=\frac{N}{2}=S_{2sum}$. This
implies that $S_P$ is partitionable. Thus, $S_P$ is a
positive instance of \textbf{PARTITION}.\\


\noindent Since \textbf{PARTITION} $\in $ NP and is NP-hard, it follows that it is
NP-complete.