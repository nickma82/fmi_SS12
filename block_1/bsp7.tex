\subsection{solution}

\noindent \textbf{Argumentation:}

The restricted problem of \textsc{Vertex k-Coloring} to trees can be solved
in polynomial time.

Let $G_{T}=(V,E)$ be an arbitrary tree with the vertex set $V=\{v_{1},\ldots
,v_{n}\}$ and a mapping $c:V(G_{T})\rightarrow \{1,,\ldots ,k\}$ such that $%
V_{i}=\{v\in V(G_{T})\mid c(v)=i\}$ and $c(u)\neq c(v)$, whenever $u,v\in V$
are adjacent $\forall (u,v)\in E(G_{T})$. Let us denote $\langle
V_{i}\rangle $ the \textit{subgraph induced by }$V_{i}$ in $G_{T}.$ (Note: $%
\langle V_{i}\rangle $ is possiblely a disconnected subgraph of $G_{T}$.)
Depending on the graph property $\Pi _{T\text{ }}$for trees, which can be
enforced on each $\langle V_{i}\rangle $, there can be defined different
coloring concepts. If each $V_{i}$ is an \textit{independent set} for $1\leq
i\leq k$, then the coloring function $c$ is a proper $k$-coloring.

Then the \textit{maximum independent set of a tree} can be found in \textit{%
linear time}, by 
\begin{inparaenum}[\itshape 1\upshape)]
\item stripping off all the end-vertices (leaf nodes),
\item adding them to the independent set,
\item deleting the newly formed end-nodes and
\item repeating from the first step until the resulting tree is empty.
\end{inparaenum}Moreover if each $V_{i}$ induces a forest (i.e. each
connected component of $V_{i}$ is a tree), then the coloring function $c$ is
called a $k$\textit{-tree coloring}. Then every graph has a proper $k$%
-coloring if the integer value $k$ is large enough. Since $G_{T}$ is an
acyclic connected graph, then every vertex $v$ is pairwise different colored
to its predecessor node $\pi (v)=u$.

Then the \textit{predecessor subgraph} of a depth-first search, denoted as $%
G_{\pi }=(V,E_{\pi })$ with $E_{\pi }=\{(\pi (v),v)\mid v\in V$ and $\pi
(v)\neq $ \textsc{Null}$\}$, forms a \textit{depth-first forest} which is
composed of serveral \textit{depth-first trees}, since the search can be
repeated from several multiple sources. The edges in $E_{\pi }$ are also
called \textit{tree edges}.

The following pseudocode (\ref{alg_k-coloring_dfs}) is a modified version of
the basic depth-first-search algorithm.

%TCIMACRO{%
%\TeXButton{DFS-Algortim for k-Coloring}{\floatname{algorithm}{Algorithm}
%\algrenewcommand{\algorithmicrequire}{\textbf{Input:}}
%\algrenewcommand{\algorithmicensure}{\textbf{Output:}}
%\algrenewcommand{\algorithmicforall}{\textbf{for each}}
%
%\begin{algorithm}[H]
%\small
%\begin{algorithmic}[1]
%	\Require An undirected tree $G_{T} = (V,E)$
%	\Ensure A $k$-coloring for $G_{T}$.
%	\newline
%
%	\Procedure{\textsc{Dfs}}{$G_{T}$}
%		\ForAll{vertex $u \in V[G_{T}]$}
%			\State $c[u] \gets 0$ \Comment{paint initially all vertices with color $0$ s.t. $0 \notin \{1,\ldots, k\}$.}
%			\State $\pi[u] \gets \text{\textsc{null}}$ \Comment{set initially the predecessor of $u$ to \textsc{null}.}
%		\EndFor
%		\State $time \gets 0$ \Comment{initialize the global timestamp variable.}
%		\ForAll{vertex $u \in V[G_{T}]$}
%						\If{$c[u] = 0$}
%				\State \textit{call} \textsc{Dfs-Visit}($u$)
%			\EndIf
%		\EndFor
%	\EndProcedure
%	\newline
%
%	\Procedure{\textsc{Dfs-Visit}}{$u$}
%		\Statex\Comment{vertex $u$ with color $0$ has been discovered; then paint vertex $u$ with color $i \in \{1,\ldots,k\}$.}
%		\State $c[u] \gets i \in \{1,\ldots,k\} \text{ s.t. } c[u] \neq c[\pi[u]]$, i.e. $i \in \{1,\ldots,k\}\backslash \{c[\pi[u]]\}$
%		\State $d[u] \gets time \gets time + 1$ \Comment{update the discovering time.}
%		\ForAll{$v \in Adj[u]$} \Comment{explore edge $(u,v)$.}
%		   	\If {$c[v] = 0$}
%				\State $\pi[v] \gets u$
%				\State \textit{call} \textsc{Dfs-Visit($v$)}
%			\EndIf
%		 \EndFor
%		 \State $f[u] \gets time \gets time + 1$ \Comment{update the finishing time; $\Rightarrow$ search in $Adj[u]$ is finished.}
%	\EndProcedure	
%\end{algorithmic}
%\caption{\small Polynomial time DFS-Algorithm for $k$-Coloring of trees.}
%\label{alg_k-coloring_dfs}
%\end{algorithm}}}%
%BeginExpansion
\floatname{algorithm}{Algorithm}
\algrenewcommand{\algorithmicrequire}{\textbf{Input:}}
\algrenewcommand{\algorithmicensure}{\textbf{Output:}}
\algrenewcommand{\algorithmicforall}{\textbf{for each}}

\begin{algorithm}[H]
\small
\begin{algorithmic}[1]
	\Require An undirected tree $G_{T} = (V,E)$
	\Ensure A $k$-coloring for $G_{T}$.
	\newline

	\Procedure{\textsc{Dfs}}{$G_{T}$}
		\ForAll{vertex $u \in V[G_{T}]$}
			\State $c[u] \gets 0$ \Comment{paint initially all vertices with color $0$ s.t. $0 \notin \{1,\ldots, k\}$.}
			\State $\pi[u] \gets \text{\textsc{null}}$ \Comment{set initially the predecessor of $u$ to \textsc{null}.}
		\EndFor
		\State $time \gets 0$ \Comment{initialize the global timestamp variable.}
		\ForAll{vertex $u \in V[G_{T}]$}
						\If{$c[u] = 0$}
				\State \textit{call} \textsc{Dfs-Visit}($u$)
			\EndIf
		\EndFor
	\EndProcedure
	\newline

	\Procedure{\textsc{Dfs-Visit}}{$u$}
		\Statex\Comment{vertex $u$ with color $0$ has been discovered; then paint vertex $u$ with color $i \in \{1,\ldots,k\}$.}
		\State $c[u] \gets i \in \{1,\ldots,k\} \text{ s.t. } c[u] \neq c[\pi[u]]$, i.e. $i \in \{1,\ldots,k\}\backslash \{c[\pi[u]]\}$
		\State $d[u] \gets time \gets time + 1$ \Comment{update the discovering time.}
		\ForAll{$v \in Adj[u]$} \Comment{explore edge $(u,v)$.}
		   	\If {$c[v] = 0$}
				\State $\pi[v] \gets u$
				\State \textit{call} \textsc{Dfs-Visit($v$)}
			\EndIf
		 \EndFor
		 \State $f[u] \gets time \gets time + 1$ \Comment{update the finishing time; $\Rightarrow$ search in $Adj[u]$ is finished.}
	\EndProcedure	
\end{algorithmic}
\caption{\small Polynomial time DFS-Algorithm for $k$-Coloring of trees.}
\label{alg_k-coloring_dfs}
\end{algorithm}%
%EndExpansion

The algorithm (\ref{alg_k-coloring_dfs}) works as follows:

Lines 2--5 paint all vertices with the color $0\notin \{1,,\ldots ,k\}$ and
initialize their predecessor (parent node) to \textsc{Null}. Line 6
initializes the global time counter with $0$. The lines 7--11 check each
vertex in $V$, if a vertex $u$ with color $0$ is found, then this node will
be visited by calling \textsc{Dfs-Visit}. Every time when \textsc{Dfs-Visit}$%
(u)$ is called in line 9, vertex $u$ becomes root of a new tree in the
depth-first forest. Hence, when the procedure \textsc{Dfs} returns, every
node $u$ has been assigned with a discovery time $d[u]$ and a finishing time 
$f[u]$.

In \textsc{Dfs-Visit}$(u)$, the vertex $u$ is initially colored with color $%
0\notin \{1,,\ldots ,k\}$. The line 14 paints node $u$ with a color $i\in
\{1,\ldots ,k\}\text{ s.t. }c[u]\neq c[\pi \lbrack u]]$, i.e. $i\in
\{1,\ldots ,k\}\backslash \{c[\pi \lbrack u]]\}$. Afterwards, line 15
records the discovery time $d[u]$ by incrementing\ and saving the global 
\textit{time} variable. The lines 16--21 examine each vertex $v$ which is
adjacent to $u$ and visit $v$ recursively, if $v$ is colored with $0$. Since
in line 16 each vertex $v\in Adj(u)$ is considered, then each edge $(u,v)$
will be \textit{explored} by the depth-first search. After every edge
leaving $u$, with color $i\in \{1,\ldots ,k\}\backslash \{c[\pi \lbrack
u]]\} $, has been explored, finally the last line 22 update and records the
finishing time in $f[u]$.

The \textsc{Dfs} procedure takes a run time of $O(|V|)$, exclusive of the
time to execute the call of the \textsc{Dfs-Visit} procedure. The procedure 
\textsc{Dfs-Visit} will be called exactely once for each vertex $v\in V$. 
\textsc{Dfs-Visit} is invoked only on nodes which are colored with $0$ and
paint them subsequently with a color $i\in \{1,\ldots ,k\}\backslash \{c[\pi
\lbrack u]]\}$. During the execution of \textsc{Dfs-Visit}$(u)$, the
for-loop on lines 16--21 is executed $|Adj[v]|$ times and hence%
\begin{equation*}
O(|E|)=\dsum\limits_{v\in V}|Adj(v)|.
\end{equation*}

Then the total running time of \textsc{Dfs} is $O(|V|+|E|)$ which is in
polynomial time.

\bigskip
