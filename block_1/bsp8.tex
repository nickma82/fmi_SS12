\subsection{solution}
% %$$F=\bigwedge_{z=1}^{9}\Bigg(\bigvee_{y=1}^{9} v_{1,y,z}\Bigg)$$

% \subsubsection{Reduction}

% We have to construct a polynomial-time reduction from \textbf{N-Queens} to
% SAT. Let an arbitrary instance of \textbf{N-Queens} be given by n queens and
% a $n\times n$ chessboard. We construct a propositional formula $\psi $ (i.e.
% an instance of \textbf{SAT}) as follows.\newline
% First of all, we use the following propositional variables: $s_{i,j}$, where 
% $s_{i,j}$ stands for the cell $(i,j)$ of the $n\times n$ chessboard. The
% variable $s_{i,j}$ is assigned \textit{true} iff a queen is assigned to the
% cell $c(i,j)$. The different constraints of the puzzle can be expressed with
% this representation in the following way:

% \noindent Rule to assert that we get at most one queen in a column: 
% \begin{equation*}
% \psi_1 \equiv
% \bigwedge_{i=1}^{n}\bigwedge_{j=1}^{n}\bigwedge_{k=j+1}^{n}(\neg s_{i;j}
% \vee \neg s_{i;k})
% \end{equation*}

% \noindent Rule to assert that we get at most one queen in each row: 
% \begin{equation*}
% \psi_2 \equiv
% \bigwedge_{i=1}^{n}\bigwedge_{j=1}^{n}\bigwedge_{k=j+1}^{n}(\neg s_{j;i}
% \vee \neg s_{k;i})
% \end{equation*}

% \noindent Left-lower to right-upper diagonal lines are splitted in two
% parts: The bottom lines are characterized by a numer $d\in {0,...,n-2}$, and
% each line is compounded by the cells $c(i,j)$ such that $i-j=d$. Notice that
% the lines with only one cell (the corners) are removed.\newline
% Rule to assert that we get at most one queen in a diagonal line: 
% \begin{equation*}
% \psi _{3}\equiv
% \bigwedge_{d=0}^{n-2}\bigwedge_{j=1}^{n-d}\bigwedge_{k=j+1}^{n-2}(\lnot
% s_{d+j;i}\vee \lnot s_{d+k;k})
% \end{equation*}

% \noindent The remaining diagonal lines are characterized by $d \in {%
% -(n-2),...,-1}$ and their rule is defined by: 
% \begin{equation*}
% \psi_4 \equiv
% \bigwedge_{d=2-n)}^{-1}\bigwedge_{j=1}^{n+d}\bigwedge_{k=j+1}^{n+d}(\neg
% s_{j;j-d} \vee \neg s_{k;k-d})
% \end{equation*}

% \noindent Analogously the left-upper to right-lower diagonal lines are
% splitted in two parts: The bottom lines are charactarized by a number $d \in 
% {3,...,n+1}$ and the\newline
% rule to assert that we get at most one queen in a diagonal line: 
% \begin{equation*}
% \psi_5 \equiv
% \bigwedge_{d=3}^{n+1}\bigwedge_{j=1}^{d-1}\bigwedge_{k=j+1}^{d-1}(\neg
% s_{dj;d-j} \vee \neg s_{k;d-k})
% \end{equation*}

% \noindent The remaining diagonal lines are characterized by $d \in {%
% n+2,...,2n-1}$ and their rule is defined by: 
% \begin{equation*}
% \psi_6 \equiv
% \bigwedge_{d=n+2}^{2n-1}\bigwedge_{j=d-n}^{n}\bigwedge_{k=j+1}^{d-1}(\neg
% s_{j;d-j} \vee \neg s_{k;d-k})
% \end{equation*}

% \noindent Rule to assert that we get at least one queen in each column: 
% \begin{equation*}
% \psi_7 \equiv \bigwedge_{i=1}^{n}\bigvee_{j=i}^{n}s_{i;j}
% \end{equation*}
% wich ensures that there is exactly one queen in each column.

% \noindent So we get the conjunction of these seven formulae: 
% \begin{equation*}
% \varphi \equiv \psi _{1}\bigwedge \psi _{2}\bigwedge \psi _{3}\bigwedge \psi
% _{4}\bigwedge \psi _{5}\bigwedge \psi _{6}\bigwedge \psi _{7}\bigwedge 
% \end{equation*}%
% is in CNF form and each truth assignment which makes it true represents a
% solution to the \textbf{N-Queens} problem.

% \subsubsection{Proof sketch}

% $\Rightarrow $ direction:\newline
% Suppose that $n$ is a positive instance of \textbf{N-Queens} and now we have
% to show that $\varphi $ also is a positive instance of \textbf{SAT}.\newline
% The formula $\varphi $ eavaluates to \textit{true} iff all clauses $\psi _{i}
% $, $1\leq i\leq 7$, evaluates to \textit{true}. Therefore, we have to proof
% that $\exists $ a truth assignment $T$ and we have to show that all clauses
% evaluates to true under the truth assignment $T$. We define:\newline
% \begin{equation*}
% T(s_{i,j})=\left\{ 
% \begin{array}{cl}
% true & \text{ if }\exists \text{ a Queen }q\text{ on }c(i,j)\text{ and }%
% \#(\text{N-Queens})=n, \\ 
% false & \text{otherwise.}%
% \end{array}%
% \right. 
% \end{equation*}%
% $\psi _{1}$ evaluates to true if for each number $j\in {1,..,n}$ the number
% has been assigned at most once which is true for a positive instance of 
% \textbf{N-Queens}. In this case every number will be assigned, exactly once,
% therefore this clause evaluates to true.\newline
% Analogously this works for $\psi _{2}$.\newline
% $\psi _{3}$ evaluates to true if for each number $d\in {1,..,n-1}$ the
% number has been assigned at most once which is also true for a positive
% instance of \textbf{N-Queens}.\newline
% Analogously this works for $\psi _{4}to\psi _{6}$.\newline
% $\psi _{7}$ evaluates to true if for each number $j\in {1,..,n}$ the number
% has been assigned at least once which is true for a positive instance of 
% \textbf{N-Queens}.\newline

% \noindent $\Leftarrow $ direction:\newline
% Suppose we have a positive instance T of \textbf{SAT} that is a truth
% assignment for the propositional formula $\varphi $. We have to show that
% it's possible to derive a poitive instance of \textbf{N-Queens} out of $T$.%
% \newline
% While $T$ is a positive instance of \textbf{SAT}, each clause of $\varphi $
% will evaluate to \textit{true}. Thus we can derive a queen set on an $%
% n\times n$ chessboard by using the first 6 clauses. What we now have to show
% is that we can derive a \textbf{N-Queens} out if $T$ and that \textbf{%
% N-Queens} is exactly of size $n$.\newline
% By definition of clauses $\psi _{1}$ to $\psi _{6}$ we already made sure
% that there can be at most $1$ queen in each line, each row and each diagonal
% line. Because of cluase $\psi _{7}$ there must be at least one queen in each
% column and therefore we can be sure that \textbf{N-Queens} is of size $n$.

% \subsubsection{NP-complete}

% Yes, \textbf{N-Queens} is NP-complete because we have constructed a
% polynomial-time reduction of \textbf{N-Queens} problem to \textbf{SAT} and 
% \textbf{SAT} is NP-complete, which is a well-known fact.

% \bigskip

\subsubsection{Reduction}

We have to construct a polynomial-time reduction from \textbf{N-Queens} to
SAT. Let an arbitrary instance of \textbf{N-Queens} be given by n queens and
a $n\times n$ chessboard. We construct a propositional formula $\psi $ (i.e.
an instance of \textbf{SAT}) as follows.\newline
First of all, we use the following propositional variables: $s_{i,j}$, where 
$s_{i,j}$ stands for the cell $(i,j)$ of the $n\times n$ chessboard. The
variable $s_{i,j}$ is assigned \textit{true} iff a queen is assigned to the
cell $c(i,j)$. The different constraints of the puzzle can be expressed with
this representation in the following way:

\noindent Rule to assert that we get at most one queen in a column: 
\begin{equation*}
\psi_1 \equiv
\bigwedge_{i=1}^{n}\bigwedge_{j=1}^{n}\bigwedge_{k=j+1}^{n}(\neg s_{i;j}
\vee \neg s_{i;k})
\end{equation*}

\noindent Rule to assert that we get at most one queen in each row: 
\begin{equation*}
\psi_2 \equiv
\bigwedge_{i=1}^{n}\bigwedge_{j=1}^{n}\bigwedge_{k=j+1}^{n}(\neg s_{j;i}
\vee \neg s_{k;i})
\end{equation*}

\noindent Left-lower to right-upper diagonal lines are splitted in two
parts: The bottom lines are characterized by a numer $d\in {0,...,n-2}$, and
each line is compounded by the cells $c(i,j)$ such that $i-j=d$. Notice that
the lines with only one cell (the corners) are removed.\newline
Rule to assert that we get at most one queen in a diagonal line: 
\begin{equation*}
\psi _{3}\equiv
\bigwedge_{d=0}^{n-2}\bigwedge_{j=1}^{n-d}\bigwedge_{k=j+1}^{n-2}(\lnot
s_{d+j;i}\vee \lnot s_{d+k;k})
\end{equation*}

\noindent The remaining diagonal lines are characterized by $d \in {%
-(n-2),...,-1}$ and their rule is defined by: 
\begin{equation*}
\psi_4 \equiv
\bigwedge_{d=2-n)}^{-1}\bigwedge_{j=1}^{n+d}\bigwedge_{k=j+1}^{n+d}(\neg
s_{j;j-d} \vee \neg s_{k;k-d})
\end{equation*}

\noindent Analogously the left-upper to right-lower diagonal lines are
splitted in two parts: The bottom lines are charactarized by a number $d \in 
{3,...,n+1}$ and the\newline
rule to assert that we get at most one queen in a diagonal line: 
\begin{equation*}
\psi_5 \equiv
\bigwedge_{d=3}^{n+1}\bigwedge_{j=1}^{d-1}\bigwedge_{k=j+1}^{d-1}(\neg
s_{dj;d-j} \vee \neg s_{k;d-k})
\end{equation*}

\noindent The remaining diagonal lines are characterized by $d \in {%
n+2,...,2n-1}$ and their rule is defined by: 
\begin{equation*}
\psi_6 \equiv
\bigwedge_{d=n+2}^{2n-1}\bigwedge_{j=d-n}^{n}\bigwedge_{k=j+1}^{d-1}(\neg
s_{j;d-j} \vee \neg s_{k;d-k})
\end{equation*}

\noindent Rule to assert that we get at least one queen in each column: 
\begin{equation*}
\psi_7 \equiv \bigwedge_{i=1}^{n}\bigvee_{j=i}^{n}s_{i;j}
\end{equation*}
wich ensures that there is exactly one queen in each column.

\noindent So we get the conjunction of these seven formulae: 
\begin{equation*}
\varphi \equiv \psi _{1}\bigwedge \psi _{2}\bigwedge \psi _{3}\bigwedge \psi
_{4}\bigwedge \psi _{5}\bigwedge \psi _{6}\bigwedge \psi _{7}\bigwedge
\end{equation*}%
is in CNF form and each truth assignment which makes it true represents a
solution to the \textbf{N-Queens} problem.

\subsubsection{Proof sketch}

We have to show that following equantion holds:%
\begin{equation*}
\varphi \text{ is a positive instance of }\mathbf{N-Queens}\text{ }\quad \iff \quad
\varphi _{S}\text{ is a positive instance of }\mathbf{SAT}\text{.}
\end{equation*}
(i.e. both formuals $\varphi $ and $\varphi _{S}$ are \textit{%
equi-satisfiable}.)

\bigskip

\noindent Then:

\medskip

\noindent $\Rightarrow $ direction:\newline
Suppose that $\varphi $ is a positive instance of \textbf{N-Queens} and now
we have to show that $\varphi _{S}$ also is a positive instance of \textbf{%
SAT}.\newline
The formula $\varphi $ eavaluates to \textit{true} iff all clauses $\psi _{i}
$, $1\leq i\leq 7$, evaluates to \textit{true}. Therefore, we have to show,
that $\exists $ a truth assignment $T$ such that all clauses in $\varphi $
evaluates in  to true under the truth assignment $T$. We define $T$ as follows:%
\newline
\begin{equation*}
T(s_{i,j})=\left\{ 
\begin{array}{cl}
true & \text{ if }\exists \text{ a Queen }q\text{ on }c(i,j)\text{ and }%
\#(N-Queens)=n, \\ 
false & \text{otherwise.}%
\end{array}%
\right. 
\end{equation*}

$\psi _{1}$ evaluates to true if for each number $j\in {1,..,n}$ the number
has been assigned at most once which is true for a positive instance of 
\textbf{N-Queens}. In this case every number will be assigned, exactly once,
therefore this clause evaluates to true.

Analogously this works for $\psi _{2}$.

$\psi _{3}$ evaluates to true if for each number $d\in {1,..,n-1}$ the
number has been assigned at most once which is also true for a positive
instance of \textbf{N-Queens}.

Analogously this works for $\psi _{4}$ to $\psi _{6}$.

$\psi _{7}$ evaluates to true if for each number $j\in {1,..,n}$ the number
has been assigned at least once which is true for a positive instance of 
\textbf{N-Queens}. Finally, we have to show that $\varphi $ is equisatisfialbe to $%
\varphi _{S}\in \,$\textbf{SAT} by showing that there exists a function $\mu
:s_{i,j}\rightarrow v$, $\forall s_{i,j}\in \varphi $ and $\forall v\in
\varphi _{S}$, such that $\varphi _{S}$ evaluates to true.

\bigskip

\noindent $\Leftarrow $ direction:\newline
Suppose, we have a positive instance $\varphi _{S}$ of \textbf{SAT.} We have
to show that it's possible to derive a poitive instance $\varphi \in $ 
\textbf{N-Queens} out of a truth-assignment $T$.

Similar as above, we have to build here also a truth-assignment function $T$
such that, $\varphi _{S}$ evaluates to true. Since $\varphi _{S}$ is a
positive instance of \textbf{SAT}, each clause of $\varphi _{S}$ will
evaluate to \textit{true}.

Then we have to show that there exists a function $\mu :v\rightarrow s_{i,j}$%
, $\forall v\in \varphi _{S}$ and $\forall s_{i,j}\in \varphi $, such that $%
\varphi \in $ \textbf{N-Queens} evaluates to true.

Hence, we can derive a queen set on an $n\times n$ chessboard by using the
first 6 clauses. Finally, we have to show that we can derive indeed a
positive instance of \textbf{N-Queens} out of $\mu $ and that the instance
of \textbf{N-Queens} has an exactly size of $n$.

\medskip

By definition of clauses $\psi _{1}$ to $\psi _{6}$ we already made sure
that there can be at most $1$ queen in each line, each row and each diagonal
line. Because of clause $\psi _{7}$ there must be at least one queen in each
column and therefore we can be sure that \textbf{N-Queens} is of size $n$.

\subsubsection{NP-complete}

Yes, \textbf{N-Queens} is NP-complete because we have constructed a
polynomial-time reduction of \textbf{N-Queens} problem to \textbf{SAT} and 
\textbf{SAT} is NP-complete, which is a well-known fact.

\bigskip
