\subsection{solution}
This can be explained by using a small procedure (see Procedure \ref%
{Alg-N-Sorted-El}) for this problem.

\bigskip 
%TCIMACRO{%
%\TeXButton{Algorithm: N-SORTED-ELEMENTS}{\floatname{algorithm}{Procedure}
%\renewcommand{\algorithmicrequire}{\textbf{Input:}}
%\renewcommand{\algorithmicensure}{\textbf{Output:}}
%\renewcommand{\algorithmicforall}{\textbf{for each}}
%
%\begin{algorithm}[h]
%\small
%\caption{N-SORTED-ELEMENTS}
%\label{Alg-N-Sorted-El}
%\begin{algorithmic}
%\REQUIRE $L=(e_{1},\ldots , e_{n}), k\in \mathbb{N}^{+}$
%\ENSURE $\mathbf{true}$ or $\mathbf{false}$
%\STATE $i \leftarrow 1$
%\STATE $count \leftarrow 1$ \COMMENT{count variable for finding $n$-sorted elements of size $k$} 
%\FORALL{$i \;$ s.t. $\; 1 \leq i \leq |L| - 1$}
%	\IF{$\left(e(i) + 1\right) = e(i+1)$}
%			\STATE $count \leftarrow count + 1$
%		\IF{$count = k$}
%			\RETURN \textbf{true}
%		\ENDIF
%	\ENDIF
%\ENDFOR
%\RETURN \textbf{false}
%\end{algorithmic}
%\end{algorithm}}}%
%BeginExpansion
\floatname{algorithm}{Procedure}
\renewcommand{\algorithmicrequire}{\textbf{Input:}}
\renewcommand{\algorithmicensure}{\textbf{Output:}}
\renewcommand{\algorithmicforall}{\textbf{for each}}

\begin{algorithm}[h]
\small
\caption{N-SORTED-ELEMENTS}
\label{Alg-N-Sorted-El}
\begin{algorithmic}
\REQUIRE $L=(e_{1},\ldots , e_{n}), k\in \mathbb{N}^{+}$
\ENSURE $\mathbf{true}$ or $\mathbf{false}$
\STATE $i \leftarrow 1$
\STATE $count \leftarrow 1$ \COMMENT{count variable for finding $n$-sorted elements of size $k$} 
\FORALL{$i \;$ s.t. $\; 1 \leq i \leq |L| - 1$}
	\IF{$\left(e(i) + 1\right) = e(i+1)$}
			\STATE $count \leftarrow count + 1$
		\IF{$count = k$}
			\RETURN \textbf{true}
		\ENDIF
	\ENDIF
\ENDFOR
\RETURN \textbf{false}
\end{algorithmic}
\end{algorithm}%
%EndExpansion

The above procedure can be solved in logarithmic space in the size of the
input $I$ with $|I|=|L|$.

The procedure needs only one pointer to an element in the list $L$ and one
count variable of constant size. I.e. the pointer variable $i$ need $\log (n)
$ bits for its representation. Observe a very large list $L$ of non-negative
intergers. Since the memory of a program is limited to constant many
pointers, the pointer variable $i$ uses at most $O(\log _{2}|L|)$ bits of
memory.

\bigskip 


%%TCIMACRO{%
%%\TeXButton{algorithm: N-SORTED-ELEMENTS}{\begin{algorithm}
%%	\KwIn{non-empty list L, integer k $\geq$ 1}
%%	\KwOut{bool}
%%		int count $\leftarrow 1;$\\
%%		\For{i in 1 \KwTo n}{
%%			\If{$L[i]==L[i+1]$}{
%%				$count = count+1;$\\ }
%%			\Else {
%%				\If{$count==k$}
%%					return ("yes");\\ 
%%				\Else 
%%					\tcc{number of sorted elements was too small}
%%					$count = 1$;\\
%%				\ Fi
%%			}
%%		}
%%		return ("no");
%%	\caption{\textfb{N-SORTED-ELEMENTS}}
%%	\label{alg:n-sorted}
%%\end{algorithm}}}%
%%BeginExpansion
%\begin{algorithm}[H]
%	\KwIn{non-empty list L, integer k $\geq$ 1}
%	\KwOut{bool}
%		int count $\leftarrow 1;$\\
%		\For{i in 1 \KwTo n} {
% 			\If{$L[i]==L[i+1]$}{
%				$count = count+1;$\\ }
%			\Else {
%				\If{$count==k$}
%					{return ("yes");\\ }
%				\Else 
%					{\tcc{number of sorted elements was too small}
%					$count = 1$;\\}
%			}
%		}
%		return ("no")
%% 	\caption{\textfb{N-SORTED-ELEMENTS}}
%	\label{alg:n-sorted}
%\end{algorithm}%
%%EndExpansion
%
%The algorithm above requires logarithmic space in the sice of the length of the input list:\\
%\newline
%\begin{itemize}
% \item $ count needs log_2|n| space.$
%\end{itemize}
