\subsection{solution}
This can be explained by using a small procedure (see Procedure \ref%
{Alg-N-Sorted-El}) for this problem.

%TCIMACRO{%
%\TeXButton{Algorithm: N-SORTED-ELEMENTS}{\floatname{algorithm}{Procedure}
%\renewcommand{\algorithmicrequire}{\textbf{Input:}}
%\renewcommand{\algorithmicensure}{\textbf{Output:}}
%\renewcommand{\algorithmicforall}{\textbf{for each}}
%
%\begin{algorithm}[ht]
%\small
%\begin{algorithmic}
%	\Require $L=(e_{1},\ldots , e_{n}), k\in \mathbb{N}^{+}$
%	\Ensure $\mathbf{true}$ if $n$-sorted elements of size $k$ are found, $\mathbf{false}$ otherwise.
%	\newline
%	\State $i \leftarrow 1$
%	\State $count \leftarrow 1$ \Comment{count variable for finding $n$-sorted elements of size $k$.} 
%	\ForAll{$i \;$ s.t. $\; 1 \leq i \leq |L| - 1$}
%		\If{$\left(e(i) + 1\right) = e(i+1)$}
%			\State $count \leftarrow count + 1$
%		\Else
%			\If{$count = k$}
%				\Return \textbf{true}
%			\Else
%				\State $count \leftarrow 1$ \Comment{the number of sorted elements was $< k$; $\Rightarrow$ reset the count value.}
%			\EndIf
%		\EndIf
%	\EndFor\\
%	\Return \textbf{false}	
%\end{algorithmic}
%\caption{\small \textsc{N-Sorted-Elements} procedure.}
%\label{Alg-N-Sorted-El}
%\end{algorithm}}}%
%BeginExpansion
\floatname{algorithm}{Procedure}
\renewcommand{\algorithmicrequire}{\textbf{Input:}}
\renewcommand{\algorithmicensure}{\textbf{Output:}}
\renewcommand{\algorithmicforall}{\textbf{for each}}

\begin{algorithm}[ht]
\small
\begin{algorithmic}
	\Require $L=(e_{1},\ldots , e_{n}), k\in \mathbb{N}^{+}$
	\Ensure $\mathbf{true}$ if $n$-sorted elements of size $k$ are found, $\mathbf{false}$ otherwise.
	\newline
	\State $i \leftarrow 1$
	\State $count \leftarrow 1$ \Comment{count variable for finding $n$-sorted elements of size $k$.} 
	\ForAll{$i \;$ s.t. $\; 1 \leq i \leq |L| - 1$}
		\If{$\left(e(i) + 1\right) = e(i+1)$}
			\State $count \leftarrow count + 1$
		\Else
			\If{$count = k$}
				\Return \textbf{true}
			\Else
				\State $count \leftarrow 1$ \Comment{the number of sorted elements was $< k$; $\Rightarrow$ reset the count value.}
			\EndIf
		\EndIf
	\EndFor\\
	\Return \textbf{false}	
\end{algorithmic}
\caption{\small \textsc{N-Sorted-Elements} procedure.}
\label{Alg-N-Sorted-El}
\end{algorithm}%
%EndExpansion

The above procedure can be solved in logarithmic space in the size of the
input $I$ with $|I|=|L|$.\newline
\noindent The procedure needs only one pointer to an element in the list $L$
and one count variable of constant size. I.e. the pointer variable $i$ need $%
\log (n) $ bits for its representation. Observe a very large list $L$ of
non-negative intergers. Since the memory of a program is limited to constant
many pointers, the pointer variable $i$ uses at most $O(\log _{2}|L|)$ bits
of memory.

\bigskip


%%TCIMACRO{%
%%\TeXButton{algorithm: N-SORTED-ELEMENTS}{\begin{algorithm}
%%	\KwIn{non-empty list L, integer k $\geq$ 1}
%%	\KwOut{bool}
%%		int count $\leftarrow 1;$\\
%%		\For{i in 1 \KwTo n}{
%%			\If{$L[i]==L[i+1]$}{
%%				$count = count+1;$\\ }
%%			\Else {
%%				\If{$count==k$}
%%					return ("yes");\\ 
%%				\Else 
%%					\tcc{number of sorted elements was too small}
%%					$count = 1$;\\
%%				\ Fi
%%			}
%%		}
%%		return ("no");
%%	\caption{\textfb{N-SORTED-ELEMENTS}}
%%	\label{alg:n-sorted}
%%\end{algorithm}}}%
%%BeginExpansion
%\begin{algorithm}[H]
%	\KwIn{non-empty list L, integer k $\geq$ 1}
%	\KwOut{bool}
%		int count $\leftarrow 1;$\\
%		\For{i in 1 \KwTo n} {
% 			\If{$L[i]==L[i+1]$}{
%				$count = count+1;$\\ }
%			\Else {
%				\If{$count==k$}
%					{return ("yes");\\ }
%				\Else 
%					{\tcc{number of sorted elements was too small}
%					$count = 1$;\\}
%			}
%		}
%		return ("no")
%% 	\caption{\textfb{N-SORTED-ELEMENTS}}
%	\label{alg:n-sorted}
%\end{algorithm}%
%%EndExpansion
%
%The algorithm above requires logarithmic space in the sice of the length of the input list:\\
%\newline
%\begin{itemize}
% \item $ count needs log_2|n| space.$
%\end{itemize}
