\documentclass [11pt]{article}
\usepackage{latexsym}
\usepackage{amssymb}
\usepackage{epsfig} 
\usepackage{enumerate}

\usepackage{float}
\usepackage{xspace}
\usepackage{relsize}
\usepackage{ifthen}
\usepackage{xspace}
\usepackage{algpseudocode}
\usepackage[ruled, boxed]{algorithm2e}


\usepackage{lmodern}
\usepackage[english]{babel}
\usepackage{graphicx}
\usepackage{scrpage2}
\usepackage{ifthen}
\usepackage{amsmath}
\usepackage{amsthm}
\usepackage{amssymb}
\usepackage{units}
\usepackage{enumerate}

\setcounter{MaxMatrixCols}{10}

%\usepackage{graphicx}
%\usepackage{typearea}
%\usepackage{multicol}
%\usepackage{amsfonts}
%\usepackage[nounderscore]{syntax}
%\usepackage{paralist}
%\usepackage{tikz}
%\usetikzlibrary{positioning}
%\usepackage{url}
%\usepackage{xspace}
%\usepackage{bbm}
%\usepackage{listings}
%%\usepackage{MnSymbol}
%%\usepackage[ruled]{algorithm2e}
%\def\NN{{\ensuremath{\mathbbm{N}_0\xspace}}}
%\usepackage{wrapfig}
%\usepackage{graphicx}
%\usetikzlibrary{arrows,automata}
%\setlength{\textheight}{25cm}


   \textwidth      15cm
   \textheight     23cm
   \oddsidemargin 0.5cm
   \topmargin    -0.5cm
   \evensidemargin\oddsidemargin

 \newcommand{\nop}[1]{}


   \pagestyle{plain}
   \bibliographystyle{plain}


\title{Formale Methoden der Informatik \\
Block 1: Computability and Complexity }
\author{Exercises 1-10}
\date{SS 2012}


  \newtheorem{theorem}{Theorem}
  \newtheorem{lemma}[theorem]{Lemma}
  \newtheorem{corollary}[theorem]{Corollary}
  \newtheorem{proposition}[theorem]{Proposition}
  \newtheorem{conjecture}[theorem]{Conjecture}
  \newtheorem{definition}[theorem]{Definition}
  \newtheorem{example}[theorem]{Example}
  \newtheorem{remark}[theorem]{Remark}
  \newtheorem{exercise}[theorem]{Exercise}

  \newcommand{\ra}{\rightarrow}
  \newcommand{\Ra}{\Rightarrow}
  \newcommand{\La}{\Leftarrow}
  \newcommand{\la}{\leftarrow}
  \newcommand{\LR}{\Leftrightarrow}

  \renewcommand{\phi}{\varphi}
  \renewcommand{\theta}{\vartheta}


\newcommand{\ccfont}[1]{\protect\mathsf{#1}}
\newcommand{\NP}{\ccfont{NP}}

\newcommand{\NN}{\textbf{N}}
\newcommand{\IN}{\textbf{Z}}
\newcommand{\bigO}{\mathrm{O}}
\newcommand{\bigOmega}{\Omega}
\newcommand{\bigTheta}{\Theta}
\newcommand{\REACHABILITY}{\mbox{\bf REACHABILITY}}
\newcommand{\MAXFLOW}{\mbox{\bf MAX FLOW}}
\newcommand{\MAXFLOWD}{\mbox{\bf MAX FLOW(D)}}
\newcommand{\MAXFLOWSUB}{\mbox{\bf MAX FLOW SUBOPTIMAL}}
\newcommand{\MATCHING}{\mbox{\bf BIPARTITE MATCHING}}
\newcommand{\TSP}{\mbox{\bf TSP}}
\newcommand{\TSPD}{\mbox{\bf TSP(D)}}

\newcommand{\ThreeCol}{\mbox{\bf 3-COLORABILITY}}
\newcommand{\TwoCol}{\mbox{\bf 2-COLORABILITY}}
\newcommand{\kCol}{\mbox{\bf k-COLORABILITY}}
\newcommand{\HamPath}{\mbox{\bf HAMILTON-PATH}}
\newcommand{\HamCycle}{\mbox{\bf HAMILTON-CYCLE}}

\newcommand{\ONESAT}{\mbox{\bf 1-IN-3-SAT}}
\newcommand{\MONONESAT}{\mbox{\bf MONOTONE 1-IN-3-SAT}}
\newcommand{\kSAT}{\mbox{\bf k-SAT}}
\newcommand{\NAESAT}{\mbox{\bf NAESAT}}
\newcommand{\CLIQUE}{\textbf{CLIQUE}\xspace} 
\newcommand{\VC}{\textbf{VERTEX COVER}\xspace}



\renewcommand{\labelenumi}{(\alph{enumi})}

%%% useful macros for Turing machines:
\newcommand{\blank}{\sqcup}
\newcommand{\ssym}{\triangleright}
\newcommand{\esym}{\triangleleft}
\newcommand{\halt}{\mbox{h}}
\newcommand{\yess}{\mbox{``yes''}}
\newcommand{\nos}{\mbox{``no''}}
\newcommand{\lmove}{\leftarrow}
\newcommand{\rmove}{\rightarrow}
\newcommand{\stay}{-}
\newcommand{\diverge}{\nearrow}
\newcommand{\yields}[1]{\stackrel{#1}{\rightarrow}}

\newcommand{\HALTING}{\mbox{\bf HALTING}}

\newcommand{\true}{{\it true}}
\newcommand{\false}{{\it false}}


\newcommand{\samplesolution}[1]{\noindent {\bf Sample solution.}  #1}




%%%%%%%%%%%%%%%%%%%%%%%%%%%%%%%%%%%%%%%%%%%%%%%%%%%%%%%%%%%%%%%%%%%

\begin{document}


\maketitle


\begin{exercise}
  Consider the problem \textbf{PROCEDURE NEG-ASSIGNMENT}, which is defined
  as follows:

  \begin{center}
    \fbox{
      \begin{minipage}[c]{.9\linewidth}
        \textbf{PROCEDURE NEG-ASSIGNMENT}

        \medskip INSTANCE: A triple $(\Pi,I, k)$, where (i) $\Pi$ is a
        program that takes one string as input and outputs true or false, (ii) $I$ is a
        string, and (iii) $k$ is an integer variable used in program $\Pi$.     \\
        QUESTION: Does variable $k$ ever get assigned a negative value when the program $\Pi$ is executed with input $I$?

        
      \end{minipage}
    }
  \end{center}
  Prove that \textbf{NEG-ASSIGNMENT} is undecidable. Prove the undecidability
  by providing a reduction from the \textbf{HALTING} problem to
  \textbf{NEG-ASSIGNMENT}, and arguing that your reduction is correct.
\end{exercise}



\textbf{Solution:}\\
Since the TPL language describes a context free grammar, 
we can construct any program due to parsing of the production rules (nonterminals). Hence

\begin{small}
\begin{eqnarray*}
\mathcal{P} &\Rightarrow &\mathcal{P};\mathcal{P} \\
&\Rightarrow &\mathcal{V}:=\mathcal{E};\mathcal{P} \\
& \Rightarrow &x:=(\mathcal{E}\ \mathcal{B}\ \mathcal{E});\mathcal{P} \\
&\Rightarrow &x:=(\mathcal{V}+\mathcal{V});\mathcal{P} \\
&\Rightarrow &x:=x+y;\ \mathcal{P} \\
&\Rightarrow &x:=x + y; \IF\ \mathcal{E}\ \THEN\ \mathcal{P}\ \ELSE\ \mathcal{P}\ \FI\ \\
&\Rightarrow & x:=x + y; \IF\ (\mathcal{E}\ \mathcal{B}\ \mathcal{E})\ \THEN\ 
 \mathcal{P}\ \ELSE\ \mathcal{Q}\ \IF\ \\
&\Rightarrow & x:=x + y; \IF\ (\mathcal{V}<\mathcal{N})\ \THEN\ \mathcal{P}
  \ELSE\ \mathcal{Q}\ \FI\ \\
% 
&\Rightarrow & x:=x + y; \IF\ (x < 0)\ \THEN\ \mathcal{P}\ \ELSE\ \mathcal{Q%
 }\ \FI\ \\
% 
&\Rightarrow & x:=x + y; \IF\ (x < 0)\ \THEN\ \ABORT; \ELSE\ \mathcal{Q}\ \FI\ \\
% 
&\Rightarrow & x:=x + y; \IF\ (x < 0)\ \THEN\ \ABORT; \ELSE\ \WHILE\ \mathcal{E}
 \ \DO\ \mathcal{P}\ \OD\ \FI\ \\
% 
&\Rightarrow & x:=x + y; \IF\ (x < 0)\ \THEN\ \ABORT; \ELSE\ \WHILE\ (\mathcal{E%
 }\ \mathcal{B}\ \mathcal{E})\ \DO\ \mathcal{P}\ \OD\ \FI\ \\
% 
% &\Rightarrow & x:=x + y; \IF\ (x < 0)\ \THEN\ \ABORT; \ELSE\ \WHILE\ (\mathcal{E%
%  }\neq \mathcal{E})\ \DO\ \mathcal{P}\ \OD\ \FI\ \\
% % 
% &\Rightarrow & x:=x + y; \IF\ (x < 0)\ \THEN\ \ABORT; \ELSE\ \WHILE\ (\mathcal{V%
%  }\neq \mathcal{E})\ \DO\ \mathcal{P}\ \OD\ \FI\ \\
% % 
% &\Rightarrow & x:=x + y; \IF\ (x < 0)\ \THEN\ \ABORT; \ELSE\ \WHILE\ (\mathcal{V%
%  }\neq \mathcal{V})\ \DO\ \mathcal{P}\ \OD\ \FI\ \\
% 
&\Rightarrow & x:=x + y; \IF\ (x < 0)\ \THEN\ \ABORT; \ELSE\ \WHILE\ (x\neq y)
\ \DO\ \mathcal{P};\mathcal{P}\ \OD\ \FI\ \\
% 
&\Rightarrow & x:=x + y; \IF\ (x < 0)\ \THEN\ \ABORT; \ELSE\ \WHILE\ (x\neq y)
\ \DO\ \mathcal{E};\mathcal{P}\ \OD\ \FI\ \\
% 
&\Rightarrow & x:=x + y; \IF\ (x < 0)\ \THEN\ \ABORT; \ELSE\ \WHILE\ (x\neq y)
\ \DO\ (\mathcal{E}\ \mathcal{B}\ \mathcal{E});\mathcal{P}\ \OD\ \FI\ \\
% 
&\Rightarrow & x:=x + y; \IF\ (x < 0)\ \THEN\ \ABORT; \ELSE\ \WHILE\ (x\neq y)
\ \DO\ (\mathcal{V}+\mathcal{N});\mathcal{P}\ \OD\ \FI\ \\
% 
&\Rightarrow & x:=x + y; \IF\ (x < 0)\ \THEN\ \ABORT; \ELSE\ \WHILE\ (x\neq y)
\ \DO\ x := x + 1;\mathcal{P};\ \OD\ \FI\ \\
% 
&\Rightarrow & x:=x + y; \IF\ (x < 0)\ \THEN\ \ABORT; \ELSE\ \WHILE\ (x\neq y)
\ \DO\ x := x + 1;\mathcal{E};\ \OD\ \FI\ \\
% 
&\Rightarrow & x:=x + y; \IF\ (x < 0)\ \THEN\ \ABORT; \ELSE\ \WHILE\ (x\neq y)
\ \DO\ x := x + 1;(\mathcal{E}\ \mathcal{B}\ \mathcal{E});\ \OD\ \FI\ \\
% 
&\Rightarrow & x:=x + y; \IF\ (x < 0)\ \THEN\ \ABORT; \ELSE\ \WHILE\ (x\neq y)\ \DO
\ x := x + 1;(\mathcal{V} +\mathcal{N});\ \OD\ \FI\ \\
% 
&\Rightarrow & x:=x + y; \IF\ (x < 0)\ \THEN\ \ABORT; \ELSE\ \WHILE\ (x\neq y)\ \DO
\ x := x + 1;y := y + 2;\ \OD\ \FI\ \\
&&
\end{eqnarray*}

\end{small}



\begin{exercise}
  Prove that the problem \textbf{NEG-ASSIGNMENT} from Exercise 1 is semi-decidable. 
To this end, provide a semi-decision procedure and justify your solution. Additionally,  show that the co-problem of \textbf{NEG-ASSIGNMENT} is not semi-decidable.
\end{exercise}


\subsection{solution}
\textbf{Proof:}

To show that \textsc{Neg-Assignment} is semi-decideable, we have to build a
procedure $\Pi ^{\prime }$ that satisfies following conditions:

\begin{itemize}
\item $\Pi ^{\prime }$ takes as input an instance of \textsc{Neg-Assignment}%
, i.e. $\left( \Pi ,I,k\right) $,

\item $\Pi ^{\prime }$ simulates a run of $\Pi $ on $I$ and assigns internal
the variable $k$ an integer value,

\item if the simulation reaches the assignment where $k$ will be set to a
negative value, then $\Pi ^{\prime }$ returns \textit{true},

\item if the simulation halts before reaching the negative value assignment
of $k$ (i.e. $k\geq 0$), then $\Pi ^{\prime }$ returns \textit{false},

\item if the simulation of $\Pi $ on $I$ does not terminate, then $\Pi
^{\prime }$ cannot return any value and thus, $k$ will never set to a
negative value.
\end{itemize}

\bigskip Hence, the termination of the program $\Pi $ is only guaranteed on 
\textit{positive instances}.

Then $\Pi ^{\prime }$ can be argued as a semi-decision procedure for \textsc{%
Neg-Assignment} as follows:

Let $x=\left( \Pi ,I,k\right) $ be an arbitrary instance of \textsc{%
Neg-Assignment}, then

\begin{description}
\item[\normalfont\slshape Case 1:] Suppose that $x$ is a positive\ instance
(a "yes"-instance), s.t. the program $\Pi $ on input $I$ terminates and $k$
will be assinged by a negative value. Then by construction of $\Pi ^{\prime
} $, the simulation in $\Pi ^{\prime }$ on $I^{\prime }=I$ will encounter an
assignment to $k$, where $k$ get a negative value (i.e. $k:=-1$) and thus, $%
\Pi ^{\prime }$ returns \textit{true}.

\item[\normalfont\slshape Case 2:] 

\item[a.)] Suppose that $x$ is a "no"-instance, i.e. the program $\Pi $ on
input $I$ halts without reaching the negative value assignment of $k$. Then
by construction of $\Pi ^{\prime }$, the simulation in $\Pi ^{\prime }$ on
input $I^{\prime }=I$ is not able to change the initial value assignment of $%
k$, with $k\geq 0$, to a negative value. Hence, $\Pi ^{\prime }$ returns 
\textit{false}.

\item[b.)] Suppose that $x$ is negative instance and $\Pi $ does not halt on
input $I$, i.e. $\Pi $ does not terminate ($\Rightarrow $ loops forever).
Then the simulation in $\Pi ^{\prime }$ on $I^{\prime }=I$ will also run
forever and $k$ will never get assigned a negative value.Therefore, $\Pi
^{\prime }$ will never return any output

\item does not reach any output. Hence

\item . This means, that $\Pi $ loops forever, or it halts with an output
which is different to $\left( I+I\right) $. Hence, the simulation $\Pi
^{\prime }$ does not reach the output $\left( I^{\prime }+I^{\prime }\right) 
$ with $I^{\prime }=I$. This means, that the simulation $\Pi ^{\prime }$
either runs forever or it halts with output \textit{false}. Thus, both
behaviors for a semi-decision procedure in the case of a negative instance
are correct.
\end{description}

\bigskip



\begin{exercise}
  Give a formal proof that \textbf{SUBSET SUM} is in $\NP$, i.e.\, define a
  certificate relation and discuss that it is polynomially balanced and
  polynomial-time decidable.
  
   \smallskip
    
  \noindent In the \textbf{SUBSET SUM} problem we are given a finite set of integer numbers $S=\{a_1, a_2, \ldots, a_n\}$ and an integer number $t$. We ask whether there is a subset $S'\subseteq S$ whose elements sum is equal to $t$?
  
\end{exercise}


  \solution{
The sparse method is a decision procedure for equality logic that computes
equi-satisfiable formulas in propositional logic.

Consider formula $\varphi ^{E}$ in equation logic:%

\begin{displaymath}
  \varphi ^{E}: (x_1 \neq x_2 \lor x_2=x_3 ) \land \big[ (x_2 \neq x_4 \land x_3=x_4
  \land x_4=x_5)
  \lor (x_6 \neq x_5 \land x_6=x_7 \land x_7=x_3)\big]
\end{displaymath}

Then the sets of equality literals and disequality literals of $\varphi ^{E}$
are:

\begin{eqnarray*}
E_{=}
&=&%
\{x_{2}=x_{3,}x_{3}=x_{4},x_{4}=x_{5},x_{6}=x_{7},x_{7}=x_{3}\}
\\
E_{\neq } &=&\{x_{1}\neq x_{2},x_{2}\neq x_{4},x_{6}\neq x_{5}\}
\end{eqnarray*}

It is very often the case, that a given equality logic formula $\varphi ^{E}$
can be simplified. Before reducing $\varphi ^{E}$ to a propositional formula 
$\varphi ^{P}$, we have to do some preprocessing for simplifying the
equality formula $\varphi ^{E}$:

\begin{enumerate}
\item Construct an equality graph $G^{E}(\varphi ^{E})=(V,E_{=},E_{\neq })$:

%\enlargethispage{100cm}
% Start of code
\begin{tikzpicture}[>=latex',line join=bevel,]
%%
\node (x2) at (101bp,99.026bp) [draw,circle] {$x_2$};
  \node (x3) at (149.22bp,182.55bp) [draw,circle] {$x_3$};
  \node (x1) at (14.5bp,99.026bp) [draw,circle] {$x_1$};
  \node (x6) at (293.9bp,99.026bp) [draw,circle] {$x_6$};
  \node (x7) at (245.67bp,182.55bp) [draw,circle] {$x_7$};
  \node (x4) at (149.22bp,15.5bp) [draw,circle] {$x_4$};
  \node (x5) at (245.67bp,15.5bp) [draw,circle] {$x_5$};
  \draw [solid] (x5) ..controls (262.06bp,43.885bp) and (277.41bp,70.467bp)  .. (x6);
  \definecolor{strokecol}{rgb}{0.0,0.0,0.0};
  \pgfsetstrokecolor{strokecol}
  \draw (276.75bp,53.203bp) node {$\neq$};
  \draw [solid] (x2) ..controls (117.39bp,70.641bp) and (132.74bp,44.059bp)  .. (x4);
  \draw (132.08bp,61.323bp) node {$\neq$};
  \draw [dashed] (x3) ..controls (149.22bp,136.43bp) and (149.22bp,61.789bp)  .. (x4);
  \draw (144.22bp,99.069bp) node {=};
  \draw [solid] (x1) ..controls (45.08bp,99.026bp) and (70.32bp,99.026bp)  .. (x2);
  \draw (57.714bp,107.03bp) node {$\neq$};
  \draw [dashed] (x2) ..controls (117.39bp,127.41bp) and (132.74bp,153.99bp)  .. (x3);
  \draw (132.08bp,136.73bp) node {=};
  \draw [dashed] (x6) ..controls (277.51bp,127.41bp) and (262.16bp,153.99bp)  .. (x7);
  \draw (262.82bp,136.73bp) node {=};
  \draw [dashed] (x4) ..controls (182bp,15.5bp) and (212.69bp,15.5bp)  .. (x5);
  \draw (197.38bp,7.5bp) node {=};
  \draw [dashed] (x7) ..controls (212.79bp,182.55bp) and (181.84bp,182.55bp)  .. (x3);
  \draw (197.36bp,174.55bp) node {=};
%
\end{tikzpicture}

The equality graph has two contradictory cycles, $%
c_{1}=(x_{2},x_{3},x_{4})$ and $%
c_{2}=(x_{3},x_{4},x_{5},x_{6},x_{7})$.

\item 
Following strictly the rules of the simplification algorithm, the algorithm
replace in this case every literal in the equality formula $\varphi ^{E}$
with \textsc{True}. In order to transform $\varphi ^{E}$ into a
propositional formula $\varphi ^{P}$, we have to make a little trick to
enforce the application of the reduction algorithm. In this case we apply
only one step of the simplification algorithm and replace all literals at
once with \textsc{True}, which are not in the cycles $c_{1}$ and $c_{2}$. So
we get:

\begin{eqnarray*}
\varphi _{1}^{E}:(\text{\textsc{True}}\vee \text{\textsc{True}})\wedge [
(x_{2}\,{\neq}\,x_{4}\wedge x_{3}\,{=}\,x_{4}\wedge x_{4}\,{=}
\,x_{5})\vee \\
(x_{5}\,{\neq}\,x_{6}\wedge x_{6}\,{=}\,x_{7}\wedge x_{3}\,{=}\,x_{7})]
\end{eqnarray*}

\begin{equation*}
\Rightarrow \;\;\varphi _{1}^{E}:(x_{2}\,{\neq}\,x_{4}\wedge x_{3}\,{=}\,x_{4}\wedge x_{4}\,{=}
\,x_{5})\vee \\
(x_{5}\,{\neq}\,x_{6}\wedge x_{6}\,{=}\,x_{7}\wedge x_{3}\,{=}\,x_{7})
\end{equation*}

Then the equality graph $G^{E}(\varphi _{1}^{E})$ looks like:

%\enlargethispage{100cm}
% Start of code
\begin{tikzpicture}[>=latex',line join=bevel,]
%%
\node (x2) at (14.5bp,90.939bp) [draw,circle] {$x_2$};
  \node (x3) at (206.21bp,14.5bp) [draw,circle] {$x_3$};
  \node (x6) at (206.21bp,167.38bp) [draw,circle] {$x_6$};
  \node (x7) at (261.75bp,90.939bp) [draw,circle] {$x_7$};
  \node (x4) at (116.35bp,43.697bp) [draw,circle] {$x_4$};
  \node (x5) at (116.35bp,138.18bp) [draw,circle] {$x_5$};
  \draw [solid] (x5) ..controls (147.27bp,148.23bp) and (175.43bp,157.38bp)  .. (x6);
  \definecolor{strokecol}{rgb}{0.0,0.0,0.0};
  \pgfsetstrokecolor{strokecol}
  \draw (158.33bp,161.8bp) node {$\neq$};
  \draw [solid] (x2) ..controls (47.335bp,75.709bp) and (83.51bp,58.93bp)  .. (x4);
  \draw (69.423bp,75.319bp) node {$\neq$};
  \draw [dashed] (x3) ..controls (175.19bp,24.579bp) and (146.8bp,33.804bp)  .. (x4);
  \draw (164.08bp,38.162bp) node {$=$};
  \draw [dashed] (x6) ..controls (225.32bp,141.07bp) and (242.72bp,117.12bp)  .. (x7);
  \draw (227.01bp,124.12bp) node {$=$};
  \draw [dashed] (x4) ..controls (116.35bp,76.211bp) and (116.35bp,105.82bp)  .. (x5);
  \draw (111.35bp,90.991bp) node {$=$};
  \draw [dashed] (x7) ..controls (242.64bp,64.635bp) and (225.23bp,40.683bp)  .. (x3);
  \draw (226.95bp,57.678bp) node {$=$};
%
\end{tikzpicture}

Now we have only the contradictory cycle $c_{2}$, and we can simplify the 
graph $G^{E}(\varphi _{1}^{E})$ once more by replacing all literals with TRUE, 
which are not in the cycle $c_{2}$. So we get a new equality graph 
$G^{E}(\varphi _{2}^{E})$:

\begin{tikzpicture}[>=latex',line join=bevel,]
%%
\node (x3) at (104.36bp,14.5bp) [draw,circle] {$x_3$};
  \node (x6) at (104.36bp,167.38bp) [draw,circle] {$x_6$};
  \node (x7) at (159.9bp,90.939bp) [draw,circle] {$x_7$};
  \node (x4) at (14.5bp,43.697bp) [draw,circle] {$x_4$};
  \node (x5) at (14.5bp,138.18bp) [draw,circle] {$x_5$};
  \draw [solid] (x5) ..controls (45.422bp,148.23bp) and (73.58bp,157.38bp)  .. (x6);
  \definecolor{strokecol}{rgb}{0.0,0.0,0.0};
  \pgfsetstrokecolor{strokecol}
  \draw (56.479bp,161.8bp) node {$\neq$};
  \draw [dashed] (x4) ..controls (14.5bp,76.211bp) and (14.5bp,105.82bp)  .. (x5);
  \draw (9.5bp,90.991bp) node {$=$};
  \draw [dashed] (x3) ..controls (73.438bp,24.547bp) and (45.28bp,33.696bp)  .. (x4);
  \draw (62.381bp,38.115bp) node {$=$};
  \draw [dashed] (x6) ..controls (123.47bp,141.07bp) and (140.87bp,117.12bp)  .. (x7);
  \draw (125.16bp,124.12bp) node {$=$};
  \draw [dashed] (x7) ..controls (140.79bp,64.635bp) and (123.38bp,40.683bp)  .. (x3);
  \draw (125.1bp,57.678bp) node {$=$};
%
\end{tikzpicture}

Recall the theorem,%
\begin{equation*}
\varphi ^{E}\text{ is satisfiable }\Leftrightarrow e(\varphi ^{E})\AND B_{t}%
\text{ is satisfiable,}
\end{equation*}

where $e(\varphi ^{E})$ denotes the \textit{propositional skeleton} of $%
\varphi ^{E}$and $B_{t}$ is a formula that describes the \textit{%
transitivity constraints} (conjunctions of implications).

So $\varphi ^{P}=e(\varphi ^{E})\AND B_{t}$ is equi-satisfiable to $\varphi
^{E}$, iff $\varphi ^{E}$ is satisfiable.

\bigskip
\item First we construct the propositional skeleton of $\varphi _{2}^{E}$ by
replacing each atom of the form $x_{i}=x_{j}$ in $\varphi _{2}^{E}$ with $%
e_{i,j}$, such that:%
\begin{equation*}
e(\varphi _{2}^{E})=(e_{3,4}\AND e_{4,5})\OR(\lnot e_{5,6}\AND e_{1,6} \AND 
e_{3,7})
\end{equation*}

\item Construct the nonpolar equality graph $G_{NP}^{E}(\varphi _{2}^{E})$:

\begin{tikzpicture}[>=latex',line join=bevel,]
%%
\node (x3) at (104.36bp,14.5bp) [draw,circle] {$x_3$};
  \node (x6) at (104.36bp,167.38bp) [draw,circle] {$x_6$};
  \node (x7) at (159.9bp,90.939bp) [draw,circle] {$x_7$};
  \node (x4) at (14.5bp,43.697bp) [draw,circle] {$x_4$};
  \node (x5) at (14.5bp,138.18bp) [draw,circle] {$x_5$};
  \draw [solid] (x5) ..controls (45.422bp,148.23bp) and (73.58bp,157.38bp)  .. (x6);
  \draw [solid] (x4) ..controls (14.5bp,76.211bp) and (14.5bp,105.82bp)  .. (x5);
  \draw [solid] (x3) ..controls (73.438bp,24.547bp) and (45.28bp,33.696bp)  .. (x4);
  \draw [solid] (x6) ..controls (123.47bp,141.07bp) and (140.87bp,117.12bp)  .. (x7);
  \draw [solid] (x7) ..controls (140.79bp,64.635bp) and (123.38bp,40.683bp)  .. (x3);
%
\end{tikzpicture}

\item Make $G_{NP}^{E}(\varphi _{2}^{E})$ chordal, using elimination
ordering $(x_{3},x_{4},x_{5},x_{6},x_{7})$:

\begin{tikzpicture}[>=latex',line join=bevel,]
%%
\node (x3) at (104.36bp,14.5bp) [draw,circle] {$x_3$};
  \node (x6) at (104.36bp,167.38bp) [draw,circle] {$x_6$};
  \node (x7) at (159.9bp,90.939bp) [draw,circle] {$x_7$};
  \node (x4) at (14.5bp,43.697bp) [draw,circle] {$x_4$};
  \node (x5) at (14.5bp,138.18bp) [draw,circle] {$x_5$};
  \draw [solid] (x5) ..controls (44.939bp,148.07bp) and (73.303bp,157.29bp)  .. (x6);
  \draw [solid] (x5) ..controls (55.749bp,124.78bp) and (117.74bp,104.64bp)  .. (x7);
  \draw [solid] (x3) ..controls (73.438bp,24.547bp) and (45.28bp,33.696bp)  .. (x4);
  \draw [solid] (x3) ..controls (78.612bp,49.939bp) and (40.44bp,102.48bp)  .. (x5);
  \draw [solid] (x6) ..controls (123.47bp,141.07bp) and (140.87bp,117.12bp)  .. (x7);
  \draw [solid] (x4) ..controls (14.5bp,76.211bp) and (14.5bp,105.82bp)  .. (x5);
  \draw [solid] (x7) ..controls (140.79bp,64.635bp) and (123.38bp,40.683bp)  .. (x3);
%
\end{tikzpicture}

\item Generate the transitivity constraints in $B_{t}$ for every triangle in
the chordal graph $G_{NP}^{E}(\varphi _{2}^{E})$:

\begin{enumerate}
\item $B_{t}=$ \textsc{True}

\item For each triangle $(e_{i,j},e_{j,k},e_{i,k})$ in $G_{NP}^{E}(\varphi
_{2}^{E})$:%
\begin{eqnarray*}
B_{t}=((e_{3,4}\AND e_{4,5}\IMPL e_{3,5})\AND && \\
(e_{3,5}\AND e_{4,5}\IMPL e_{3,4})\AND && \\
(e_{3,4}\AND e_{3,5}\IMPL e_{4,5})\AND && \\
(e_{3,5}\AND e_{5,7}\IMPL e_{3,7})\AND && \\
(e_{3,7}\AND e_{3,5}\IMPL e_{5,7})\AND && \\
(e_{3,7}\AND e_{5,7}\IMPL e_{3,5})\AND && \\
(e_{5,6}\AND e_{6,7}\IMPL e_{5,7})\AND && \\
(e_{6,7}\AND e_{5,7}\IMPL e_{5,6})\AND && \\
(e_{5,6}\AND e_{5,7}\IMPL e_{6,7})) &&
\end{eqnarray*}

Hence, $\varphi ^{E}$ is satisfiable, iff $\varphi ^{P}=e(\varphi _{2}^{E})%
\AND B_{t}$ is satisfiable.
\end{enumerate}
\end{enumerate}


  }

\begin{exercise}
  \label{ex:partition}
  Formally prove that \textbf{PARTITION} is $\NP$-complete. For this you may use
  the fact that \textbf{SUBSET SUM} is $\NP$-complete.   
  
  \smallskip
  \noindent In the \textbf{PARTITION} problem we are given a finite set of integers $S=\{a_1, a_2, \ldots, a_n\}$. We ask whether the set $S$ can be partitioned into two sets $S_1, S_2$ such that the sum of the numbers in $S1$ equals the sum of the numbers in $S_2$? 
  

\end{exercise}


\documentclass[a4paper,parskip=half]{scrartcl}
%%%%%%%%%%%%%%%%%%%%%%%%%%%%%%%%%%%%%%%%%%%%%%%%%%%%%%%%%%%%%%%%%%%%%%%%%%%%%%%%%%%%%%%%%%%%%%%%%%%%%%%%%%%%%%%%%%%%%%%%%%%%%%%%%%%%%%%%%%%%%%%%%%%%%%%%%%%%%%%%%%%%%%%%%%%%%%%%%%%%%%%%%%%%%%%%%%%%%%%%%%%%%%%%%%%%%%%%%%%%%%%%%%%%%%%%%%%%%%%%%%%%%%%%%%%%
\usepackage{eurosym}
\usepackage[utf8]{inputenc}
\usepackage[T1]{fontenc}
\usepackage{lmodern}
\usepackage[english]{babel}
\usepackage{graphicx}
\usepackage{scrpage2}
\usepackage{ifthen}
\usepackage[ruled, boxed]{algorithm2e}
\usepackage{amsmath}
\usepackage{amsthm}
\usepackage{amssymb}
\usepackage{units}
\usepackage{enumerate}

\setcounter{MaxMatrixCols}{10}
%TCIDATA{OutputFilter=Latex.dll}
%TCIDATA{Version=5.00.0.2606}
%TCIDATA{<META NAME="SaveForMode" CONTENT="1">}
%TCIDATA{BibliographyScheme=Manual}
%TCIDATA{LastRevised=Friday, April 01, 2011 07:06:22}
%TCIDATA{<META NAME="GraphicsSave" CONTENT="32">}

\newcounter{exercise}
\setcounter{exercise}{1}
\newenvironment{exercise}[1]{\large \textsf{\textbf{Problem \arabic{exercise}}} \ifthenelse{#1>0}{(#1 points)}{} \\ \normalsize \addtocounter{exercise}{1}}{\vspace{2ex}}
\newenvironment{solution}{\large \textsf{\textbf{Solution:}} \\ \normalsize}{\vspace{2ex}}
\newenvironment{remark}{\large \textsf{\textbf{Remark}} \\ \normalsize}{\vspace{2ex}}
\newtheorem{claim}{Claim}
\newtheorem{lemma}{Lemma}
\input{tcilatex}

\begin{document}


\thispagestyle{scrheadings} ~

\subsection{solution}
\hfill \newline
At first we have to check if \textbf{Set-Partition} is a membership of NP.%
\newline
This can be shown by using a simple guess \& check procedure:\newline

Guess an arbitrary subset $S_1\subseteq S$ and verify if the sum of the
elements in $S_1$ and in $S\backslash S_1$ are equal. The summing of the
elements in the subsets takes linear time in the size of $S$.

To show NP-hardness of \textbf{Set-Partition}, we reduce \textbf{%
Subset-Sum} to \textbf{Set-Partition}.

Let $S=\{b_{1},\ldots ,b_{n}\}$ be a set of integers and an integer $t$ be a target. Let $S^{\prime }$ be an instance of 
\textbf{Set-Partition}, such that $S^{\prime }=S\cup \{c\}$ with $c=M-2\cdot
t$ and $M=\tsum\limits_{i\in S}b_{i}$.

We have to show following equation:%
\begin{equation*}
(S,t)\in 
%TCIMACRO{\TeXButton{Subset-Sum}{\mbox{\textbf{Subset-Sum}}}}%
%BeginExpansion
\mbox{\textbf{Subset-Sum}}%
%EndExpansion
\Leftrightarrow S^{\prime }\in 
%TCIMACRO{\TeXButton{Set-Partition}{\mbox{\textbf{Set-Partition}}}}%
%BeginExpansion
\mbox{\textbf{Set-Partition}}%
%EndExpansion
\end{equation*}

\begin{proof}
\hfill\newline
$\Leftarrow :$ Suppose that $S^{\prime }=\{b_{1},\ldots ,b_{n+1}\}$ is a
positive instance of \textbf{Set-Partition}. Then there exists a subset $%
S_1\subseteq S^{\prime }$,which specify the partition, such that $S_2=S^{\prime
}\backslash S_1$ and the sum of the elements in\ $S_1$ and $S_2$ are equal.

Let $S^1=\tsum\limits_{i\in S_1}b_{i}$ and $S^2=\tsum\limits_{i\in B}b_{i}$
be the sum of the elements in the subsets. Since $c=M-2\cdot t$ and $c\in
S^{\prime }$, hence $c$ is in one of the two partitions.

Without loss of generality, suppose that $c\in S_1$. Then $N=M+c=M+M-2\cdot
t=2\cdot M-2\cdot t=\tsum\limits_{i\in S^{\prime }}b_{i}$. Since $S^1=S^2
$ and $N=S^1+S^2$, it follows that $S^1=S^2=\frac{N}{2}=$ $\frac{%
2\cdot M-2\cdot t}{2}=M-t$.

Since $c\in S_1$ and $c=M-2\cdot t$, we can conclude that $S^1=t+c$. If $c$
is in the fist part of the partition, then $S_1\backslash \{c\}$ is a subset
of $S$ that sum to $t$, and if $c$ is in the second part of the partition,
then $S_2\backslash \{c\}$ is a subset of elements of $S$ that sum to $t$.
Hence, $(S,t)$ is a positive instance of \textbf{Subset-Sum}.

$\Rightarrow :$ Suppose that $(S,t)$ is a positive instance of \textbf{%
Subset-Sum}. Then there exists a subset $T\subseteq S$ such that $%
\tsum\limits_{i\in T}b_{i}=t$.

We have to show that $S^{\prime }=S\cup \{c\}$ is a positive instance of 
\textbf{Set-Partition}. Thus, we have to find two subsets of $S^{\prime }$
such that $S^1=S^2$. This can be done easily by setting $S_1=T\cup \{c\}$.
Then $S^1=t+M-2\cdot t=M-t=\frac{N}{2}$, wich is exactly the half of the
sum of elements in $S^{\prime }$, i.e. $S^1=\frac{N}{2}=S^2$. This
implies that $S^{\prime }$ is partitionable. Thus, $S^{\prime }$ is a
positive instance of \textbf{Set-Partition}.
\end{proof}

Since \textbf{Set-Partition} $\in $ NP and is NP-hard, it follows that it is
NP-complete.
\end{solution}

\begin{remark}
To see how algorithms are represented in pseudo-code see how algorithms are
described in the book \^{a}\euro \oe Introduction to Algorithms\^{a}\euro 
\"{\i}\textquestiondown 
%TCIMACRO{\U{bd} }%
%BeginExpansion
$\frac12$
%EndExpansion
by Cormen, Leiserson, Rivest, and Stein. A link to the book (that let\^{a}%
\euro \texttrademark s you browse the pages of the book) can be found on
Moodle.

Also recall the $\mathcal{O}$-notation: Let $f$ and $g$ be two functions
from the natural numbers onto the positive real numbers. If there exist
constants $c>0$ and $n_{0}\geq 0$ such that for all $n\geq n_{0}$, $f(n)\leq
c\cdot g(n)$, then we write $f(n)=\mathcal{O}(g(n))$. Let $n$ be the size of
the cost matrix of a finite game. The running time $T(n)$ of an algorithm
takes time polynomial in $n$, if there exists an integer constant $d>0$ such
that $T(n)=\mathcal{O}(n^{d})$. For more details see Chapter 3 in the above
book.
\end{remark}

\end{document}




\begin{exercise}
  \label{ex:frequency}
  Formally prove that \textbf{FREQUENCY ASSIGNMENT} is $\NP$-complete. For this you may use
  the fact that a similar problem used in lectures is $\NP$-complete.
      
    \smallskip
        
  \noindent In the \textbf{FREQUENCY ASSIGNMENT} problem we are given a set of transmitters $T=\{t_1, t_2, \ldots, t_n\}$,  $k$ frequencies, and the list of pairs of transmitters that interfer and therefore cannot use the same frequency. We ask whether there is an assignment of each transmitter to one of $k$ frequencies such that there is no interference between the transmitters. 
            
\end{exercise}



\begin{solution}

test

\end{solution}


\begin{exercise}
  \label{ex:CO-NP}
  Fomally prove that logical entailment is $co-\NP$-complete. The formal definition of entailment ( $\models$) is this: $\alpha \models \beta$ if and only if, in every truth assignment in which $\alpha$ is true, $\beta$ is also true.  
\end{exercise}


%\solution{
\textbf{Solution:}\newline
\\
\noindent We have choosen the following\\
Precondition$:= x=x_0 \land x \ge 0$ and\\
Postcondition$:= x=2x_0$ and\\
extended the Invariant$:= 2x_0+x=y \land x \ge 0$\\

\bigskip
\noindent Now we want to show the assertion is totally correct:\\
\\
\begin{ALG}
\ASSERTN1{\text{Precondtion: } x=x_0 \land x \ge 0}\\
\ASSERTN4{\INV \sub{y<-3x}}	\quad(as)$\ua$\\
$y \leftarrow 3x;$\\
\ASSERTN3{\INV := 2x_0+x=y \land x \ge 0}	\quad(wht'')\\
\WHILE\ $2x \neq y$ \DO\\
\ASSERTN9{\INV \land 2x \neq y \land t=t_0}	\quad(wht'')\\
\ASSERTN8{\INV \land 0 \le t < t_0 \sub{y<-y+1} \sub{x<-x+1}} 
\quad(as)$\ua$\\
\>\ $x \leftarrow x+1;$\\
\ASSERTN7{\INV \land 0 \le t < t_0 \sub{y<-y+1}}	
\quad(as)$\ua$\\
\>\ $y \leftarrow y+1;$\\
\ASSERTN6{\INV \land 0 \le t < t_0}	\quad(wht'')\\
\OD\\
\ASSERTN5{\INV \land 2x = y}	\quad(wht'')\\
\ASSERTN2{\text{Postcondition: } x=2x_0}
\end{ALG}

\bigskip
\noindent We define the bound function $t$ as:\\
$t:=y-2x$,\\
which fulfils the criteria for bound functions (listed in exercise 7).\\

\bigskip
\noindent Now the implications:\\�
\\
\textbf{$1 \rightarrow 4:$}\\
\indent $x=x_0 \land x \ge 0 \rightarrow Inv [y|3x]$\\
$= x=x_0 \land x \ge 0 \rightarrow 2x_0+x = y \land x \ge 0 [y|3x]$\\
$= \textcolor{green}{x=x_0} \land x \ge 0 \rightarrow \textcolor{green}{\underbrace{2x_0}_{=2x}}+x = 3x \land x \ge 0$\\
$= x=x_0 \land x \ge 0 \rightarrow \underbrace{3x = 3x}_{true} \checkmark\\
\indent \land x \ge 0$\\
$= x=x_0 \land \textcolor{magenta}{x \ge 0} \rightarrow \underbrace{3x = 3x}_{true} \checkmark\\
\indent \land \textcolor{magenta}{x \ge 0} \checkmark$\\
\\
\textbf{$5 \rightarrow 6:$}\\
\indent $Inv \land 2x \neq y \land t = t_0 \rightarrow Inv \land 0 \le t < t_0 [y|y+1] [x|x+1]$\\
$= 2x_0+x = y \land x \ge 0 \land 2x \neq y \land t = t_0\\ 
\indent \rightarrow 2x_0+x = y \land x \ge 0 \land 0 \le t < t_0 [y|y+1] [x|x+1]$\\
$= 2x_0+x = y \land x \ge 0 \land 2x \neq y \land t = t_0\\ 
\indent \rightarrow 2x_0+x+1 = y+1 \land x+1 \ge 0 \land 0 \le t < t_0$\\
$= 2x_0+x = y \land x \ge 0 \land 2x \neq y \land t = t_0\\ 
\indent \rightarrow 2x_0+x+1 = y+1 \land x+1 \ge 0 \land 0 \le \underbrace{\underbrace{t}_{=y+1-2(x+1)}}_{=y-2x-1} < \underbrace{t_0}_{=y-2x}$\\
$= 2x_0+x = y \land x \ge 0 \land 2x \neq y \land t = t_0\\ 
\indent \rightarrow 2x_0+x+1 = y+1 \land x+1 \ge 0 \land \underbrace{0 \le y-2x-1 < y-2x}_{true}$\\
$= 2x_0+x = y \land x \ge 0 \land 2x \neq y \land t = t_0\\ 
\indent \rightarrow 2x_0+x$ \sout{$+1$} $= y$ \sout{$+1$} $\land x+1 \ge 0 \land \underbrace{0 \le y-2x-1 < y-2x}_{true}$\\
$= \textcolor{green}{2x_0+x = y} \land x \ge 0 \land 2x \neq y \land t = t_0\\ 
\indent \rightarrow \textcolor{green}{2x_0+x = y} \checkmark \\
\indent \land x+1 \ge 0\\ 
\indent \land 0 \le y-2x-1 < y-2x \checkmark$\\
$= 2x_0+x = y \land \textcolor{magenta}{x \ge 0} \land 2x \neq y \land t = t_0\\ 
\indent \rightarrow 2x_0+x = y \checkmark \\
\indent \land \textcolor{magenta}{\underbrace{x+1 \ge 0}_{=x \ge -1}} \checkmark \\ 
\indent \land 0 \le y-2x-1 < y-2x \checkmark$\\
\\
\textbf{$9 \rightarrow 2:$}\\
\indent $Inv \land 2x = y \rightarrow x = 2x_0$\\
$= \textcolor{green}{2x_0+x = \underbrace{y}_{=2x}} \land x \ge 0 \land \textcolor{green}{2x = y} \rightarrow x = 2x_0$\\
$= 2x_0$ \sout{$+x$} $=$ \sout{$2$} $x \land x \ge 0 \land 2x = y \rightarrow x = 2x_0$\\
$= \textcolor{magenta}{2x_0 = x} \land x \ge 0 \land 2x = y \rightarrow \textcolor{magenta}{x = 2x_0} \checkmark$\\
\\
So the correctness asertion above is totally correct.\\
It terminates for all integers $(x_0)$ which are greater or equal 0.
%}



\begin{exercise}
  \label{ex:Colors}
  It is well known that the \textbf{k-COLORABILITY} problem is $\NP$-complete for every $k \geq 3$. Recall that the instance of \textbf{k-COLORABILITY} is an undirected graph $G = (V, E)$. Suppose that we restrict this instance of \textbf{k-COLORABILITY} to trees. Can the restricted problem be solved with an algorithm that runs in polynomial time? If yes, provide such an algorithm. 
\end{exercise}


\subsection{solution}

\subsubsection{Algorithm}
\begin{algorithmic}
\Function{DFS}{$a$}
    \For {each vertex $u \in V[G]$}
        \State color[u] $\gets$ k
        \State $\pi[u] \gets NULL$
    \EndFor \\
    \State $time \gets 0$
\EndFunction
\end{algorithmic}

\smallskip

\begin{algorithmic}
\Function{DFS-Visit}{$u$}
    \State color[u] $\gets$ k
    \State d[u] $\gets$ time $\gets$ time + 1
    \For {each $v \in Adj[u]$}
      \State $\triangleright$ explore edge $(u,v)$
      \If {color[v] = k}
          \State $\pi[v] \gets u$
          \State DFS-Visit($v$)
      \EndIf
    \EndFor
    \State color[v] $\gets$  k
    \State f[u] $\gets$ time $\gets$ time + 1
\EndFunction
\end{algorithmic}

\subsubsection{Description}
The definition of the tree saies that:
\begin{itemize}
 \item there are no cycles allowed
 \item paths are connected
\end{itemize}



\begin{exercise}
  \label{ex:Nqueens}
  Provide a reduction of \textbf{N-Queens} problem to \textbf{SAT}. Give a proof sketch of the correctness of your reduction. Does this implies that the \textbf{N-Queens} is an $\NP$-complete problem? Argue your answer.  
  
  \smallskip 
  
\noindent In the \textbf{N-Queens} problem we are given $n$ queens and an $n \times n$ chessboard. We ask whether we can place these $n$ queens on the  chessboard such that no two queens attack each other. Two queens attack each other if they are placed in the same row, or in the same column, or in the same diagonal.    
       
   
\end{exercise}


\textbf{Solution:}\newline

To prove the $\kw{while}$-rule for partial correctnes we have to consider
following definition: 
\begin{definition}
	Let $\{F\}\, p\, \{G\}$ be an arbitrary partial correctness assertion. Then $\{F\}\, p\, \{G\}$ is
	sound if
	\begin{equation*}
		\vdash_{par} \{F\}\, p\, \{G\}\quad \text{ then } \vDash_{par} \{F\}\, p\, \{G\}.
	\end{equation*}
\end{definition}

Then let $I$ be an arbitrary interpretation and assume that $\vDash _{%
\mathrm{par}}^{I}\{\mathrm{Inv}\,\wedge \,e\}\,p\,\{\mathrm{Inv}\}$, i.e. is
partially correct. Then for some fixed state $\tau $ which satisfies the
invariant $\mathrm{Inv}$ and $e$ in the precondition, there $\exists \tau
^{\prime }\in \mathcal{S}$ such that $[p]\tau =\tau ^{\prime }$ is defined
and $\tau ^{\prime }$ satisfies $\mathrm{Inv}$ in the postcondition. To
prove that $\vDash _{\mathrm{par}}^{I}\{\mathrm{Inv}\}\,\kw{while}\ e\ %
\kw{do}\ p\ \kw{od}\,\{\mathrm{Inv}\,\wedge \,\lnot e\}$ is valid, we have
to show for all states $\sigma \in \mathcal{S}$, such that%
\begin{equation*}
\mathrm{Inv}\,\Rightarrow \,F\quad \lbrack F]\sigma \text{ is true, then }[%
\kw{while}\ e\ \kw{do}\ p\ \kw{od}]\sigma \text{ is true with }%
G\,\Rightarrow \,(\mathrm{Inv}\,\wedge \,\lnot e).
\end{equation*}

The proof can be done by using structural induction on the given axiom rule.

We shall show by induction that $P(n),n\in \mathbb{N}$ holds, where%
\begin{eqnarray*}
P(n)\quad \overset{defined}{\Longleftrightarrow }\quad \forall \sigma
,\sigma ^{\prime \prime } &\in &\mathcal{S}\text{ such that }(\sigma ,\sigma
^{\prime \prime })\in \Theta _{n}\text{ and} \\
\sigma &\vDash &_{\mathrm{par}}^{I}\mathrm{Inv}\;\Rightarrow \;\sigma
^{\prime \prime }\vDash _{\mathrm{par}}^{I}(\in \mathcal{S}),
\end{eqnarray*}

for all $n$ iteration-steps of $\kw{while}\ e\ \kw{do}\ p\ \kw{od}$. ( $%
(\sigma ,\sigma ^{\prime \prime })\in \Theta _{n}$ denotes a partial
function.)

\bigskip

\textit{Base case:}

If $n=0$, then the induction hypothesis $P(0)$ is vacuosly true.

\bigskip

\textit{Induction step:}

We assume that the induction hypothesis $P(n)$ for $n\geq 0$ holds and try
to prove $P(n+1)$.

Let $w\equiv \kw{while}\ e\ \kw{do}\ p\ \kw{od}$, and consider some
arbitrary states $\sigma ,\sigma ^{\prime },\sigma ^{\prime \prime }\in 
\mathcal{S}$ with $[\mathrm{Inv}]\sigma =true$ (in the precondition).

We have to show that $\sigma ^{\prime \prime }\vDash _{\mathrm{par}}^{I}(%
\mathrm{Inv}\,\wedge \,\lnot e)$, i.e. $[\mathrm{Inv}\,\wedge \,\lnot
e]\sigma ^{\prime \prime }=true$. Then there are existing two cases for any $%
\tau \in \mathcal{S}$:

\bigskip

\begin{description}
\item[\textbf{case} \textit{i)}] $[e]\tau =true$. Since by assumption $%
\vDash _{\mathrm{par}}^{I}\{\mathrm{Inv}\,\wedge \,e\}\,p\,\{\mathrm{Inv}\}$
is partially correct, i.e.%
\begin{eqnarray*}
\forall \sigma  &\in &\mathcal{S}, \\
&&\text{if }[\mathrm{Inv}\,\wedge \,e]\sigma =true\;\Rightarrow \;[p]\sigma 
\text{ is defined and }[\mathrm{Inv}][p]\sigma =true.
\end{eqnarray*}

\item Then we have $\sigma \vDash _{\mathrm{par}}^{I}e$ and hence $\sigma
\vDash _{\mathrm{par}}^{I}(\mathrm{Inv}\,\wedge \,e)$.

\item Thus,%
\begin{equation*}
(w,\sigma )\,\Rightarrow \,(p;w,\sigma )\,\Rightarrow ^{\ast }\,\sigma
^{\prime \prime }.
\end{equation*}

\item Hence, $[w]\sigma =\sigma ^{\prime \prime }$ is defined and satisfies $%
\mathrm{Inv}$ in the postcondition.

\item From the induction hypothesis $P(n)$ we obtain $\sigma ^{\prime \prime
}\vDash _{\mathrm{par}}^{I}(\mathrm{Inv}\,\wedge \,\lnot e)$, since for $%
[e]\sigma \neq 0$,%
\begin{equation*}
(p,\sigma )\,\Rightarrow \,\sigma ^{\prime }\quad \text{and\quad }(w,\sigma
^{\prime })\,\Rightarrow ^{\ast }\,\sigma ^{\prime \prime }.\,
\end{equation*}

\item Since $[\mathrm{Inv}\,\wedge \,e]\sigma =true$, $[p]\sigma $ is
defined and we obtain $\sigma ^{\prime }\vDash _{\mathrm{par}}^{I}\mathrm{Inv%
}$, as $\vDash \{\mathrm{Inv}\,\wedge \,\lnot e\}\,p\,\{\mathrm{Inv}\}$.

\item Then the induction hypothesis can be applied to $[w]\sigma ^{\prime
}=\sigma ^{\prime \prime }$ which gives us $[\mathrm{Inv}\,\wedge \,\lnot
e]\sigma ^{\prime \prime }=true$.

\item[\textbf{case} \textit{ii)}] $[e]\tau =false$. Then we have $\sigma
\vDash _{\mathrm{par}}^{I}\lnot e$ and \ hence $\sigma \vDash _{\mathrm{par}%
}^{I}(\mathrm{Inv}\,\wedge \,\lnot e)$, i.e. $[\mathrm{Inv}\,\wedge \,\lnot
e]\sigma =true$. Then according to%
\begin{equation*}
(w,\sigma ^{\prime \prime })\,\Rightarrow \,\sigma \quad \text{if }[e]\sigma
^{\prime \prime }=0,
\end{equation*}

\item i.e. $[w]\sigma ^{\prime \prime }=\sigma $. This establishes the
induction hypothesis $P(n+1)$.$\quad \Rightarrow \,P(n)$ holds for all $n$
and the rule for $\kw{while}$ is \textit{sound}.

\item 
\end{description}



\begin{exercise}
  Consider the following problem:
  \begin{center}
    \fbox{
      \begin{minipage}[c]{.95\linewidth}
        \textbf{N-SORTED-ELEMENTS}

        \medskip

        INSTANCE: A non-empty list $L=(e_1,\ldots,e_n)$ of non-negative integers. \\
        QUESTION: Does the list $L$ contain a sub-list of $k$ consecutive sorted numbers in ascending order (from left to right)?
      \end{minipage}
    }
  \end{center}

  \medskip Argue that \textbf{N-SORTED-ELEMENTS} can be solved using only logarithmic
  space.
\end{exercise}


\subsection{solution}
This can be explained by using a small procedure (see Procedure \ref%
{Alg-N-Sorted-El}) for this problem.

%TCIMACRO{%
%\TeXButton{Algorithm: N-SORTED-ELEMENTS}{\floatname{algorithm}{Procedure}
%\renewcommand{\algorithmicrequire}{\textbf{Input:}}
%\renewcommand{\algorithmicensure}{\textbf{Output:}}
%\renewcommand{\algorithmicforall}{\textbf{for each}}
%
%\begin{algorithm}[ht]
%\small
%\begin{algorithmic}
%	\Require $L=(e_{1},\ldots , e_{n}), k\in \mathbb{N}^{+}$
%	\Ensure $\mathbf{true}$ if $n$-sorted elements of size $k$ are found, $\mathbf{false}$ otherwise.
%	\newline
%	\State $i \leftarrow 1$
%	\State $count \leftarrow 1$ \Comment{count variable for finding $n$-sorted elements of size $k$.} 
%	\ForAll{$i \;$ s.t. $\; 1 \leq i \leq |L| - 1$}
%		\If{$\left(e(i) + 1\right) = e(i+1)$}
%			\State $count \leftarrow count + 1$
%		\Else
%			\If{$count = k$}
%				\Return \textbf{true}
%			\Else
%				\State $count \leftarrow 1$ \Comment{the number of sorted elements was $< k$; $\Rightarrow$ reset the count value.}
%			\EndIf
%		\EndIf
%	\EndFor\\
%	\Return \textbf{false}	
%\end{algorithmic}
%\caption{\small \textsc{N-Sorted-Elements} procedure.}
%\label{Alg-N-Sorted-El}
%\end{algorithm}}}%
%BeginExpansion
\floatname{algorithm}{Procedure}
\renewcommand{\algorithmicrequire}{\textbf{Input:}}
\renewcommand{\algorithmicensure}{\textbf{Output:}}
\renewcommand{\algorithmicforall}{\textbf{for each}}

\begin{algorithm}[ht]
\small
\begin{algorithmic}
	\Require $L=(e_{1},\ldots , e_{n}), k\in \mathbb{N}^{+}$
	\Ensure $\mathbf{true}$ if $n$-sorted elements of size $k$ are found, $\mathbf{false}$ otherwise.
	\newline
	\State $i \leftarrow 1$
	\State $count \leftarrow 1$ \Comment{count variable for finding $n$-sorted elements of size $k$.} 
	\ForAll{$i \;$ s.t. $\; 1 \leq i \leq |L| - 1$}
		\If{$\left(e(i) + 1\right) = e(i+1)$}
			\State $count \leftarrow count + 1$
		\Else
			\If{$count = k$}
				\Return \textbf{true}
			\Else
				\State $count \leftarrow 1$ \Comment{the number of sorted elements was $< k$; $\Rightarrow$ reset the count value.}
			\EndIf
		\EndIf
	\EndFor\\
	\Return \textbf{false}	
\end{algorithmic}
\caption{\small \textsc{N-Sorted-Elements} procedure.}
\label{Alg-N-Sorted-El}
\end{algorithm}%
%EndExpansion

The above procedure can be solved in logarithmic space in the size of the
input $I$ with $|I|=|L|$.\newline
\noindent The procedure needs only one pointer to an element in the list $L$
and one count variable of constant size. I.e. the pointer variable $i$ need $%
\log (n) $ bits for its representation. Observe a very large list $L$ of
non-negative intergers. Since the memory of a program is limited to constant
many pointers, the pointer variable $i$ uses at most $O(\log _{2}|L|)$ bits
of memory.

\bigskip


%%TCIMACRO{%
%%\TeXButton{algorithm: N-SORTED-ELEMENTS}{\begin{algorithm}
%%	\KwIn{non-empty list L, integer k $\geq$ 1}
%%	\KwOut{bool}
%%		int count $\leftarrow 1;$\\
%%		\For{i in 1 \KwTo n}{
%%			\If{$L[i]==L[i+1]$}{
%%				$count = count+1;$\\ }
%%			\Else {
%%				\If{$count==k$}
%%					return ("yes");\\ 
%%				\Else 
%%					\tcc{number of sorted elements was too small}
%%					$count = 1$;\\
%%				\ Fi
%%			}
%%		}
%%		return ("no");
%%	\caption{\textfb{N-SORTED-ELEMENTS}}
%%	\label{alg:n-sorted}
%%\end{algorithm}}}%
%%BeginExpansion
%\begin{algorithm}[H]
%	\KwIn{non-empty list L, integer k $\geq$ 1}
%	\KwOut{bool}
%		int count $\leftarrow 1;$\\
%		\For{i in 1 \KwTo n} {
% 			\If{$L[i]==L[i+1]$}{
%				$count = count+1;$\\ }
%			\Else {
%				\If{$count==k$}
%					{return ("yes");\\ }
%				\Else 
%					{\tcc{number of sorted elements was too small}
%					$count = 1$;\\}
%			}
%		}
%		return ("no")
%% 	\caption{\textfb{N-SORTED-ELEMENTS}}
%	\label{alg:n-sorted}
%\end{algorithm}%
%%EndExpansion
%
%The algorithm above requires logarithmic space in the sice of the length of the input list:\\
%\newline
%\begin{itemize}
% \item $ count needs log_2|n| space.$
%\end{itemize}



\begin{exercise}
  \label{ex:turing}

  Design a Turing machine that increments by one a value represented by a string of 0s and 1s.

\end{exercise}


\subsection{solution 10}

$M=(K,\Sigma, \delta, start)$\\
$K=\{s, l, r', r, e\}$\\
$\Sigma=(0,1,\triangleright, \sqcup)$

\begin{center}
\begin{tabular}[t]{| l | l | l |}
\hline
$p \in K$ & $\sigma \in \Sigma $ & $ \delta ( p, \sigma ) $ \\ \hline
s & $\triangleright$ & (s, $\triangleright$, $\rightarrow$ )\\ 
s & 0 & (s, 0, $\rightarrow$ )\\ 
s & 1 & (s, 1, $\rightarrow$ )\\ 
s & $\sqcup$ & (ok, 1, --)\\ \hline
l & 0 & (ok, 1, -- )\\ 
l & 1 & (l, 0, $\leftarrow$ )\\ 
l & $\triangleright$ & (r, $\triangleright$, $\rightarrow$ )\\ \hline
r' & 0 & (r, 1, $\rightarrow$ )\\ \hline
r & 0 & (r, 0, $\rightarrow$ )\\ 
r & $\sqcup$ & (e, 0, $\rightarrow$ )\\ \hline
e & $\sqcup$ & (ok, $\sqcup$, -- )\\ \hline
\end{tabular} 
\end{center}

\subsubsection{High lvl definition}
\begin{itemize}
\item state s:
The tourig-machine M starts at state \textbf{s} and then checks if it got an end sign.
If this is the case it writes \textbf{1} end stops with positive validity 
furthermore scans M the whole input from left to right without changes and jumps to state
l if it got the $\sqcup$ sign
\item state l:
Directed to the left it tries to fill in a \textbf{1} on the first position with a \textbf{0}
and ends with positive validity afterwards.
If this isn't effective it flips every \textbf{1} to 0 
gets back to the $\triangleright$ and jumps in state r'
\item state r':
It fills position with a \textbf{1} and changes to state \textbf{r}
\item state r: 
It goes right without changes till the $\sqcup$ sign, writes a \textbf{0} 
instead of $\sqcup$ and jumps to state \textbf{e}
\item state e:
Goes right, writes an $\sqcup$ and stops.

\end{itemize}
%%%%%%%%%%%%%%%% END EXERCISES




\end{document}


