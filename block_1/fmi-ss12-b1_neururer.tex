\documentclass [11pt]{article}
\usepackage{latexsym}
\usepackage{amssymb}
\usepackage{epsfig} 
\usepackage{enumerate}
\usepackage{xspace}

%\usepackage{graphicx}
%\usepackage{typearea}
%\usepackage{multicol}
%\usepackage{amsfonts}
%\usepackage[nounderscore]{syntax}
%\usepackage{paralist}
%\usepackage{tikz}
%\usetikzlibrary{positioning}
%\usepackage{url}
%\usepackage{xspace}
%\usepackage{bbm}
%\usepackage{listings}
%%\usepackage{MnSymbol}
%%\usepackage[ruled]{algorithm2e}
%\def\NN{{\ensuremath{\mathbbm{N}_0\xspace}}}
%\usepackage{wrapfig}
%\usepackage{graphicx}
%\usetikzlibrary{arrows,automata}
%\setlength{\textheight}{25cm}


   \textwidth      15cm
   \textheight     23cm
   \oddsidemargin 0.5cm
   \topmargin    -0.5cm
   \evensidemargin\oddsidemargin

 \newcommand{\nop}[1]{}


   \pagestyle{plain}
   \bibliographystyle{plain}


\title{Formale Methoden der Informatik \\
Block 1: Computability and Complexity }
\author{Exercises 1-10}
\date{SS 2012}


  \newtheorem{theorem}{Theorem}
  \newtheorem{lemma}[theorem]{Lemma}
  \newtheorem{corollary}[theorem]{Corollary}
  \newtheorem{proposition}[theorem]{Proposition}
  \newtheorem{conjecture}[theorem]{Conjecture}
  \newtheorem{definition}[theorem]{Definition}
  \newtheorem{example}[theorem]{Example}
  \newtheorem{remark}[theorem]{Remark}
  \newtheorem{exercise}[theorem]{Exercise}

  \newcommand{\ra}{\rightarrow}
  \newcommand{\Ra}{\Rightarrow}
  \newcommand{\La}{\Leftarrow}
  \newcommand{\la}{\leftarrow}
  \newcommand{\LR}{\Leftrightarrow}

  \renewcommand{\phi}{\varphi}
  \renewcommand{\theta}{\vartheta}


\newcommand{\ccfont}[1]{\protect\mathsf{#1}}
\newcommand{\NP}{\ccfont{NP}}

\newcommand{\NN}{\textbf{N}}
\newcommand{\IN}{\textbf{Z}}
\newcommand{\bigO}{\mathrm{O}}
\newcommand{\bigOmega}{\Omega}
\newcommand{\bigTheta}{\Theta}
\newcommand{\REACHABILITY}{\mbox{\bf REACHABILITY}}
\newcommand{\MAXFLOW}{\mbox{\bf MAX FLOW}}
\newcommand{\MAXFLOWD}{\mbox{\bf MAX FLOW(D)}}
\newcommand{\MAXFLOWSUB}{\mbox{\bf MAX FLOW SUBOPTIMAL}}
\newcommand{\MATCHING}{\mbox{\bf BIPARTITE MATCHING}}
\newcommand{\TSP}{\mbox{\bf TSP}}
\newcommand{\TSPD}{\mbox{\bf TSP(D)}}

\newcommand{\ThreeCol}{\mbox{\bf 3-COLORABILITY}}
\newcommand{\TwoCol}{\mbox{\bf 2-COLORABILITY}}
\newcommand{\kCol}{\mbox{\bf k-COLORABILITY}}
\newcommand{\HamPath}{\mbox{\bf HAMILTON-PATH}}
\newcommand{\HamCycle}{\mbox{\bf HAMILTON-CYCLE}}

\newcommand{\ONESAT}{\mbox{\bf 1-IN-3-SAT}}
\newcommand{\MONONESAT}{\mbox{\bf MONOTONE 1-IN-3-SAT}}
\newcommand{\kSAT}{\mbox{\bf k-SAT}}
\newcommand{\NAESAT}{\mbox{\bf NAESAT}}
\newcommand{\CLIQUE}{\textbf{CLIQUE}\xspace} 
\newcommand{\VC}{\textbf{VERTEX COVER}\xspace}



\renewcommand{\labelenumi}{(\alph{enumi})}

%%% useful macros for Turing machines:
\newcommand{\blank}{\sqcup}
\newcommand{\ssym}{\triangleright}
\newcommand{\esym}{\triangleleft}
\newcommand{\halt}{\mbox{h}}
\newcommand{\yess}{\mbox{``yes''}}
\newcommand{\nos}{\mbox{``no''}}
\newcommand{\lmove}{\leftarrow}
\newcommand{\rmove}{\rightarrow}
\newcommand{\stay}{-}
\newcommand{\diverge}{\nearrow}
\newcommand{\yields}[1]{\stackrel{#1}{\rightarrow}}

\newcommand{\HALTING}{\mbox{\bf HALTING}}

\newcommand{\true}{{\it true}}
\newcommand{\false}{{\it false}}


\newcommand{\samplesolution}[1]{\noindent {\bf Sample solution.}  #1}




%%%%%%%%%%%%%%%%%%%%%%%%%%%%%%%%%%%%%%%%%%%%%%%%%%%%%%%%%%%%%%%%%%%

\begin{document}


\maketitle


\begin{exercise}
  Consider the problem \textbf{PROCEDURE NEG-ASSIGNMENT}, which is defined
  as follows:

  \begin{center}
    \fbox{
      \begin{minipage}[c]{.9\linewidth}
        \textbf{PROCEDURE NEG-ASSIGNMENT}

        \medskip INSTANCE: A triple $(\Pi,I, k)$, where (i) $\Pi$ is a
        program that takes one string as input and outputs true or false, (ii) $I$ is a
        string, and (iii) $k$ is an integer variable used in program $\Pi$.     \\
        QUESTION: Does variable $k$ ever get assigned a negative value when the program $\Pi$ is executed with input $I$?

        
      \end{minipage}
    }
  \end{center}
  Prove that \textbf{NEG-ASSIGNMENT} is undecidable. Prove the undecidability
  by providing a reduction from the \textbf{HALTING} problem to
  \textbf{NEG-ASSIGNMENT}, and arguing that your reduction is correct.
\end{exercise}


\subsection{solution}
\textbf{Proof}: Reduction from \textsc{Halting} problem to \textsc{%
Neg-Assignment}.

\noindent Let $(\Pi ,I)$ be an arbitrary instance of \textsc{Halting}, i.e. $\Pi $ \
is a program that takes an input I. Based on this, we have to construct an
instance $(\Pi ^{\prime },I^{\prime },k)$ for \textsc{Neg-Assignment} by
setting $I^{\prime }=I$ and setting initially $k\geq 0$. Then $\Pi ^{\prime
} $ is defined as follows:

%TCIMACRO{%
%\TeXButton{Procedure}{\begin{center}
%\fbox{
%\begin{minipage}[c]{.9\linewidth}
%\small
%boolean $\Pi'$(String $I$) $\{$\\
%   int $k := 0$;\\
%   call $\Pi(I)$;\\
%   $k := -1$;\\
%   return true;\\
%$\}$\\
%
%\end{minipage}
%}
%\end{center}}}%
%BeginExpansion
\begin{center}
\fbox{
\begin{minipage}[c]{.9\linewidth}
\small
boolean $\Pi'$(String $I$) $\{$\\
   int $k := 0$;\\
   call $\Pi(I)$;\\
   $k := -1$;\\
   return true;\\
$\}$\\

\end{minipage}
}
\end{center}%
%EndExpansion

\noindent Let $x=(\Pi ,I)$ be an instance of \textsc{Halting} and $R(x)=(\Pi ^{\prime
},I^{\prime },k)$ the resulting instance from the reduction. Then, we have
to show that $x$ is reducible to $R(x)$ by following equivalence relation:%
\begin{eqnarray*}
&&(\Pi ,I)\text{ is a positive instance of \textsc{Halting}}%
\quad\Leftrightarrow\quad (\Pi ^{\prime },I^{\prime },k)\text{ is apositive
instance} \\
&&\text{of the procedure \textsc{Neg-Assignment}.}
\end{eqnarray*}

(i.e. $\Pi $ halts on $I$ and assigns to $k$ a negative integer value$\quad
\Leftrightarrow\quad \Pi^{\prime }$ returns \textit{true}.)

$\Rightarrow :$ Suppose that $\left( \Pi ,I\right) $ is a positive instance
of \textsc{Halting}, i.e. $\Pi $ terminates on $I$.

\noindent Then by construction of $(\Pi ^{\prime },I^{\prime },k)$, the call of $\Pi
(I^{\prime })$ in program $\Pi ^{\prime }$ terminates since $I^{\prime }=I$.
Hence, $\Pi ^{\prime }$ halts on input $I^{\prime }$, assignes $k:=-1$ and
returns \textit{true} respectively.

\noindent Thus, $(\Pi ^{\prime },I^{\prime },k)$ is a positive instance of \textsc{%
Neg-Assignment}.

\bigskip

$\Leftarrow :$ Suppose that $\left( \Pi ^{\prime },I^{\prime },k\right) $ is
a positive instance of \textsc{Neg-Assignment}, i.e. $\Pi ^{\prime }$
returns \textit{true} after assigning a negative value to $k$. Since $\Pi
^{\prime }$ involves the call of $\Pi (I^{\prime })$ with $I^{\prime }=I$, $%
\Pi $ terminates on $I$ and $k$ will be set to $-1$. Hence, $(\Pi ,I)$ is a
positive instance of \textsc{Halting}.

\bigskip



\begin{exercise}
  Prove that the problem \textbf{NEG-ASSIGNMENT} from Exercise 1 is semi-decidable. 
To this end, provide a semi-decision procedure and justify your solution. Additionally,  show that the co-problem of \textbf{NEG-ASSIGNMENT} is not semi-decidable.
\end{exercise}


\subsection{solution}

test



\begin{exercise}
  Give a formal proof that \textbf{SUBSET SUM} is in $\NP$, i.e.\, define a
  certificate relation and discuss that it is polynomially balanced and
  polynomial-time decidable.
  
   \smallskip
    
  \noindent In the \textbf{SUBSET SUM} problem we are given a finite set of integer numbers $S=\{a_1, a_2, \ldots, a_n\}$ and an integer number $t$. We ask whether there is a subset $S'\subseteq S$ whose elements sum is equal to $t$?
  
\end{exercise}


\subsection{solution}

test


\begin{exercise}
  \label{ex:partition}
  Formally prove that \textbf{PARTITION} is $\NP$-complete. For this you may use
  the fact that \textbf{SUBSET SUM} is $\NP$-complete.   
  
  \smallskip
  \noindent In the \textbf{PARTITION} problem we are given a finite set of integers $S=\{a_1, a_2, \ldots, a_n\}$. We ask whether the set $S$ can be partitioned into two sets $S_1, S_2$ such that the sum of the numbers in $S1$ equals the sum of the numbers in $S_2$? 
  

\end{exercise}


\subsection{solution}

test



\begin{exercise}
  \label{ex:frequency}
  Formally prove that \textbf{FREQUENCY ASSIGNMENT} is $\NP$-complete. For this you may use
  the fact that a similar problem used in lectures is $\NP$-complete.
      
    \smallskip
        
  \noindent In the \textbf{FREQUENCY ASSIGNMENT} problem we are given a set of transmitters $T=\{t_1, t_2, \ldots, t_n\}$,  $k$ frequencies, and the list of pairs of transmitters that interfer and therefore cannot use the same frequency. We ask whether there is an assignment of each transmitter to one of $k$ frequencies such that there is no interference between the transmitters. 
            
\end{exercise}



\subsection{solution}
The NP-hardness of the \textsc{Frequency-Assignment} problem can be shown by
a reduction to $k$\textsc{-Colorability}, since $k$\textsc{-Colorability} is
in NP.

\noindent At first we have to check \ if \textsc{Frequency-Assignment} id a membership
of NP.

\noindent This can be done by a guess \& check procedure:

\subsubsection{definition}
Let\ $G_{F}=(T,E)$ be an arbitrary undirected graph for \textsc{%
Frequency-Assignment} with $T=\{t_{1},t_{2},\ldots ,t_{n}\}$ and $E^{\prime
}=\{(t_{i},t_{j})\mid t_{i},t_{j}\in T(G_{F})$ with $i\neq j$ s.t. $t_{i}$
and $t_{j}$ are interfering each other$\}\subseteq E=$.$\{(t_{i},t_{j})\mid
t_{i},t_{j}\in T(G_{F})$ with $i\neq j\}$. Let $F=\{f_{1},\ldots ,f_{k}\}$
be an arbitrary set of frequencies and $fr:T(G_{F})\rightarrow F$ a function
for assigning each transmitter-vertex $t\in T(G_{F})$ to a frequency $f\in F$%
.

Guess an arbitrary set of frequencies $F=\{f_{1},\ldots ,f_{k}\}$ of size $k$%
. Then for each transmitter-vertex $t\in T(G_{F})$, verify that $t$ has no
adjacent verteices $u\in T(G_{F})$ with the same frequency $f\in F$, i.e. $%
\forall t_{i},t_{j}\in T(G_{F})$ with $i\neq j$, s.t. $(t_{i},t_{j})\in E$
and $fr(t_{i})\neq fr(t_{j})$. The verification of the edges in $G_{F}$
takes polynomial time in the size of $k=|F|$ and the number of edges $n=|E|$
in $G_{F}$.

It remains to show that \textsc{Frequency-Assignment} is NP-hard, by
reducing \textsc{Frequency-Assignment} to $k$\textsc{-Colorability}.

Let\ $G_{F}=(T,E)$ be an arbitrary undirected graph for \textsc{%
Frequency-Assignment} (as defined above). To prove the correctness of the
reduction, we have to show that following equation holds:%
\begin{equation*}
(G_{F},k)\text{ is a positive instance of \textsc{Frequency-Assignment}}%
\quad \Leftrightarrow \quad (G_{C},k^{\prime })\text{ is a positive instance
of }k\text{\textsc{-Colorability}.}
\end{equation*}

\textbf{Proof:}

\noindent Given an arbitrary instance $(G_{F},k)$ of \textsc{Frequency-Assignment}.
Then an instance of $(G_{C},k^{\prime })$ of $k$\textsc{-Colorability} can
be produced iff, $k=k^{\prime }$ and there exists a function $%
c:V(G_{C})\rightarrow C$, where $C=\{1,\ldots ,k^{\prime }\}$ is a set of
colors of size $k^{\prime }$.

$\Rightarrow :$ Suppose $(G_{F},k)$ is a positive instance of \textsc{%
Frequency-Assignment}. Then $\forall t_{i},t_{j}\in T(G_{F})$ with $i\neq j$%
, every edge $(t_{i},t_{j})\in E(G_{F})$, has pairwise different frequencies
such that $fr(t_{i})\neq fr(t_{j})$. We have to define an assignment
function $\mu $ such that $\mu :fr(t)\rightarrow c(v)$, for all $t\in
T(G_{F})$ and for\ all $v\in V(G_{C})$. Since $(G_{F},k)$ is a positive
instance of \textsc{Frequency-Assignment}, then there exists a corresponding
mapping $\mu $ from each node $t\in T(G_{F})$ to each node $v\in V(G_{C})$,
such that $\forall (t_{i},t_{j})\in E(G_{F}):fr(t_{i})\neq fr(t_{j})$ and 

$\forall (v_{i},v_{j})\in E(G_{C}):c(v_{i})\neq c(v_{j})$ and $i\neq j$ and $%
k=k^{\prime }$. This means that both functions are behaving identical on
each graph, such that $fr=c$, i.e. there is a \textit{bijection} between the
vertex sets of $G_{F}$ and $G_{C}$. Thus, both undirected graphs $G_{F}$ and 
$G_{C}$ are \textit{isomorph}. Hence, $(G_{C},k^{\prime })$ is a positive
instance of $k$\textsc{-Colorability}$.$

$\Leftarrow :$ Suppose that $(G_{C},k^{\prime })$ is a positive instance of $%
k$\textsc{-Colorability}. Then there exists an assignment function $\mu $
(similar as above but in the other direction), such that $\mu
:c(v)\rightarrow fr(t)$ for all $v\in V(G_{C})$ and for all $t\in T(G_{F})$.
Since there exists a bijection on both vertex sets of the graphs $G_{C}$ and 
$G_{F}$, i.e. both graphs are \textit{isomorph}, $(G_{F},k)$ is also a
positive instance of \textsc{Frequency-Assignment}.

\bigskip 



\begin{exercise}
  \label{ex:CO-NP}
  Fomally prove that logical entailment is $co-\NP$-complete. The formal definition of entailment ( $\models$) is this: $\alpha \models \beta$ if and only if, in every truth assignment in which $\alpha$ is true, $\beta$ is also true.  
\end{exercise}


\subsection{solution}

By trying to prove that $\vDash$ is \textit{co-NP-complete} we whave to show that
\begin{itemize}
 \item entailment is in \textit{co-NP} \dots Membership (through reduction) and
 \item NP-hardness \dots co-NP-complete Problem can be reduced to \textit{entailment}
\end{itemize}

\subsubsection{Membership}
And the counterprove for us that we have to show is that:
$\alpha \vDash \beta \iff \alpha \wedge \neg\beta$ isn't satisfiable.

\noindent We try to show membership through a reduction from \textit{entailment} the well known 
co-NP-complete problem: \textit{co-SAT}.\newline

\noindent Proof $\Rightarrow$:\\
We take a look at \textbf{every truth assignments} for \textit{entailment} what means 
that $\alpha \wedge \neg\beta$ isn't satisfied and therefore \textbf{also true}. \\
The last case - a \textbf{false} \textit{entailment} ($\alpha = true$ and $\beta = false$) 
followes a satisfied $\alpha \wedge \neg\beta$ which is \textbf{also false}.

\noindent Proof $\Leftarrow$:\\
if $\alpha \wedge \neg\beta$ isn't satisfiable the entailment is true.\\
So in case $\alpha \wedge \neg\beta$ is true ($\alpha = true$ and $\beta = false$)
the entailment is not valid too.

\subsubsection{NP-hardness}
We try to show a reduction from \textit{co-SAT} to \textit{entailment}\\

$\alpha \wedge \neg\beta$ isn't satisfiable $\iff \alpha \vDash \beta$\\
And fortunately we already proved this reduction while showing Membership above.\\

\noindent What show's us that \textit{entailment} is NP-hard and complete.



\begin{exercise}
  \label{ex:Colors}
  It is well known that the \textbf{k-COLORABILITY} problem is $\NP$-complete for every $k \geq 3$. Recall that the instance of \textbf{k-COLORABILITY} is an undirected graph $G = (V, E)$. Suppose that we restrict this instance of \textbf{k-COLORABILITY} to trees. Can the restricted problem be solved with an algorithm that runs in polynomial time? If yes, provide such an algorithm. 
\end{exercise}


\subsection{solution}

\subsubsection{Algorithm}
\begin{algorithm}[h]
\small
\caption{DFS}

\begin{algorithmic}
    \FOR{each vertex $u \in V[G]$}
       \STATE color[u] $\gets$ k
       \STATE $\pi[u] \gets NULL$
    \ENDFOR
    \STATE $time \gets 0$
\end{algorithmic}
\end{algorithm}
% 
% \smallskip
% 
% \begin{algorithmic}
% \Function{DFS-Visit}{$u$}
%     \State color[u] $\gets$ k
%     \State d[u] $\gets$ time $\gets$ time + 1
%     \For {each $v \in Adj[u]$}
%       \State $\triangleright$ explore edge $(u,v)$
%       \If {color[v] = k}
%           \State $\pi[v] \gets u$
%           \State DFS-Visit($v$)
%       %\EndIf
%     %\EndFor
%     \State color[v] $\gets$  k
%     \State f[u] $\gets$ time $\gets$ time + 1
% \EndFunction
% \end{algorithmic}


\subsubsection{Description}
The definition of the tree saies that:
\begin{itemize}
 \item there are no cycles allowed
 \item paths are connected
\end{itemize}



\begin{exercise}
  \label{ex:Nqueens}
  Provide a reduction of \textbf{N-Queens} problem to \textbf{SAT}. Give a proof sketch of the correctness of your reduction. Does this implies that the \textbf{N-Queens} is an $\NP$-complete problem? Argue your answer.  
  
  \smallskip 
  
\noindent In the \textbf{N-Queens} problem we are given $n$ queens and an $n \times n$ chessboard. We ask whether we can place these $n$ queens on the  chessboard such that no two queens attack each other. Two queens attack each other if they are placed in the same row, or in the same column, or in the same diagonal.    
       
   
\end{exercise}


\subsection{solution}

test



\begin{exercise}
  Consider the following problem:
  \begin{center}
    \fbox{
      \begin{minipage}[c]{.95\linewidth}
        \textbf{N-SORTED-ELEMENTS}

        \medskip

        INSTANCE: A non-empty list $L=(e_1,\ldots,e_n)$ of non-negative integers. \\
        QUESTION: Does the list $L$ contain a sub-list of $k$ consecutive sorted numbers in ascending order (from left to right)?
      \end{minipage}
    }
  \end{center}

  \medskip Argue that \textbf{N-SORTED-ELEMENTS} can be solved using only logarithmic
  space.
\end{exercise}


\begin{solution}

test

\end{solution}


\begin{exercise}
  \label{ex:turing}

  Design a Turing machine that increments by one a value represented by a string of 0s and 1s.

\end{exercise}


\subsection{solution 10}

$M=(K,\Sigma, \delta, start)$\\
$K=\{s, l, r', r, e\}$\\
$\Sigma=(0,1,\triangleright, \sqcup)$

\begin{center}
\begin{tabular}[t]{| l | l | l |}
\hline
$p \in K$ & $\sigma \in \Sigma $ & $ \delta ( p, \sigma ) $ \\ \hline
s & $\triangleright$ & (s, $\triangleright$, $\rightarrow$ )\\ 
s & 0 & (s, 0, $\rightarrow$ )\\ 
s & 1 & (s, 1, $\rightarrow$ )\\ 
s & $\sqcup$ & (ok, 1, --)\\ \hline
l & 0 & (ok, 1, -- )\\ 
l & 1 & (l, 0, $\leftarrow$ )\\ 
l & $\triangleright$ & (r, $\triangleright$, $\rightarrow$ )\\ \hline
r' & 0 & (r, 1, $\rightarrow$ )\\ \hline
r & 0 & (r, 0, $\rightarrow$ )\\ 
r & $\sqcup$ & (e, 0, $\rightarrow$ )\\ \hline
e & $\sqcup$ & (ok, $\sqcup$, -- )\\ \hline
\end{tabular} 
\end{center}

\subsubsection{High lvl definition}
\begin{itemize}
\item state s:
The tourig-machine M starts at state \textbf{s} and then checks if it got an end sign.
If this is the case it writes \textbf{1} end stops with positive validity 
furthermore scans M the whole input from left to right without changes and jumps to state
l if it got the $\sqcup$ sign
\item state l:
Directed to the left it tries to fill in a \textbf{1} on the first position with a \textbf{0}
and ends with positive validity afterwards.
If this isn't effective it flips every \textbf{1} to 0 
gets back to the $\triangleright$ and jumps in state r'
\item state r':
It fills position with a \textbf{1} and changes to state \textbf{r}
\item state r: 
It goes right without changes till the $\sqcup$ sign, writes a \textbf{0} 
instead of $\sqcup$ and jumps to state \textbf{e}
\item state e:
Goes right, writes an $\sqcup$ and stops.

\end{itemize}
%%%%%%%%%%%%%%%% END EXERCISES




\end{document}


