\solution{
	In graph theory, i.e. in \textit{control flow graphs} a dominator node is
	defined as follows:

	\begin{definition}
	Let $G=(V,E, n_{0})$ be a directed control flow graph (CFG) with the start node $n_0 \in V(G)$.
	A node $u \in V(G)$ dominates a node $x \in V(G)$, if every path from $x$ to a node $\kappa$ must go through $u$.
	I.e. all path from $x$ to $\kappa$ contains vertex $u$.\\
	This is written as $u \mathrm{dom} x$ (or $u \gg x$).
	\label{def:dominator_node}
	\end{definition}

	\begin{lemma}
	By definition, every node $n \in V(G)$ dominates itself.
	\end{lemma}

	Then in an implication graph $\mathrm{IG}$, a UIP $u$ dominates the decision
	node $x$ with respect to the conflict node $\kappa $, i.e. a UIP\ is a 
	\textit{dominator} in $\mathrm{IG}$, and represents an alternative decision
	assignment at the current decision level that results in the same conflict.
	An implication graph can have more than one implication point (UIP), but
	there is always at least one unique UIP, namely the \textit{decision node}.
	The main goal for searching UIPs is to reduce the number of learnt clauses.
	If the decision node is the only UIP at decision level $d$, then of course
	it is the first UIP which is closest to the conflict node. If in an
	implication graph $\mathrm{IG}$ are more than one UIP, then only one of them
	can be the first UIP at decision level $d$. This can be proven als follows:

	\smallskip

	Assume in an implication graph $\mathrm{IG}$ that there are existing more
	than one UIP. By definition (\ref{def:dominator_node}), the UIPs (dominator
	nodes) lie on every path $p$ from the decision node $x$ to the conflict node 
	$\kappa $, including the shortest path $p_{i}$ from $x$ to $\kappa $. Then
	only one of the UIPs can be a dominator node. Assume that there is a
	implication graph $\mathrm{IG}$ with two UIPs and there is a path $p_{0}$ in 
	$\mathrm{IG}$ which is the shortest path from $x$ to one of the two UIPs,
	called $u_{1}$. Then $p_{0}$ is followed by the shortest path $p_{1}$ from $%
	u_{1}$ to the next UIP\ $u_{2}$ (s.t. $u_{1}\neq u_{2}$), followed by the
	shortest path $p_{2}$ from $u_{2}$ to the conflict node $\kappa $. Then the
	shortest path $p$ from $u_{1}$ to $\kappa $ consists of the path $p_{1}$ and 
	$p_{2}$ which is longer than the path $p_{2}$ only. Hence, $u_{1}$ can not
	be the first UIP, since $u_{2}$ is closer to the conflict node $\kappa $.
	Thus, there is only one first UIP.

	(In addition the proof can be extended in a more general form with $i$ paths
	instead of two.)

	\bigskip
}
