\solution{
	In general, a formula $\psi $ is a \textit{logical consequence} of a formula 
	$\varphi $, denoted $\varphi \models \psi $, if for all truth assignments $%
	\pi $, $\pi \models \varphi $ implies $\pi \models \psi $.

	\medskip

	The learned clauses in an implication graph are a \textit{logical consequence%
	} of the original clauses and it prohibits the same conflict assignement in
	the further search. Thus, the learned clauses only adds redundancy to the
	original clauses.

	\textbf{Proof}:

	Let $\mathcal{C}=\{C_{1},\dots ,C_{n}\}$ be a countable set of clauses and
	let $C_{l}$ be the learnt clause derived from the set of clauses $\mathcal{C}
	$ using resolution.

	As usual a set of clauses $\mathcal{C}=\{C_{1},\dots ,C_{n}\}$ can be
	interpreted as the formula $\varphi =\bigwedge_{i=1}^{n}C_{i}$. Then, 
	\begin{eqnarray*}
	\mathcal{C} &\models &C_{l}\text{\quad \textit{iff\quad }}\Forall\pi \text{,
	s.t. }\pi \models \mathcal{C}\Rightarrow \,\pi \models C_{l}\,, \\
	&&\text{i.e. }\bigwedge\nolimits_{i=1}^{n}C_{i}\text{ is satisfiable\quad 
	\textit{iff\quad }}
	\end{eqnarray*}

	Using following equivalence, $(a\AND b)\equiv \lnot (a\IMPL\lnot b)$,%
	\begin{eqnarray*}
	\text{\textit{iff}\quad }\Forall\pi \text{, }\pi &\vDash &\lnot
	(\bigwedge\nolimits_{i=1}^{n}C_{i}\IMPL\lnot C_{l})\text{ is satisfiable,} \\
	\text{\textit{iff}\quad }\Forall\pi \text{, }\pi &\nvDash
	&\bigwedge\nolimits_{i=1}^{n}C_{i}\IMPL\lnot C_{l}\text{ is not satisfiable,}
	\\
	\text{\textit{iff}\quad }\Forall\pi \text{, }\pi &\nvDash &\lnot
	(\bigwedge\nolimits_{i=1}^{n}C_{i}\AND C_{l})\text{ is not satisfiable,} \\
	\text{\textit{iff}\quad }\Forall\pi \text{, }\pi &\vDash
	&\bigwedge\nolimits_{i=1}^{n}C_{i}\AND C_{l}\text{ is satisfiable,} \\
	\text{\textit{iff}\quad }\Forall\pi \text{, }\pi &\nvDash
	&\bigwedge\nolimits_{i=1}^{n}C_{i}\AND\lnot C_{l}\text{ is \textit{%
	unsatisfiable}.}
	\end{eqnarray*}

	Moreover, if $\mathcal{C}\vdash C_{l}\,\Rightarrow \,\mathcal{C}\models
	C_{l} $, then the deductive system is \textit{sound}.

	Thus, if $\mathcal{C}\models C_{l}$ then $\mathcal{C}\cup \{\lnot C_{l}\}$
	is not satisfiable (\textit{inconsistent}), i.e. $\mathcal{C}\nvdash \lnot
	C_{l}$ for some $C_{l}$ such that,%
	\begin{equation*}
	\mathcal{C}\models C_{l}\text{\quad \textit{iff\quad }}\mathcal{C}%
	_{o}\models C_{l}\text{ \ for some finite set }\mathcal{C}_{o}\subseteq 
	\mathcal{C}.
	\end{equation*}

	Recalling the \textit{compactness theorem} of propositional calculus:

	\begin{definition}
	Let $S$ be a countable infinite set of formulas. If every finite subset of $S$ is satisfiable,
	then $S$ is satisfiable.
	\label{def:compact_thm}
	\end{definition}

	Then by the compactness theorem there exists a finite subset $\mathcal{C}%
	_{o}\subseteq \mathcal{C}$ such that $\mathcal{C}_{o}\cup \{\lnot C_{l}\}$
	is inconsistent, i.e. it has no models. Then $\mathcal{C}_{o}\models C_{l}$,
	since we have a unsatisfiable set of the form $\mathcal{C}_{o}\cup \{\lnot
	C_{l}\}$, where $\mathcal{C}_{o}\subseteq \mathcal{C}$. Thus, by the \textit{%
	extended completeness theorem} of propositional calculus $\mathcal{C}%
	_{o}\vdash C_{l}$ and hence $\mathcal{C}\vdash C_{l}$.

	\bigskip
}
