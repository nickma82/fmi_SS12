\solution{

A formular $\upphi$ is a logical consequence of a formular $\upphi$
denoted  $\upphi \vDash \upphi'$, 

if $\Pi \vDash \upphi$ implies $\Pi \vDash \upphi'$ for all truth
assignements $\Pi$.

\textbf{Proof}:\\
$$C=\{C_1, \dots, C_n \} = \bigwedge_{i=1}^{n}C_i$$

$\overbrace{\bigwedge{Ci}}^{C} \vDash C_l$ iff

$\forall \Pi$ s.t. $\Pi \vDash \bigwedge{C_i} \Rightarrow \Pi \vDash C_l$\\

i.e. $\underbrace{\bigwedge{C_i}}_{C}$ is satisfied:

\begin{itemize}
 \item iff $\bigwedge{C_i} \land C_l$ is satisfied *\\
Equivalence rule:
* $(a \land b) \equiv \neg (a \rightarrow \neg b)$
 \item iff $\Pi \nvDash \bigwedge{\neg C_i \rightarrow C_l}, \forall\Pi$ is unsatisfied
 \item iff $\Pi \nvDash \bigwedge{C_i \land \neg C_l}$ is unsatisfied
\end{itemize}

\textbf{Substitution rule}:\\
We transform stepwise
\begin{center}
 iff $Pi \vDash \neg(\bigwedge{C_i} \rightarrow \neg C_l)$ \\
 iff $Pi \nvDash \bigwedge{C_i} \rightarrow \neg C_l$\\
 iff $Pi \nvDash \neg(\bigwedge{C_i} \land C_l)$
\end{center}

And therefor it follows:
\begin{center}
 iff $Pi \vDash \bigwedge{C_i} \land C_l$
\end{center}

learned clauses are a logical consequence of the original clauses and 
it prohibits the same conflict assignement in the further search.\\
$\Rightarrow$ learned clauses only adds redundancy to the original clauses.

\textbf{Proof}

$\Rightarrow$ if $C \vdash C_l$ then $\Rightarrow C \nvDash C_l$ the 
the deductive system is sound.\\
Suppose that $C$ is a countable set of proposition formulas i.e. 
set of clauses.

Then $C \cup \{\neg C_l\}$ is inconsistent i.e. $C \nvdash C_l$ for 
some $C_l$.

By the compactness theorem of propositional clauses there $\exists$
a finite subset $C_o \subseteq C$ s.t. $C_o \cup \{ \neg C_l\}$ is 
inconsistent i.e. has no models.\\

Then $C \vDash C_l$\\
By the existing completeness theorem of propositional calculus $C_o \vdash C_l$
Hence $C \vdash C_l$.

}