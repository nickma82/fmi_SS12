\documentclass[11pt,a4paper]{uebung}

\usepackage[british]{babel}
\usepackage{epsfig}
\usepackage{rotate}
\usepackage{amsmath,amsthm,amssymb}
\usepackage{color}
\makeatletter\let\@amsfonts=P\makeatother
\usepackage{graphicx}
\usepackage{typearea}
\usepackage{multicol}
\usepackage{amsfonts}
\usepackage[nounderscore]{syntax}
\usepackage{enumitem}
\newcommand{\comment}[1]{\marginpar{\small{\bf Comment:} #1}}

\usepackage{tikz}
\usetikzlibrary{shapes,arrows,backgrounds,%
matrix,patterns,arrows,decorations.pathmorphing,decorations.pathreplacing,%
positioning,fit,calc,decorations.text,shadows%
}

\newcommand{\solution}[1]{\par {\bf Solution:}\\#1}



%put your Matrikelnummer here instead of the XXXXXXXX
% if your group has less than 3 members, just delete the remaining XXXXXXXX
\newcommand\matrikelnummerA[0]{XXXXXXXX}
\newcommand\matrikelnummerB[0]{XXXXXXXX}
\newcommand\matrikelnummerC[0]{XXXXXXXX}
%put your Matrikelnummer here instead of the XXXXXXXX



\def\cT{\mathcal{T}}


\begin{document}
\newcommand{\Vorlesung}{Formal Methods in Computer Science}
\newcommand{\Semester}{SS 2012}
\newcommand{\Prof}{Uwe Egly}
\newcommand{\AssisA}{Antonius Weinzierl}
\newcommand{\AssisB}{}

%%%%%%%%%%%%%%%%%%%%%%%%%%%%%%%%%%%%%%%%%%%%%%%%%%%%%%%%%%%%%%%%%%%%%%%%%%%%%%

\Uebungsblatt{2 (10 points)}{
  \begin{tabular}{rl}
   Matrikelnummer(n): &\matrikelnummerA \\
   &\matrikelnummerB \\
   &\matrikelnummerC
  \end{tabular}
}

%%%%%%%%%%%%%%%%%%%%%%%%%%%%%%%%%%%%%%%%%%%%%%%%%%%%%%%%%%%%%%%%%%%%%%%%%%%%%%


\Aufgabe[Tseitin Transformation \hfill \bf (0.5 + 1 + 1.5 points)]

\begin{enumerate}
\item Extend Tseitin's transformation for the connectives $\leftrightarrow$
  (equivalence) and $\oplus$ (XOR). Find the necessary clauses for the new schemes
  $l_i \leftrightarrow (l_{i'} \leftrightarrow l_{i''})$ and $l_k
  \leftrightarrow (l_{k'} \oplus l_{k''})$.
  
  \solution{
  }

\item Apply Tseitin's transformation to the following formula $\psi$: $a \rightarrow
  \big( b \lor \neg (a \leftrightarrow c)\big)$.
  
  Hint: You do not need to introduce labels for propositions $a,b,$ and $c$.
  \solution{
  }

\item Let $\psi$ be a propositional formula and $D^\psi$ the set of clauses
  resulting from Tseitin's transformation on $\psi$. Prove that the following
  holds:
  
  \centerline{If $\psi$ is satisfiable then $D^\psi$ is satisfiable.}

  You only need to prove this for the connectives $\land$ and $\neg$.
  %\lor,\neg, \rightarrow$.
  Use the below clause schemes, which introduce a new label for every boolean
  variable.
  \begin{align*}
    L_a \leftrightarrow a && (\neg L_a \lor a)&& (L_a \lor \neg a)\\
    L_\phi \leftrightarrow (L_1 \land L_2) && (\neg L_\phi \lor L_1)&& (\neg
    L_\phi \lor L_2)&& (L_\phi \lor \neg L_1 \lor \neg L_2)\\
    L_\phi \leftrightarrow \neg L_1 && (\neg L_\phi \lor \neg L_1)&& (L_\phi
    \lor L_1)
  \end{align*}
  
  \solution{
  }
\end{enumerate}


%%%%%%%%%%%%%%%%%%%%%%%%%%%%%%%%%%%%%%%%%%%%%%%%%%%%%%%%%%%%%%%%%%%%%%%%%%%%%%

\newpage
\Aufgabe[Implication Graphs \hfill \bf (2+1+1.5 points)]
\begin{enumerate}
\item Let $\mathcal{D}$ be the following set of clauses:
  \begin{align*}
    c_1:& (A \lor B)\\
    c_2:& (A \lor G \lor H)\\
    c_3:& (\neg B \lor \neg D \lor E)\\
    c_4:& (E \lor F)\\
    c_5:& (\neg F \lor \neg G \lor D)\\
    c_6:& (\neg C \lor G \lor J)\\
    c_7:& (\neg J \lor \neg H)
  \end{align*}
  Draw the implication graph resulting from $\mathcal{D}$ with decisions
  $A=0@1$, $C=1@2$, $E=0@3$. Find the first UIP, and learn a new clause using
  the first-UIP scheme (use resolution).

  \solution{
  }

\item Prove that in a conflict graph the first UIP is uniquely defined, i.e.,
  prove that there is exactly one node in the graph which is a first UIP.

  \solution{
  }

\item Let $\mathcal{C}$ be a set of clauses and $G$ a conflict graph with
  respect to $\mathcal{C}$. Prove: if a clause $C_l$ is learned following the
  first-UIP scheme, then $C_l$ is a consequence of $\mathcal{C}$.

  \solution{
  }
\end{enumerate}


%%%%%%%%%%%%%%%%%%%%%%%%%%%%%%%%%%%%%%%%%%%%%%%%%%%%%%%%%%%%%%%%%%%%%%%%%%%%%%

\newpage
\Aufgabe[Sparse Method \hfill \bf (1.5 points)]
Apply the Sparse Method including preprocessing on the formula $\varphi^E$
below to obtain a propositional formula.
\begin{displaymath}
  (x_1 \neq x_2 \lor x_2=x_3 ) \land \big[ (x_2 \neq x_4 \land x_3=x_4
  \land x_4=x_5)
  \lor (x_6 \neq x_5 \land x_6=x_7 \land x_7=x_3)\big]
\end{displaymath}

  \solution{
  }


%%%%%%%%%%%%%%%%%%%%%%%%%%%%%%%%%%%%%%%%%%%%%%%%%%%%%%%%%%%%%%%%%%%%%%%%%%%%%%

\newpage
\Aufgabe[Ackermann's Reduction \hfill \bf (1 point)]
Apply Ackermann's reduction on the following EUF-formula $\varphi$ to obtain
an EU formula:
\begin{displaymath}
  f\left(f\left(g\left(a\right),b\right),a\right) = f(g(a),b) \rightarrow \big[ f(x,y) = g(f(g(a),b)) \land
  g(f(a,y))=d \big]
\end{displaymath}


\solution{
}


\end{document}
