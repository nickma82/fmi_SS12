\textbf{Solution a):}\newline
\\
%\ For better reading we define the following notation:\\

For better reading, the notation $\sigma_{a,b}$ describes the assignment of the variables $x$ and $y$ to the values $a$ and $b$ respectively, such that:
$$\sigma_{a,b}\, \equiv\, \sigma\langle x \mapsto a, y \mapsto b \rangle$$
and so $\sigma(x) = \sigma(y) = 1 \equiv \sigma_{1,1}$.\\

\bigskip
$[p]\sigma)=$\\
\indent$(p,\sigma_{1,1}) =  (x:=x+y; \text{ if } x<0 \text{ then } ... \text{ fi}, \sigma_{1,1})\\
\indent(x:=x+y, \sigma_{1,1}) \Rightarrow \sigma\langle x \mapsto \underbrace{[x+y]\sigma_{1,1}}_{\equiv[2]\sigma_{1,1}} \rangle = \sigma_{2,1}\\
\Rightarrow (\text{if } x<0 \text{ then } abort \text{ else } while \text{ } x \neq y \text{ } do \text{ } ... \text{} od \text{ fi},\sigma_{2,1})\\
\Rightarrow (\text{while } x \neq y \text{ do } x:=x+1; y;=y+2; \text{ od},\sigma_{2,1})[x<0]=0$ (false)\\
$\Rightarrow (x:=x+1; y:=y+2; \text{while } x \neq y \text{ do } x:=x+1; y;=y+2; \text{ od},\sigma_{2,1})\\
\Rightarrow (x:=x+1; y:=y+2,\sigma_{2,1})\\
\text{ }\hspace{2cm} (x:=x+1, \sigma_{2,1}) \Rightarrow \sigma(x \mapsto \underbrace{[x+1]\sigma_{2,1}}_{\equiv[3]\sigma_{2,1}})= \sigma_{3,1}\\
\Rightarrow (\text{while } x \neq y \text{ do } x:=x+1; y;=y+2; \text{ od},\sigma_{3,1})\\
\text{ }\hspace{2cm} (y:=y+2, \sigma_{3,1}) \Rightarrow \sigma(y \mapsto \underbrace{[y+2]\sigma_{3,1}}_{\equiv[3]\sigma_{3,1}})= \sigma_{3,3}\\
\Rightarrow (\text{while } x \neq y \text{ do } x:=x+1; y;=y+2; \text{ od},\sigma_{3,3})\\
\Rightarrow \sigma_{3,3} [x \neq y]\sigma_{3,3}=0$ (false)
$\Rightarrow \sigma_{3,3}$
\bigskip\\