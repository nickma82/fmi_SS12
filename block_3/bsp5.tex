\newcommand\Ext[2]{\infer[]{#1}{#2}}
\newcommand\PC[2]{\infer[\text{\tiny(pc)}]{#1}{#2}}
\newcommand\TC[2]{\infer[\text{\tiny(tc)}]{#1}{#2}}

\textbf{Solution:}\newline

The syntax of the TPL-language can be extended as follows:%
\begin{equation*}
\mathcal{P}::=\mathsf{skip}\mid \mathsf{abort}\mid \mathcal{V}:=\mathcal{E}%
\mid \mathcal{P};\mathcal{P}\mid \mathsf{if}\;\mathcal{E}\;\mathsf{then}\;%
\mathcal{P}\;\mathsf{else}\;\mathsf{P}\;\mathsf{fi}\mid \mathsf{while}\;%
\mathcal{E}\;\mathsf{do}\;\mathcal{P}\;\mathsf{od}\mid \kw{assert}\ \mathcal{%
E}
\end{equation*}

where,%
\begin{eqnarray*}
\mathcal{E} &:&:=\mathcal{V}\mid \mathcal{N}\mid \mathcal{UE}\mid (\mathcal{%
EBE}) \\
\mathcal{V} &:&:=x\mid y\mid z\mid \ldots \mid x_{0}\mid x_{1}\mid \ldots \\
\mathcal{N} &:&:=0\mid 1\mid 2\mid 3\mid \ldots \\
\mathcal{U} &:&:=+\mid -\mid \lnot \mid \ldots \\
\mathcal{B} &:&:=+\mid -\mid \ast \mid /\mid <\mid \ldots
\end{eqnarray*}

Beside the syntax extension of \textrm{TPL}, the semantics for $\kw{assert}\
e$ can be defined as a \textsl{first-class citizen} (this means that no
other program-statements are used), as follows:

\begin{enumerate}
\item \textit{Natural Semantics (NS):}%
\begin{equation*}
\lbrack \kw{assert}\ e]\sigma =\sigma \;\text{if }[e]\sigma \neq 0
\end{equation*}

\item \textit{Structural Operational Semantics (SOS):}%
\begin{equation*}
(\kw{assert}\ e,\sigma )\,\Rightarrow \,\sigma \;\text{if }[e]\sigma \neq 0
\end{equation*}

The transition relation for \textrm{SOS} and also for the rule of \textrm{NS}
are only \textit{defined} for $[e]\sigma \neq 0$. If $[e]\sigma =0$, then
the transition of the rule in \textrm{SOS}\ and \textrm{NS} is \textit{%
undefined}, since $\kw{assert}\ $\textit{false} get stuck in any state $%
\sigma \in \mathcal{S}$.

\item \textit{Hoare Calculus Rules:}

\begin{itemize}
\item \textsl{partial correctness:}{\small \ 
\begin{equation*}
\infer[]{ \CA{F}{\kw{assert}\ e}{G} }{ \Ia{ \CA{F}{\IF\ e\ \THEN\ \SKIP\
\ELSE\ \ABORT\ \FI}{G} } { \La{ \CA{F\land c}{\SKIP}{G} } {(F\land
e)\Rightarrow F^{\prime}} { \Ext{ \CA{F^{\prime}}{\SKIP}{F^{\prime}} } {-} }
{F^{\prime}\Rightarrow G} } { \Lb{ \CA{F\land \neg e}{\ABORT}{G} }
{(F\land\neg e) \Rightarrow F^{\prime} } { \Ext{ \CA{F^{\prime}}{\ABORT}{G}
} {-} } } }
\end{equation*}%
}

Additionally it is to mention that on the right side of the above derivation
tree the postcondition $G$ will be never reached, since for the $\kw{abort}$%
-statement no state-transition is defined.

Moreover, $\kw{assert}\ $\textit{false} is not semantically equivalent to $%
\kw{while}\ \text{\textit{true}}\ \kw{do}\ \kw{skip}\ \kw{od}$ (which is
semantically equiv. to $\kw{abort}$), since the derivation sequence of $%
\kw{while}\ $\textit{true}$\ \kw{do}\ \kw{skip}\ \kw{od}$ is \textit{infinite%
} from any state $\sigma \in \mathcal{S}$ and in contrast, $\kw{assert}\ $%
\textit{false} gets \textsc{stuck} in any state $\sigma $. Additionally, $%
\kw{assert}\ $\textit{false} is also not semantically equivalent to $%
\kw{skip}$, since with $\kw{skip}$ for any state $\sigma $, a terminal
configuration will be reached. \bigskip

Then from the above tree we have following implication forumlas:%
\begin{eqnarray*}
\left. 
\begin{array}{l}
(F\,\wedge \,e)\,\Rightarrow \,F^{\prime } \\ 
(F\,\wedge \,\lnot e)\,\Rightarrow \,F^{\prime }%
\end{array}%
\right\} &&((F\,\wedge \,e)\,\vee \,(F\,\wedge \,\lnot e))\,\Rightarrow
\,F^{\prime } \\
&\equiv &\lnot ((F\,\wedge \,e)\,\vee \,(F\,\wedge \,\lnot e))\,\vee
\,F^{\prime } \\
&\equiv &\lnot ((F\,\wedge \,e)\,\vee \,(F\,\wedge \,\lnot e))\,\vee
\,F^{\prime } \\
&\equiv &((\lnot F\,\vee \,\lnot e)\,\wedge \,(\lnot F\,\vee \,e))\,\vee
\,F^{\prime } \\
&\equiv &(\lnot F\,\vee \,\lnot e\,\vee \,F^{\prime })\,\wedge \,(\lnot
F\,\vee \,e\,\vee \,F^{\prime }) \\
&\equiv &(F\,\Rightarrow \,(\lnot e\,\vee \,F^{\prime }))\,\wedge \,%
\underbrace{(F\,\Rightarrow \,(e\,\vee \,F^{\prime }))}_{true} \\
&\equiv &(F\,\Rightarrow \,(e\,\Rightarrow \,F^{\prime }))
\end{eqnarray*}

Since $F^{\prime }\,\Rightarrow \,G$, it follows that%
\begin{equation*}
F\,\Rightarrow \,(e\,\Rightarrow \,G).
\end{equation*}

Then the Hoare rule for partial correctness of $\kw{assert}\ e$ is%
\begin{equation*}
\PC { \CA{F}{\kw{assert}\ e}{G} } { F\Rightarrow (e\Rightarrow G)}
\end{equation*}

From this we can derive the weakest liberal precondition (\textit{wlp}) for
partial correctness, which guarantees that the post condition is met if the
program terminates, i.e. is the same as the weakest precondition (\textit{wp}%
), except the program may not terminate. The inference rules for the weakest
liberal precondition are the same as for the \textit{wp} except for:%
\begin{equation*}
wlp(\kw{assert}\ e,G)=\lnot e\,\vee \,G\equiv e\,\Rightarrow \,G.
\end{equation*}

\item \textsl{total correctness:}

The derivation of the total correctness rule for $\kw{assert}\ e$ can be
done as following by using the $\kw{abort}$-rule for total correctness: 
{\small 
\begin{equation*}
\Ext{ \CA{F}{\kw{assert}\ e}{G} } { \Ia{ \CA{F}{\IF\ e\ \THEN\ \SKIP\ \ELSE\
\ABORT\ \FI}{G} } { \La{ \CA{F\land c}{\SKIP}{G} } {(F\land e)\Rightarrow
F'} { \Ext{ \CA{F'}{\SKIP}{F'} } {-} } {F'\Rightarrow G} } { \Lb{ \CA{F\land
\neg e}{\ABORT}{G} } {(F\land\neg e) \Rightarrow F' } { \Lb{
\CA{F'}{\ABORT}{G} } {F'\Rightarrow \FALSE} { \Ext{ \CA{\FALSE}{\ABORT}{G} }
{-} } } } }
\end{equation*}%
}

From this derivation tree we get following implication formulas, such that:%
\begin{eqnarray*}
\left. 
\begin{array}{l}
(F\,\wedge \,e)\,\Rightarrow \,F^{\prime } \\ 
(F\,\wedge \,\lnot e)\,\Rightarrow \,F^{\prime }%
\end{array}%
\right\} &&((F\,\wedge \,e)\,\vee \,(F\,\wedge \,\lnot e))\,\Rightarrow
\,F^{\prime } \\
&\equiv &(\lnot F\,\vee \,\lnot e\,\vee \,F^{\prime })\,\wedge \,(\lnot
F\,\vee \,e\,\vee \,F^{\prime })
\end{eqnarray*}

and since $F^{\prime }\,\Rightarrow \,false$ and $F^{\prime }\,\Rightarrow
\,G$,%
\begin{eqnarray*}
&\equiv &(\lnot F\,\vee \,\lnot e\,\vee \,G)\,\wedge \,(\lnot F\,\vee
\,e\,\vee \,false) \\
&\equiv &\lnot F\,\vee \,((\lnot e\,\vee \,G)\,\wedge \,(e\,\vee \,false)) \\
&\equiv &\lnot F\,\vee \,((\lnot e\,\vee \,G)\,\wedge \,e) \\
&\equiv &\lnot F\,\vee \,(\underbrace{(\lnot e\,\wedge \,e)}_{false}\,\vee
(e\,\wedge G)) \\
&\equiv &\lnot F\,\vee \,(e\,\wedge \,G)\equiv F\,\Rightarrow \,(e\,\wedge
\,G).
\end{eqnarray*}

Finally we get following rule of $\kw{assert}\ e$ for total correctness:%
\begin{equation*}
\TC { \CA{F}{\kw{assert}\ e}{G} } { F\Rightarrow (e\land G)}
\end{equation*}

From the total correctness rule of $\kw{assert}\ e$, we can derive the
weakest precondition (\textit{wp}) and the strongest postcondition (\textit{%
sp}), such that%
\begin{eqnarray*}
wp(\kw{assert}\ e,G) &=&e\,\wedge \,G\;\text{and} \\
sp(F,\kw{assert}\ e) &=&\left\{ 
\begin{array}{l}
F\,\wedge \,e\;\text{if }[e]\sigma \neq 0 \\ 
F\,\wedge \,\lnot e\;\text{if }[e]\sigma =0%
\end{array}%
\right. 
\end{eqnarray*}

The strongest postcodition of $\kw{assert}\ e$ has two possebilities.
Especially if $\kw{assert}\ e$ fails, then we know that $\lnot e$ is true
and $sp(F,\kw{assert}\ e)=F\,\wedge \,\lnot e$.
\end{itemize}
\end{enumerate}




% \newcommand\Ext[2]{\infer[]{#1}{#2}}
% \newcommand\PC[2]{\infer[\text{\tiny(pc)}]{#1}{#2}}
% \newcommand\TC[2]{\infer[\text{\tiny(tc)}]{#1}{#2}}

% \textbf{Solution:}\newline

% The syntax of the TPL-language can be extended as follows:%
% \begin{equation*}
% \mathcal{P}::=\mathsf{skip}\mid \mathsf{abort}\mid \mathcal{V}:=\mathcal{E}%
% \mid \mathcal{P};\mathcal{P}\mid \mathsf{if}\;\mathcal{E}\;\mathsf{then}\;%
% \mathcal{P}\;\mathsf{else}\;\mathsf{P}\;\mathsf{fi}\mid \mathsf{while}\;%
% \mathcal{E}\;\mathsf{do}\;\mathcal{P}\;\mathsf{od}\mid \kw{assert}\ \mathcal{%
% E}
% \end{equation*}

% where,%
% \begin{eqnarray*}
% \mathcal{E} &:&:=\mathcal{V}\mid \mathcal{N}\mid \mathcal{UE}\mid (\mathcal{%
% EBE}) \\
% \mathcal{V} &:&:=x\mid y\mid z\mid \ldots \mid x_{0}\mid x_{1}\mid \ldots  \\
% \mathcal{N} &:&:=0\mid 1\mid 2\mid 3\mid \ldots  \\
% \mathcal{U} &:&:=+\mid -\mid \lnot \mid \ldots  \\
% \mathcal{B} &:&:=+\mid -\mid \ast \mid /\mid <\mid \ldots 
% \end{eqnarray*}

% Beside the syntax extension of TPL, the semantics for $\kw{assert}\ e$ can
% be defined as a first member citizen (this means that no other
% program-statements will be used), as follows:

% \begin{enumerate}
% \item Natural Semantics:%
% \begin{equation*}
% \lbrack \kw{assert}\ e\rbrack\sigma =\sigma \;\text{if }[e]\sigma \neq 0
% \end{equation*}

% \item Structural Operational Semantics:%
% \begin{equation*}
% (\kw{assert}\ e,\sigma )\,\Rightarrow \,\sigma \;\text{if }[e]\sigma \neq 0
% \end{equation*}


% \item Hoare Calculus Rules:
	% \begin{itemize}
	% \item for \textit{partial correctnes}:

% \begin{small}
% $
% \Ext{ \CA{F}{\kw{assert}\ e}{G} }
% {
	  % \Ia{ \CA{F}{\IF\ e\ \THEN\ \SKIP\ \ELSE\ \ABORT\ \FI}{G} }
	  % {
		% \La{ \CA{F\land c}{\SKIP}{G} }
		% {(F\land e)\Rightarrow F'}
		% {
			% \Ext{ \CA{F'}{\SKIP}{F'} }
			% {-}
		% }
		% {F'\Rightarrow G}
	  % }
	  % { 
		% \Lb{ \CA{F\land \neg e}{\ABORT}{G} }
		% {(F\land\neg e) \Rightarrow F' }
		% {
			% \Ext{ \CA{F'}{\ABORT}{G} }
			% {-}
		% }
	  % }
% }
% $
% \end{small}

% $$
% \PC { \CA{F}{\kw{assert}\ e}{G} }
% { F\Rightarrow (e\Rightarrow G)}
% $$

	% \item for \textit{total correctness}:

% \begin{small}
% $
% \Ext{ \CA{F}{\kw{assert}\ e}{G} }
% {
	  % \Ia{ \CA{F}{\IF\ e\ \THEN\ \SKIP\ \ELSE\ \ABORT\ \FI}{G} }
	  % {
		% \La{ \CA{F\land c}{\SKIP}{G} }
		% {(F\land e)\Rightarrow F'}
		% {
			% \Ext{ \CA{F'}{\SKIP}{F'} }
			% {-}
		% }
		% {F'\Rightarrow G}
	  % }
	  % { 
		% \Lb{ \CA{F\land \neg e}{\ABORT}{G} }
		% {(F\land\neg e) \Rightarrow F' }
		% {
			% \Lb{ \CA{F'}{\ABORT}{G} }
			% {F'\Rightarrow \FALSE}
			% {
				% \Ext{ \CA{\FALSE}{\ABORT}{G} }
				% {-}
			% }
		% }
	  % }
% }
% $
% \end{small}

% thus\\
% $$
% \TC { \CA{F}{\kw{assert}\ e}{G} }
% { F\Rightarrow (e\land G)}
% $$

	% \end{itemize}
% \end{enumerate}
