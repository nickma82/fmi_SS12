\textbf{Solution:}\newline

To prove the $\kw{while}$-rule for partial correctnes we have to consider
following definition: 
\begin{definition}
	Let $\{F\}\, p\, \{G\}$ be an arbitrary partial correctness assertion. Then $\{F\}\, p\, \{G\}$ is
	sound if
	\begin{equation*}
		\vdash_{par} \{F\}\, p\, \{G\}\quad \text{ then } \vDash_{par} \{F\}\, p\, \{G\}.
	\end{equation*}
\end{definition}

Then let $I$ be an arbitrary interpretation and assume that $\vDash _{%
\mathrm{par}}^{I}\{\mathrm{Inv}\,\wedge \,e\}\,p\,\{\mathrm{Inv}\}$, i.e. is
partially correct. Then for some fixed state $\tau $ which satisfies the
invariant $\mathrm{Inv}$ and $e$ in the precondition, there $\exists \tau
^{\prime }\in \mathcal{S}$ such that $[p]\tau =\tau ^{\prime }$ is defined
and $\tau ^{\prime }$ satisfies $\mathrm{Inv}$ in the postcondition. To
prove that $\vDash _{\mathrm{par}}^{I}\{\mathrm{Inv}\}\,\kw{while}\ e\ %
\kw{do}\ p\ \kw{od}\,\{\mathrm{Inv}\,\wedge \,\lnot e\}$ is valid, we have
to show for all states $\sigma \in \mathcal{S}$, such that%
\begin{equation*}
\mathrm{Inv}\,\Rightarrow \,F\quad \lbrack F]\sigma \text{ is true, then }[%
\kw{while}\ e\ \kw{do}\ p\ \kw{od}]\sigma \text{ is true with }%
G\,\Rightarrow \,(\mathrm{Inv}\,\wedge \,\lnot e).
\end{equation*}

The proof can be done by using structural induction on the given axiom rule.

We shall show by induction that $P(n),n\in \mathbb{N}$ holds, where%
\begin{eqnarray*}
P(n)\quad \overset{defined}{\Longleftrightarrow }\quad \forall \sigma
,\sigma ^{\prime \prime } &\in &\mathcal{S}\text{ such that }(\sigma ,\sigma
^{\prime \prime })\in \Theta _{n}\text{ and} \\
\sigma &\vDash &_{\mathrm{par}}^{I}\mathrm{Inv}\;\Rightarrow \;\sigma
^{\prime \prime }\vDash _{\mathrm{par}}^{I}(\in \mathcal{S}),
\end{eqnarray*}

for all $n$ iteration-steps of $\kw{while}\ e\ \kw{do}\ p\ \kw{od}$. ( $%
(\sigma ,\sigma ^{\prime \prime })\in \Theta _{n}$ denotes a partial
function.)

\bigskip

\textit{Base case:}

If $n=0$, then the induction hypothesis $P(0)$ is vacuosly true.

\bigskip

\textit{Induction step:}

We assume that the induction hypothesis $P(n)$ for $n\geq 0$ holds and try
to prove $P(n+1)$.

Let $w\equiv \kw{while}\ e\ \kw{do}\ p\ \kw{od}$, and consider some
arbitrary states $\sigma ,\sigma ^{\prime },\sigma ^{\prime \prime }\in 
\mathcal{S}$ with $[\mathrm{Inv}]\sigma =true$ (in the precondition).

We have to show that $\sigma ^{\prime \prime }\vDash _{\mathrm{par}}^{I}(%
\mathrm{Inv}\,\wedge \,\lnot e)$, i.e. $[\mathrm{Inv}\,\wedge \,\lnot
e]\sigma ^{\prime \prime }=true$. Then there are existing two cases for any $%
\tau \in \mathcal{S}$:

\bigskip

\begin{description}
\item[\textbf{case} \textit{i)}] $[e]\tau =true$. Since by assumption $%
\vDash _{\mathrm{par}}^{I}\{\mathrm{Inv}\,\wedge \,e\}\,p\,\{\mathrm{Inv}\}$
is partially correct, i.e.%
\begin{eqnarray*}
\forall \sigma  &\in &\mathcal{S}, \\
&&\text{if }[\mathrm{Inv}\,\wedge \,e]\sigma =true\;\Rightarrow \;[p]\sigma 
\text{ is defined and }[\mathrm{Inv}][p]\sigma =true.
\end{eqnarray*}

\item Then we have $\sigma \vDash _{\mathrm{par}}^{I}e$ and hence $\sigma
\vDash _{\mathrm{par}}^{I}(\mathrm{Inv}\,\wedge \,e)$.

\item Thus,%
\begin{equation*}
(w,\sigma )\,\Rightarrow \,(p;w,\sigma )\,\Rightarrow ^{\ast }\,\sigma
^{\prime \prime }.
\end{equation*}

\item Hence, $[w]\sigma =\sigma ^{\prime \prime }$ is defined and satisfies $%
\mathrm{Inv}$ in the postcondition.

\item From the induction hypothesis $P(n)$ we obtain $\sigma ^{\prime \prime
}\vDash _{\mathrm{par}}^{I}(\mathrm{Inv}\,\wedge \,\lnot e)$, since for $%
[e]\sigma \neq 0$,%
\begin{equation*}
(p,\sigma )\,\Rightarrow \,\sigma ^{\prime }\quad \text{and\quad }(w,\sigma
^{\prime })\,\Rightarrow ^{\ast }\,\sigma ^{\prime \prime }.\,
\end{equation*}

\item Since $[\mathrm{Inv}\,\wedge \,e]\sigma =true$, $[p]\sigma $ is
defined and we obtain $\sigma ^{\prime }\vDash _{\mathrm{par}}^{I}\mathrm{Inv%
}$, as $\vDash \{\mathrm{Inv}\,\wedge \,\lnot e\}\,p\,\{\mathrm{Inv}\}$.

\item Then the induction hypothesis can be applied to $[w]\sigma ^{\prime
}=\sigma ^{\prime \prime }$ which gives us $[\mathrm{Inv}\,\wedge \,\lnot
e]\sigma ^{\prime \prime }=true$.

\item[\textbf{case} \textit{ii)}] $[e]\tau =false$. Then we have $\sigma
\vDash _{\mathrm{par}}^{I}\lnot e$ and \ hence $\sigma \vDash _{\mathrm{par}%
}^{I}(\mathrm{Inv}\,\wedge \,\lnot e)$, i.e. $[\mathrm{Inv}\,\wedge \,\lnot
e]\sigma =true$. Then according to%
\begin{equation*}
(w,\sigma ^{\prime \prime })\,\Rightarrow \,\sigma \quad \text{if }[e]\sigma
^{\prime \prime }=0,
\end{equation*}

\item i.e. $[w]\sigma ^{\prime \prime }=\sigma $. This establishes the
induction hypothesis $P(n+1)$.$\quad \Rightarrow \,P(n)$ holds for all $n$
and the rule for $\kw{while}$ is \textit{sound}.

\item 
\end{description}
