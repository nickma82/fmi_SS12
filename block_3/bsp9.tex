\textbf{Solution:}\newline

The weakest precondition (\textit{wp}) guarantees termination and
corresponds to the total correctness of Hoare triples. The computation of
the \textit{wp} is defined as follows:%
\begin{equation*}
wp(\kw{while}\ e\ \kw{do}\ p\ \kw{od},G)=F_{0}(G)\,\vee \,F_{1}(G)\,\vee
\,F_{2}(G)\,\vee \ldots =\exists i\,(i\geq 0\,\wedge \,F_{i}(G)),\,
\end{equation*}

where%
\begin{eqnarray*}
F_{0}(G) &=&\lnot e\,\wedge \,G \\
F_{i+1}(G) &=&e\,\wedge \,wp(p,F_{i}(G))\quad \text{(for }i\geq 0\text{).}
\end{eqnarray*}

These computation steps ensures, that the loop terminates in a state that
satisfies $G$.

In addition, there exists a variant for the computation steps of estimating
the weakest precondition,

\begin{eqnarray*}
F_{0}(G) &=&false \\
F_{i+1}(G) &=&e\,\wedge \,wp(p,F_{i}(G))= \\
&=&(\lnot e\,\Rightarrow \,G)\,\wedge \,(e\,\Rightarrow \,wp(p,F_{i}(G))) \\
&=&(\lnot e\,\wedge \,G)\,\vee \,(e\,\wedge \,wp(p,F_{i}(G)))\quad \text{%
(for }i\geq 0\text{).}
\end{eqnarray*}

The computation of the weakest liberal precondition which does not guarantee
the termination of the loop can be done by a similar way:

\begin{eqnarray*}
wlp(\kw{while}\ e\ \kw{do}\ p\ \kw{od},G) &=&wp(\kw{while}\ e\ \kw{do}\ p\ %
\kw{od},G)= \\
&=&F_{0}(G)\,\vee \,F_{1}(G)\,\vee \,F_{2}(G)\,\vee \ldots =\forall
i\,(i\geq 0\,\wedge \,F_{i}(G)),\,
\end{eqnarray*}

where%
\begin{eqnarray*}
F_{0}(G) &=&\lnot e\,\Rightarrow \,G \\
F_{i+1}(G) &=&e\,\Rightarrow \,wp(p,F_{i}(G))\quad \text{(for }i\geq 0\text{%
),}
\end{eqnarray*}

or alternatively,
\begin{eqnarray}
F_{0}(G) &=&true  \label{wlp_step_1} \\
F_{i+1}(G) &=&(\lnot e\,\Rightarrow \,G)\,\wedge \,(e\,\Rightarrow
\,wp(p,F_{i}(G)))\quad \text{(for }i\geq 0\text{).}  \label{wlp_step_2}
\end{eqnarray}

This ensures that the $\kw{while}$-loop terminates in a state that
satisfies $G$ or runs forever.

Then the weakest liberal precondition (\textit{wlp}) for the given
formula \ASS z0;\WHILE\ $y\neq0$ \DO \ASS z{z+x};\ASS y{y-1}\OD can be
calculated as follows, which works similiar as the computation of the 
\textit{wp} for the same formula that was shown in the lecture slides. To
compute the \textit{wlp}, we use the alternative computation steps (\ref%
{wlp_step_1}) and (\ref{wlp_step_2}), then%
\begin{eqnarray*}
F_{0}(G) &=&true \\
F_{1}(G) &=&(y=0\,\Rightarrow \,z=xy_{0})\,\wedge \,(y\neq 0\,\Rightarrow
\,wlp(z:=z=x;y:=y-1,true)) \\
&=&(y=0\,\Rightarrow \,z=xy_{0})\,\wedge \,true \\
&=&(y=0\,\Rightarrow \,z=xy_{0}) \\
F_{2}(G) &=&(y=0\,\Rightarrow \,z=xy_{0})\,\wedge \,(y\neq 0\,\Rightarrow
\,wlp(z:=z+x;y:=y-1,F_{1}(G))) \\
&=&(y=0\,\Rightarrow \,z=xy_{0})\,\wedge \,(y\neq 0\,\Rightarrow
\,wlp(z:=z+x,wlp(y:=y-1,\lnot (y=0)\,\vee \,\,z=xy_{0})) \\
&=&(y=0\,\Rightarrow \,z=xy_{0})\,\wedge \,(y\neq 0\,\Rightarrow
\,wlp(z:=z+x,\lnot (y-1=0)\,\vee \,\,z=xy_{0}) \\
&=&(y=0\,\Rightarrow \,z=xy_{0})\,\wedge \,(y\neq 0\,\Rightarrow \,\lnot
(y=1)\,\vee \,\,z+x=xy_{0}) \\
&=&(\lnot (y=0)\,\vee \,\,z=xy_{0})\,\wedge \,(\lnot (y\neq 0)\,\vee
\,\,y\neq 1\,\vee \,\,z=x(y_{0}-1)) \\
&=&(y\neq 0\,\vee \,\,z=xy_{0})\,\wedge \,(y=0\,\vee \,y\neq 1\,\vee
\,\,z=x(y_{0}-1)) \\
&=&(\underbrace{y\neq 0\,\vee \,y=0\,}_{true}\vee \,y\neq 1\,\vee
\,\,z=x(y_{0}-1))\,\wedge \\
&&(z=xy_{0}\,\vee \,\cancel{y=0}\,\vee \,\,y\neq 1\,\vee \,\,z=x(y_{0}-1)) \\
&=&(\,y\neq 1\,\vee \,\cancelto{y_{0}=y_{0}-1}{xy_{0}=x(y_{0}-1)}) \\
F_{3}(G) &=&(y=0\,\Rightarrow \,z=xy_{0})\,\wedge \,(y\neq 0\,\Rightarrow
\,wlp(z:=z+x;y:=y-1,F_{2}(G))) \\
&=&(y=0\,\Rightarrow \,z=xy_{0})\,\wedge \,(y=0\,\vee
\,wlp(z:=z+x,wlp(y:=y-1,\,\,y\neq 1\,\vee \,y_{0}=y_{0}-1)) \\
&=&(y\neq 0\,\vee \,\,z=xy_{0})\,\wedge \,(y=0\,\vee \,\cancelto{y\neq
2}{y-1\neq 1}\,\vee \,y_{0}=y_{0}-1)\, \\
&=^{\ast }&(y\neq 2\,\vee \,z=x(y_{0}-2))
\end{eqnarray*}

\textit{Guess:}\\%
\begin{equation*}
F_{i}(G)=(y\neq i\,\vee \,z=x(y_{0}-i)),\forall i\geq 0
\end{equation*}

Since $y\neq (i\geq 0)\,\Rightarrow \,y\neq 0\leq i\,\Rightarrow \,y<i$
thus,%
\begin{equation*}
F_{i}(G)=(y<i\,\vee \,z=x(y_{0}-i)),\forall i\geq 0.
\end{equation*}

\textit{Inductive proof:}%
\begin{equation*}
F_{0}(G)=(\lnot e\,\Rightarrow \,G),\quad F_{i+1}(G)=\,(e\,\Rightarrow
\,wlp(p,F_{i}(G))
\end{equation*}

\textit{Base case:}%
\begin{equation*}
F_{0}(G)=(y<0\,\vee \,z=x(y_{0}-0))=(y<0\,\vee \,z=xy_{0})
\end{equation*}

\textit{Induction step:}%
\begin{eqnarray*}
F_{i+1}(G) &=&\,(e\,\Rightarrow \,wlp(p,F_{i}(G))= \\
&=&^{\ast }(\lnot (y\neq 0)\,\vee \,wlp(z:=z+x,wlp(y=y-1,y<i\,\vee
\,z=x(y_{0}-i)))) \\
&=&^{\ast }(\cancel{y=0}\,\vee \,\cancelto{y<i+1}{y-1<i}\,\vee \,%
\cancelto{z=x(y_{0}-(i+1))}{z+x=x(y_{0}-i))}) \\
&=&(y<i+1\,\vee \,z=x(y_{0}-(i+1)))
\end{eqnarray*}

$\Rightarrow \quad wp(\ASS z0;\WHILE\ y\neq 0\DO\ASS z{z+x};\ASS y{y-1}%
\OD)=$%
\begin{eqnarray*}
&=&\forall i\,(i\geq 0\,\wedge \,F_{i}(G))= \\
&=&\forall i\,(i\geq 0\,\wedge \,(y<i\,\vee \,z=x(y_{0}-i))) \\
&=&\forall i\,(i\geq 0\,\wedge \,(y<(i\geq 0)\,\vee \,z=x(y_{0}-i))) \\
&=&\forall i\,(i\geq 0\,\wedge \,(\cancelto{\equiv y<0}{y<0\leq i}\,\vee
\,z=x(y_{0}-\underbrace{i}_{\text{since }y<0\leq i\;\Rightarrow \;[i/y]\text{
??}}))) \\
&=&\underbrace{\forall i\,(i\geq 0)}_{true}\,\wedge \,(y<0\,\vee
\,z=x(y_{0}-y)) \\
&=&y<0\,\vee \,z=x(y_{0}-y)
\end{eqnarray*}

$\Rightarrow \quad wlp(\ASS z0;\WHILE\ y\neq 0\DO\ASS z{z+x};\ASS y{y-1%
}\OD)=$%
\begin{eqnarray}
&=&wlp(z:=0,y<0\,\vee \,z=x(y_{0}-y))  \notag \\
&=&(y<0\,\vee \,0=x(y_{0}-y))=(y<0\,\vee \,x=0\,\vee \,%
\cancelto{y_{0}=y}{0=(y_{0}-y)})  \label{wlp_rule}
\end{eqnarray}

Thus,%
\begin{equation*}
\underbrace{(y<0\,\vee \,y_{0}=y)}_{\text{\textit{wlp}}}\,\Rightarrow
\,(y<0\,\vee \,x=0\,\vee \,y_{0}=y).
\end{equation*}

Another possebility to get the \textit{wlp} is, to calculate the 
\textit{wp} as normal and compute a weaker approximation of $F_{i}(G)$ by
using the derived rule (\ref{wlp_rule}) above.
