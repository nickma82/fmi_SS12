\Aufgabe[LTL - Monotonicity and Negation Normal Form \hfill\textbf{(2 Point)}]

\begin{enumerate}

\item Let $K_1 = (S, R, L_1)$ and  $K_2 = (S, R, L_2)$ be two Kripke structures with the same set of states $S$ and the same transition relation $R$ such that $L_1(s) \subseteq L_2(s)$ for all states $s \in S$.
    Prove that $K_1,s \models \phi$ implies $K_2,s \models \phi$ for all LTL formulae $\phi$ that do not contain negation.
    \emph{(Hint: prove this statement by structural induction)}.

\item

Exercise 1 defined the \emph{release operator} \textbf{R}.
Prove that the release operator enjoys the following equivalence using the semantics of LTL \emph{(Hint: use the semantics of LTL formulae)}:
\begin{displaymath}
    \phi \mathbf{R} \psi \equiv \neg(\neg \psi \mathbf{U} \neg \phi)
\end{displaymath}

\item

An LTL formula in \emph{negation normal form}, if
\begin{itemize}
\item all negations appear only in front of the atomic propositions,
\item only the logical operators \emph{true}, \emph{false}, $\vee$, and $\wedge$ are used, and
\item only the temporal operators \textbf{X}, \textbf{U}, and \textbf{R} are used.
\end{itemize}

Show that every LTL formula $\phi$ can be transformed into an equivalent formula $\psi$ that is in negation normal form.
\emph{(Hint: prove this statement by structural induction)}.

\end{enumerate}

b.)
\textbf{Solution:}

\bigskip

The release-operator $\mathbf{R}$ is defined by%
\begin{equation}
\varphi \mathbf{R}\psi \equiv \lnot (\lnot \varphi \mathbf{U\lnot }\psi )
\label{release_op}
\end{equation}

and its interpretation is as follows. The formula $\varphi \mathbf{R}\psi $
holds if $\psi $ remains true, i.e. always holds, up to and including a
moment $i\geq 0$, when $\varphi $ becomes valid (if there is such a
(required) moment such that $\psi $ get released).

Let $2^{AP}$ be an alphabeth over $AP$. Then formally, the equivalence (\ref%
{release_op}) can be proven by using \ the semantics of the LTL\ formulae
over all Kripke structures $M$ and a given infinite word $%
s=a_{0}a_{1}a_{2}\ldots \in (2^{AP})^{w}$ such that:%

\renewcommand\thefootnote{\fnsymbol{footnote}}

\begin{eqnarray*}
M,s &\vDash &\lnot \psi  \\
&\Leftrightarrow &\text{\qquad (by def. of the operator }\mathbf{U}\text{
and the negation of it)} \\
\lnot \Exists j &\geq &0\,(M,s_{j}\vDash \lnot \psi \AND\Forall %
i<j.M,s_{i}\vDash \lnot \varphi ) \\
&\Leftrightarrow &\text{\qquad (by semantics of negation)} \\
\lnot \Exists j &\geq &0\,(M,s_{j}\nvDash \psi \AND\Forall %
i<j.M,s_{i}\nvDash \varphi ) \\
&\Leftrightarrow &\text{\qquad (by the duality of }\exists \text{ and }%
\forall \text{)} \\
\Forall j &\geq &0\,\lnot (M,s_{j}\nvDash \psi \AND\Forall %
i<j.M,s_{i}\nvDash \varphi ) \\
&\Leftrightarrow &\text{\qquad (using the law of de Morgan)} \\
\Forall j &\geq &0\,(\lnot (M,s_{j}\nvDash \psi )\OR\lnot \Forall %
i<j.M,s_{i}\nvDash \varphi ) \\
&\Leftrightarrow &\text{\qquad (by semantics of negation)} \\
\Forall j &\geq &0\,(M,s_{j}\vDash \psi \OR\underbrace{\Exists %
i<j.M,s_{i}\vDash \varphi}_{(\dag)})\;^{(}\footnotemark^{)} \\
&\Leftrightarrow &\text{\qquad (by (}\footnotemark\text{))} \\
\Forall j &\geq &0.\,M,s_{j}\vDash \psi \quad \text{or\quad }(\Exists i\geq
0.M,s_{i}\vDash \varphi \AND\Forall k\leq i.M,s_{k}\vDash \varphi ).
\end{eqnarray*}%

\addtocounter{footnote}{-1}
\footnotetext{%
This is equivalent to the common definition of $\mathbf{R}$, s. t. 
\begin{equation*}
M.\pi \vDash \varphi \mathbf{R}\psi \quad \text{iff}\quad \Forall j\geq 0%
\text{, then for every }i<j\text{, s.t. } \underbrace{M,\pi ^{i}\nvDash \varphi \IMPL %
M,\pi ^{j}\vDash \psi. }_{\lnot (M,\pi ^{i}\nvDash \varphi )\OR M,\pi
^{j}\vDash \psi }
\end{equation*}%
}
\stepcounter{footnote}
\footnotetext{Since $j\geq 0$ and $i$ cannot be $< 0$, i.e. $\Exists i\geq 0$, then
$\Forall j\geq 0$ and $\Exists i < j$ implies that $\Forall i \leq j$. To avoid confusion,
we have to introduce a new variable $k$ and substitution, s.t. $i\rightarrow k$ and $j\rightarrow i$.}
