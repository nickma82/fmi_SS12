\Aufgabe[ACTL \& LTL\hfill\textbf{(2 Points)}]
\newcommand{\ACTL}{\mathbf{ACTL}}
\newcommand{\LTL}{\mathbf{LTL}}
\newcommand{\CTLstar}{\mathbf{CTL}^*}
\newcommand{\AP}{\mathbf{AP}}
\newcommand{\true}{\mathbf{true}}
\newcommand{\false}{\mathbf{false}}
\newcommand{\trans}{\mathsf{trans}}
Given an LTL formula $\varphi$ in Negation Normal Form, the following function 
$\mathsf{trans} : \LTL \rightarrow \ACTL$ translates $\varphi$ into an ACTL 
formula $\trans(\varphi)$ as follows:
\begin{center}
\begin{tabular}{l|l}
	$\varphi$ & $\trans(\varphi)$\\
	\hline
	$\true$ & $\true$\\
	$\false$ & $\false$\\
	$a$ & $a$\\
	$\neg a$ & $\neg a$\\
	$\varphi_1 \vee \varphi_2$ & $\trans(\varphi_1) \vee \trans(\varphi_2)$\\
	$\varphi_1 \wedge \varphi_2$ & $\trans(\varphi_1) \wedge \trans(\varphi_1)$\\
	$\mathbf{X} \varphi_1$ & $\mathbf{AX}~\trans(\varphi_1)$\\
	$\mathbf{F} \varphi_1$ & $\mathbf{AF}~\trans(\varphi_1)$\\
	$\mathbf{G} \varphi_1$ & $\mathbf{AG}~\trans(\varphi_1)$\\
	$\varphi_1 \mathbf{U} \varphi_2$ & $\mathbf{A}\left[ \trans(\varphi_1)\;\mathbf{U}\;\trans(\varphi_2) \right]$\\
	$\varphi_1 \mathbf{R} \varphi_2$ & $\mathbf{A}\left[ \trans(\varphi_1)\;\mathbf{R}\;\trans(\varphi_2) \right]$\\
\end{tabular}
\end{center}
The semantics of the ``release'' operator $\mathbf{R}$ is defined as follows:
$$M, \pi \models \varphi_1 \mathbf{R} \varphi_2 \stackrel{def}{\Leftrightarrow} \forall j \geq 0. \pi^j \models \varphi_2 \textnormal{ or } \exists i \geq 0. (\pi^i \models \varphi_1) \wedge (\forall k \leq i. \pi^k \models \varphi_2)$$

\paragraph{a)} Show that, for all $\LTL$ formulas $\varphi$ in negation normal form, the $\CTLstar$ 
formula $\trans(\varphi) \Rightarrow \mathbf{A}\varphi$ is a tautology.
\emph{Hint: Show this by showing that $M, s \models \trans(\varphi)$ implies 
$M, s \models \mathbf{A}\varphi$ for all $\LTL$ formulas $\varphi$ in negation normal form, all Kripke 
structures $M$ and all states $s$ in $M$.
Use induction over the structure of~$\varphi$.}\hfill\textbf{(1 Point)}


\paragraph{b)} Show that, in general, $M, s \models \varphi$ does not imply $M, s \models \trans(\varphi)$.
\emph{Hint: Give a Kripke structure $M$ and an $\LTL$ formula $\varphi$ such that $M, s \models \varphi$ and $M, s \not\models \trans(\varphi)$. Discuss why $M, s \models \varphi$ and $M, s \not\models \trans(\varphi)$ holds on $M$ and the given state $s$ in $M$.}\\{\color{white} whitespace}\hfill\textbf{(1 Point)}



