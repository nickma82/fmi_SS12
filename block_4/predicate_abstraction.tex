\Aufgabe[Predicate Abstraction\hfill\textbf{(1.5 Point)}]
Consider the following program:
\begin{verbatim}
void foo(int j, int z) {
  assume(z != 0);
  int i := j;
  while(z != 0) {
      i := i + z;
      if(z > 0)
          z--
      else
          z ++;
  };
  assert(i != j)
}
\end{verbatim}

The {\em assume statement} at the beginning of the function forces the parameter $z$ not to be 0 when the function is called.

\begin{enumerate}

 \item Argue in your own words why the assertion at the end of the program allways holds, i.e., why the error state can never be reached.

 \item Provide a labeled transition system for the given program.

 \item Provide an abstraction for the labeled transition system that uses the predicates $i = j$, $i < j$, $i > j$.

 \item Check whether the error state can be reached in the abstraction, if so state a trace to the error state and refine the abstraction with suitable predicates such that the error state is not reachable anymore.

\end{enumerate}
