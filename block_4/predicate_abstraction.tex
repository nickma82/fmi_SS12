\Aufgabe[Predicate Abstraction\hfill\textbf{(1.5 Point)}]
Consider the following program:
\begin{verbatim}
void foo(int j, int z) {
  assume(z != 0);
  int i := j;
  while(z != 0) {
      i := i + z;
      if(z > 0)
          z--
      else
          z ++;
  };
  assert(i != j)
}
\end{verbatim}

The {\em assume statement} at the beginning of the function forces the parameter $z$ not to be 0 when the function is called.

\begin{enumerate}

 \item Argue in your own words why the assertion at the end of the program allways holds, i.e., why the error state can never be reached.

 \item Provide a labeled transition system for the given program.

 \item Provide an abstraction for the labeled transition system that uses the predicates $i = j$, $i < j$, $i > j$.

 \item Check whether the error state can be reached in the abstraction, if so state a trace to the error state and refine the abstraction with suitable predicates such that the error state is not reachable anymore.

\end{enumerate}


$\ddot\smile$ \textbf{Solution}\\
\begin{enumerate}
\item

\item
\newpage
\textit{Labeled Transiton System (LTS):}
\begin{figure}[th]
\centering
\begin{tikzpicture}[->, >=stealth, line join=bevel]
   \pgfsetlinewidth{1bp}
   \pgfsetcolor{black}
   
   \tikzstyle{state} = [draw,shape=circle,minimum size=8mm,font=\footnotesize];
   \tikzstyle{accept_state} = [shape=circle, minimum size=8mm, accepting, font=\small, draw];
   \tikzstyle{every path} = [draw, thick];
   
   \node[state, initial] at (0,10) (1) {$1$};
   \node[state] at (0,8) (2) {$2$}; 
   \node[state] at (0,6) (3) {$3$}; 
   \node[state] at (0,4) (4) {$4$}; 
   \node[state] at (-2,2) (5) {$5$}; 
   \node[state] at (2,2) (6) {$6$}; 
   \node[state] at (5.2,8) (7) {$7$}; 
   \node[state] at (5.2,6) (8) [label=left:{\footnotesize END}] {$8$}; 
   \node[state] at (8.2,8) (9) [label=right:{\footnotesize ERROR}] {$9$}; 
   
   \draw (1) to node[auto] {$\mbox{\fontsize{10}{11}\selectfont $i := j;$}$} (2);
   \draw (2) to node[auto] {$\mbox{\fontsize{10}{11}\selectfont $assume(z \neq 0)$}$} (3);
   \draw (3) to node[auto] {$\mbox{\fontsize{10}{11}\selectfont $i := i+z;$}$} (4);
   \draw (4) to node[auto, swap] {$\mbox{\fontsize{10}{11}\selectfont $assume(z > 0)$}$} (5);
   \draw (4) to node[auto] {$\mbox{\fontsize{10}{11}\selectfont $assume(!(z > 0))$}$} (6);
   \draw (2) to node[auto] {$\mbox{\fontsize{10}{11}\selectfont $assume(z = 0)$}$} (7);
   \draw (7) to node[auto] {$\mbox{\fontsize{10}{11}\selectfont $assume(i \neq j)$}$} (8);
   \draw (7) to node[auto] {$\mbox{\fontsize{10}{11}\selectfont $assume(i = j)$}$} (9);
   \draw[->] (5) .. controls (-5,3) .. (2) node[pos=.4, left] {z := z-1;};
   \draw[->] (6) .. controls (5,3) .. (2) node[pos=.4, right] {z := z+1;};
   
   \end{tikzpicture}
\caption{{\protect\small LTS of the givenprogram.}}
\label{lts_program}
\end{figure}

\item
Notation + Graph folgen noch

\item
There is no reachable error trace in the abstraction. $\ddot\smile$
\end{enumerate}